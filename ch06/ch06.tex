\chapter{The magnetic field}

\section{Definition of the magnetic field}

A charge which is moving parallel to a current of other charges
experiences a force perpendicular to its own velocity. This emerged
from our analysis of the situation shown in Fig. 5.20 of the last
chapter. We can see it happening in the deflection of an electron
beam by a nearby current, as in Fig. 5.3. We have yet to show what
happens if we send our test charge moving in some other direction.
You can recognize already, however, the behavior that is usually
described by saying that there is a magnetic field around a wire that
carries current and a moving charge experiences a force in a magnetic
field. Just as we have defined the electric field vector $\vc{E}$ as the force
on unit test charge at rest, so we can now define another field by the
velocity-proportional part of the force that acts on a test charge in
motion. To state this precisely, suppose that, at some position and
time in a particular frame of reference, experiments show that the
force on a test charge $q$ moving with uniform velocity $\vc{v}$ is given by
\begin{equation}
  \vc{F} = q\vc{E}+\vc{v}\times\vc{B}
\end{equation}
where $\vc{E}$ and $\vc{B}$ are vectors that do not depend on v. If that holds,
we define $\vc{E}$ as the electric field at that place and we \emph{define} $\vc{B}$ as the
magnetic field at that place.\index{magnetic field!defined}

To justify the definition fully we would have to show, by experiment
or otherwise, that such a relation always can be found. We
have not completed that task, but we have shown that it holds in some
important and instructive special cases. In Sec. 5.8 we proved that
the force on the test charge is strictly independent of its velocity if
all other charges are stationary. In that case, Eq. 1 holds with $\vc{B} = 0$
everywhere. The argument we have just been through shows that
the particle moving parallel to the steady stream of charge carriers
in the ``wire'' experiences a force that is proportional to its velocity
and perpendicular thereto, as required by the cross-product relation
in Eq. 1. In fact we can say exactly what magnetic field vector $\vc{B}$
would be consistent with our findings and with the symmetry of the
system. It would be a vector perpendicular to the plane of the diagram,
that is, perpendicular to both the wire and the test-charge
% p. 185
velocity $\vc{v}$. To make Eq. 5.45 and Eq. 1 consistent, the magnitude
of $\vc{B}$ must be:
\begin{equation}
  B = \frac{2I}{rc}
\end{equation}

We have derived the magnetic field of a straight current by analyzing
only the \emph{electric} field of moving charges. Our derivation is 
incomplete, but in ways that are not essential. We ought to show that
a velocity-dependent force, of the same magnitude and proper 
direction, arises also if we make the test charge move radially toward or
away from the ``wire.'' That can be done in the same way, although
the mathematical details are a little more tedious if we try to carry
it through without any approximation. What is interesting is not the
mathematics of the coordinate transformations but the reason for
the existence of the force. We can discover that without doing any
calculations.

In Fig. 6.1a we have sketched the ''lab'' frame with the test particle
moving toward the wire. In the ``particle'' frame (Fig. 6.1b) where
the test charge is at rest, we see the line of positive and negative
charges advancing down toward the test charge, while the individual
charges move obliquely, the positives obliquely down to the right,
the negatives obliquely down to the left. It is rather like some
% p. 186
marching maneuver of a college band. What we \emph{hope} to find in this
frame is an electric field $\vc{E}'$ pointing 
to the left.\footnote{We are looking for a force which will appear to an observer in the lab frame as a
force perpendicular to the velocity of the test particle, just as the force in the first case
was perpendicular to the velocity, and to the left of the velocity vector as seen from
above.} Still, everything
appears quite symmetrical; how can such a field arise?

Consider one of the positive charges $P_1$ in the particle frame, and
a symmetrically located positive charge $P_2$. (If we were to sum the
contributions of all the positive charges, we could always group them
in such symmetrical pairs, one from the right half of the line, one
from the left.) The electric fields of these two charges are indicated
in Fig. 6.2a. It is now clear why they do not produce equal effects
at $q$. The relativistic contraction of the field, described by the factor
$(1-\beta^2\sin^2\theta')^{-3/2}$ in Eq. 5.12, makes the field of $P_2$ stronger at $q$
than the field of $P_1$. If the fields were spherically symmetrical they
would cancel, so far as an $x'$ component is concerned. Instead, $P_2$
wins, resulting in an $x'$ component to the left.

Now take two symmetrically located negative charges, $N_1$ and $N_2$,
as shown in Fig. 6.2b. In this case $N_1$ produces the stronger field and
that also results in a component of electric field in the negative $x'$
direction. On the other hand, the $y'$ components of the fields of the
positive and negative charges obviously cancel one another. We are
left with an electric field $\vc{E'}$ and consequently a force on $q$, which is
parallel to the wire, in the negative $x'$ direction.

We can now see plainly that the magnetic interaction between
moving charges is a relativistic effect. That was already indicated
by the appearance of the factor $vv_0/c^2$ in Eq. 5.44. In a world of
moving electric charges, magnetism would vanish if $c$ were infinite.
That being so, it seems odd that magnetic forces can be strong enough
to turn motor armatures and lift weights. We remarked earlier that
electrostatic forces between large-scale objects are not, as a rule, very
impressive. Why aren't magnetic forces very much weaker still?
The explanation is the almost exact electrical neutrality of bulk
matter. Our next example will provide a good illustration of this.
We shall calculate the force to be expected, in a practical case, between
two parallel wires carrying current.

Because this example involves us, for the first time, with a magnetic
field inside matter, we must interrupt the discussion to remark on
that point. Most metals, including copper (but not including iron),
and most substances generally, have almost no influence on a magnetic
field. We may assume that the field $\vc{B}$ inside copper is practically
% p. 187
the same as it would be in vacuum with the same currents
flowing. We shall go into these questions thoroughly in Chap. 10.
Until then we'll avoid iron.

Consider two copper wires 1 mm in diameter, 5 cm apart (Fig. 6.3).
A current is flowing in each wire; let us assume the average velocity
of the conduction electrons in each wire to be 0.5 cm/sec. As we
know, this average drift velocity is much smaller than the random
velocities of the electrons. If we assume there is one conduction
electron per copper atom, we can easily find the number of conduction
electrons in a 1-cm length of 1-mm diameter wire. It is about
$6 X 10^{20}$. (For this estimate we need the density of copper,
$8\ \zu{g}/\cmunit^3$; the atomic weight of copper, 64; and Avogadro's
number.) The linear density $\lambda$ of the negative charge that moves is
then
\begin{equation}
  \lambda = (6\times 10^{20})\times(4.8\times 10^{-10})\approx 3\times10^{11}\ \esu/\cmunit
\end{equation}
The positive charges are stationary. The product of $\lambda$ times the drift
velocity is $(3\times10^{11})(0.5) = 1.5 \times 10^{11}\ \esu/\sunit$, which is about
50 amperes of current.

Let us now find the force on an electron in the other wire, at
distance $r = 5\ \cmunit$. This electron also moves with the average speed
0.5 cm/sec, if the same current is flowing in that wire. Applying
Eq. 5.45, we have:
\begin{align}
\begin{split}
  F &= \frac{2q\lambda v v_0}{rc^2}
     = \frac{2(4.8\times10^{-10}\ \esu)(3\times10^{11}\ \esu/\cmunit)(0.5\ \cmunit/\sunit)^2}
            {(5\ \cmunit)(3\times10^{10}\ \cmunit/\sunit)^2} \\
    &\approx 1.6\times 10^{-20}\ \zu{dyne}
\end{split}
\end{align}
Since every conduction electron in the other wire experiences such
a force, the total force on the electrons in 1 cm of wire is 
$(6 \times 10^{20}) \times (1.6 \times 10^{-20})$,
or approximately 10 dynes/cm. We observe this as
a force on the wire itself; any momentum imparted to the electrons
is eventually transferred to the lattice that confines them. As for the
direction of the force, our analysis of the situation in Fig. 5.20 shows
that when charges of the same sign move parallel to one another, the
magnetic interaction tends to draw them together. Hence our copper
wires, if they carry current in the \emph{same} direction, will be drawn toward
one another with a force amounting to 10 dynes per centimeter of
wire. This will be true of any two parallel straight wires the same distance
apart carrying the equivalent current, 50 amperes.

Now 10 dynes is not an enormous force, but it is easily measurable.
Let us compare it with the electrostatic force that would act between
% p. 188
the wires if they carried uncompensated static charge of 
$3 \times 10^{11}\ \esu/\cmunit$ on each wire. The electric field of a line charge of density
$\lambda$ being $2\lambda/r$ (Eq. 1.26), the force that acts per centimeter of length
on a similar line charge $r$ cm away must be $2\lambda^2/r$. Substituting the
linear charge density we assumed for the electrons in the wire,
$3 \times 10^{11}\ \esu/\cmunit$, we obtain a force just $(c/v)^2$ times larger than the
magnetic interaction of 10 dynes/cm, namely $36 \times 10^{21}\ \zu{dyne}/\cmunit$,
or roughly $4 \times 10^{13}$ tons per cm! Magnetic phenomena would be relatively
inconspicuous, to put it mildly, if Nature did not provide two
kinds of charge capable of canceling off`` the electrostatic interaction
---but of course a world with only one kind of charge would be unimaginably
different anyway. In the atomic domain, where the full
Coulomb force between elementary particles comes into play, magnetic
effects do indeed take second place compared to electric inter-
actions. They are weaker, generally speaking, by just the factor we
should expect, the square of the ratio of the particle speed to the
speed of light.

The magnetic interaction between parallel currents depends only
on the products of the currents, not on the charge densities and
velocities separately. We introduced charge density and velocity into
the example above in order to make the comparison of the electrostatic
and the magnetic interactions. Ordinarily we are concerned
only with the total current in each wire, and the mechanism of charge
transport, whether by many charges moving slowly or few charges
moving rapidly, doesn't matter at all. Let conductor 1 in Fig. 6.4
carry a current $1_1$ esu/sec. Suppose that in conductor 2, at distance $r$,
there is an amount of charge $A$ esu/cm which is moving with
velocity $v_2$. We know that the force per unit length of conductor
2 must be $2 I_1\lambda v_2/rc^2$, but that is equivalent to:
\begin{equation}
  \text{Force per centimeter} = \frac{2I_1I_2}{rc^2}
\end{equation}
because $\lambda v_2$ is the current $I_2$ in the second conductor.

Expressed in terms of a magnetic field $\vc{B}$ the magnitude of the force
on a conductor which carries a steady current $I$, in a magnetic field $\vc{B}$
due to other currents, is $IB/c$ per centimeter of conductor. The force
is perpendicular to the conductor and to the magnetic field, so we
can write this statement in vector form as follows:
\begin{equation}
\boxed{
  \der\vc{F} = \frac{I}{c} \der\bell\times\vc{B}
}
\end{equation}
% p. 189
In Eq. 6, $\der\vc{F}$ is the force on a short section of a conductor of length $\der\bell$,
carrying a steady current $I$ measured in esu/sec. The vector $\der\bell$ points
in the direction of positive current flow. Equation 6 follows directly
from our definition of $\vc{B}$ in Eq. 1, and from the definition of current
in terms of charge transport. It holds for a conducting path of any
shape; we need only know the magnetic field vector $\vc{B}$ at every point
on the path. Note that you can get Eq. 6 by remembering that the
force on a charge element $\der q$ which is moving with speed $v$ is
$\der\vc{F} = \der q(\vc{v}/c) \times \vc{B}$. Then with $\vc{v} = \der\bell/\der t$
and $\der q = I \der t$, this turns at once into
\begin{equation}
  \der\vc{F} = \frac{I \der t}{c} \frac{\der\bell}{\der t} \times \vc{B} = \frac{I}{c} \der\bell\times\vc{B}
\end{equation}
So far we have chosen to measure $I$ in esu/sec. The practical unit
of current, the ampere, is $3 \times 10^9\ \esu/\sunit$.\footnote{We
remarked in Chap. 1 that such ``3's'' come from the speed of light. Now we
see how that happens. The exact number is of course 2.9979 \ldots
} If the current in a
straight wire is expressed in amperes, the magnetic field at a distance
$r$ cm from the wire is
\begin{equation}
  B\ (\text{gauss}) = \frac{2}{10}\frac{I\ (\text{amperes})}{r\ \text{(centimeters)}}
\end{equation}
The factor 2/10 is exact, the 1/10 coming from the ratio of $3 \times 10^9$
to $c$. Similarly, the force on a conductor carrying $I$ amperes in a field
of $\vc{B}$ gauss is
\begin{equation}
  \der\vc{F}\ (\text{dynes}) 
        = \frac{1}{10}I\ (\text{amperes})\; \der\bell \times \vc{B}\ (\cmunit\unitdot\zu{gauss})
\end{equation}
The unit for $B$ implied by these formulas, with force in dynes and
distance in centimeters, is called the \intro{gauss}. It has long been used by
physicists and engineers, and in spite of competition from other systems
of units, is likely to remain the most familiar unit of magnetic
field strength. The intensity of the earth's magnetic field, near the
surface of the earth, is of the order of one-half gauss. The field at
one of the wires, in Fig. 6.3, arising from the current in the other wire,
is about 2 gauss. The field between the poles of a large electromagnet
is likely to be measured in thousands of gauss (kilogauss). It is quite
easy to reach 10 to 20 kilogauss in an iron magnet, and 60 to 80 kilogauss
in a commercially available superconducting magnet. Fields
above $10^5$ gauss require rather special efibrt. On the sun's surface,
local regions (sunspots) display magnetic fields of hundreds of gauss,
% p. 190
and a few stars are known which have general surface fields greater
than one kilogauss. On the whole, the large-scale magnetic fields in
the universe are rather weak in absolute magnitude, by comparison.
A recent measurement (a special kind of spectroscopic measure-
ment) of the interstellar magnetic field in one small region in our
Galaxy gave around $10^{-5}$ gauss. On the scale of the Galaxy a field
of that magnitude is not insignificant. In fact the magnetic field plays
an essential, sometimes dominant, role in galactic dynamics. Thus a
unit of one gauss, about the magnitude of the only magnetic field
man had explored for centuries, is now roughly the geometric mean
between the important magnetic fields of cosmology, and the strongest
fields in the laboratory.

We should add that the quantity we have defined as $\vc{B}$ and have
been calling the magnetic field strength, is called in some books the
magnetic \emph{induction}.\index{magnetic induction|see{magnetic field}}

\section{Some properties of the magnetic field}

The magnetic field, like the electric field, is a device for describing
how charged particles interact with one another. If we say that the
magnetic field at the point (4.5,3.2,6.0) at 12:00 noon points horizontally
in the negative $y$ direction and has a magnitude of 5 gauss,
we are making a statement about the acceleration which a moving
charged particle at that point in space-time would exhibit. The remarkable
thing is that a statement of this form, giving simply a vector
quantity $\vc{B}$, says all there is to say. With it one can predict uniquely
the velocity-dependent part of the force on \emph{any} charged particle moving
with any velocity. It makes unnecessary any further description
of the other charged particles which are the sources of the field. In
other words, if two quite different systems of moving charges happen
to produce the same $\vc{E}$ and $\vc{B}$ at a particular point, the behavior of
any test particle at the point would be exactly the same in the two
systems. It is for this reason only that the concept of field. as an intermediary
in the interaction of particles, is useful. And it is for this
reason that we think of the field as an independent entity.

Is the field more, or less, real than the particles whose interaction,
as seen from our present point of view, it was invented to describe?
That is a deep question which we would do well to set aside for a long
time. People to whom the electric and magnetic fields were vividly
real---Faraday and Maxwell, to name two---were led thereby to new
insights and great discoveries. Let's view the magnetic field as concretely
as they did and learn some of its properties.

% p. 191

So far we have studied only the magnetic field of a straight wire
or filament of steady current. The field direction, we found, is everywhere
perpendicular to a plane containing the filament and the point
where the field is observed. The magnitude of the field is proportional
to $1/r$. The field lines are circles surrounding the filament, as
sketched in Fig. 6.5. The sense of direction of $\vc{B}$ is determined by our
previously adopted convention about the vector cross-product, by
the (arbitrary) decision to write the second term in Eq. 1 as
$+ (q/c) \vc{v} \times \vc{B}$, and by the \emph{physical fact} that a positive charge moving
in the direction of a positive current is attracted to it rather than
repelled. These are all consistent if we relate the direction of $\vc{B}$ to
the direction of the current that is its source in the manner shown
in Fig. 6.5. Looking in the direction of positive current, $\vc{B}$ lines curl
clockwise. Or you may prefer to remember it as a right-hand-thread
relation.

Let's look at the line integral of $\vc{B}$ around a closed path in this field.
(Remember that a similar inquiry in the case of the electric field of
a point charge led us to a simple and fundamental property of all
electrostatic fields.) Consider first the path $ABCD$ in Fig. 6.6a. This
lies in a plane perpendicular to the wire; in fact, we need only work
in this plane, for $\vc{B}$ has no component parallel to the wire. The line
integral of $\vc{B}$ around the path shown is zero, for the following reason.
Paths $BC$ and $DA$ are perpendicular to $\vc{B}$ and contribute nothing.
Along $AB$, $\vc{B}$ is stronger in the ratio $r_2/r_1$ than it is along $CD$; but $CD$
is longer than $AB$ by the same factor, for these two arcs subtend the
same angle at the wire. So the two arcs give equal and opposite 
contributions, and the whole integral is zero.

It follows that the line integral is also zero on any path that can
be constructed out of radial segments and arcs, such as the path in
Fig. 6.61). From this it is a short step to conclude that the line integral
is zero around \emph{any} path that does not enclose the wire. To smooth
out the corners we would only need to show that the integral around
a little triangular path vanishes to a sufficient order. The same step
was involved in the case of the electric field.

A path which does not enclose the wire is one like the path in
Fig. 6.6c, which, if it were made of string, could be pulled free. The
line integral around any such path is zero.

Now consider a circular path that encloses the wire, as in Fig. 6.6d.
Here the circumference is $2\pi r$, the field is $2I/cr$ and everywhere parallel
to the path, so the value of the line integral around this particular
path is $(2\pi r)(2I/rc)$, or $4\pi I/c$. We now claim that \emph{any} path looping
once around the wire must give the same value. Consider, for instance,
% p. 192
the crooked path $C$ in Fig. 6.6e. Let us construct the path $C'$
in Fig. 6.6f made from a path like $C$ and a circular path, but \emph{not}
enclosing the wire. The line integral around $C'$ must be zero and therefore
the integral around $C$ must be the negative of the integral around
the circle, which we have already evaluated as $4\pi I/c$ in magnitude.
The sign will depend in an obvious way on the sense of traversal of
the path. Our general conclusion is:
\begin{equation*}
\boxed{
  \int \vc{B}\cdot\der\vc{s} = \frac{4\pi}{c} \times \text{current enclosed by path}
}
\end{equation*}
Equation 10 holds when the path loops the current filament once.
Obviously a path which loops it $N$ times, like the one in Fig. 6.6g, will
give just $N$ times as big a result for the line integral.

% p. 193
The magnetic field, as we have emphasized before, depends only
on the rate of charge transport, the number of units of charge passing
a given point in the circuit, per second. Figure 6.7 shows a circuit
with a current of 5 mA (milliamperes), equivalent to $15 \times 10^6\ \esu/\sunit$.
The average velocity of the charge carriers ranges from
$10^{-4}\ \cmunit/\sunit$ in one part of the circuit to 0.8 times the speed of light
in another. The line integral of $\vc{B}$ over a closed path has the same
value around every part of this circuit, namely:
\begin{align*}
\begin{split}
  \int \vc{B}\cdot\der\vc{s} &= \frac{4\pi I}{c}
        = \frac{4\pi \:\times\: (15\times10^6\ \esu/\sunit)}{3\times10^{10}\ \cmunit/\sunit} \\
    &= 0.00628\ \zu{gauss}\cdot\cmunit
\end{split}
\end{align*}

What we have proved for the case of a long straight filament of
current clearly holds, by superposition, for the field of any system
of straight filaments. In Fig. 6.8 several wires are carrying currents
in different directions. If Eq. 10 holds for the magnetic field of one
of these wires, it must hold for the total field which is the vector sum,
at every point, of the fields of the individual wires. That is a pretty
complicated field. Nevertheless, we can predict the value of the line
integral of $\vc{B}$ around the closed path in Fig. 6.8 merely by noting
which currents the path encircles, and in which sense.

However, we are interested in other things than long straight wires.
We want to understand the magnetic field of any sort of current dis-
tribution---for example, that of a current flowing in a closed loop. It
turns out that these more general fields \emph{obey exactly the same law},
Eq. 10. The line integral of $\vc{B}$ around a bent wire is equal to that
around a long straight wire carrying the same current. This goes
beyond anything we have so far deduced, so we must look on it here
as a postulate confirmed by the experimental tests of its con-
sequences.

To state the law in the most general way, we must talk about volume
distributions of current. A general steady current distribution
is described by a volume current density $\vc{J}(x,y,z)$ which varies from
place to place but is constant in time. A current in a wire is merely
a special case in which $\vc{J}$ has a large value within the wire but is zero
elsewhere. We discussed volume distribution of current in Chap. 4,
where we noted that for time-independent currents, $\vc{J}$ has to satisfy
the continuity equation, or conservation-of-charge condition,
\begin{equation}
  \div\vc{J} = 0
\end{equation}

% p. 194

Take any closed curve $C$ in a region where currents are flowing.
The total current enclosed by $C$ is the flux of $\vc{J}$ through the surface
spanning $C$, that is, the surface integral $\int_S\vc{J}\cdot\der\vc{a}$ over this surface $S$
(see Fig. 6.9). A general statement of the relation in Eq. 10 is
therefore:
\begin{equation}
  \int_C \vc{B}\cdot\der\vc{s} = \frac{4\pi}{c} \int_S \vc{J}\cdot\der\vc{a}
\end{equation}
Let us compare this with Stokes' theorem, which we developed in
Chap. 2:
\begin{equation}
  \int_C \vc{F}\cdot\der\vc{s} = \int_S(\curl \vc{F})\cdot\der\vc{a}
\end{equation}
fCF - ds : fS(curl F) - da (14)
We see that a statement equivalent to Eq. 13 is this:
\begin{equation}
\boxed{
  \curl\vc{B} = \frac{4\pi\vc{J}}{c}
}
\end{equation}
This is the simplest and most general statement of the relation between
the magnetic field and the moving charges which are its source.

However, Eq. 15 is not enough to determine $\vc{B}(x,y,z)$, given
$\vc{J}(x,y,z)$, for many different vector fields could have the same curl.
We need to complete it with another condition. We had better think
about the divergence of $\vc{B}$. Going back to the magnetic field of a
single straight wire, we observe that the divergence of that field is
zero. You can't draw a little box anywhere, even one enclosing the
wire, which will have a net outward or inward flux. It is enough to
note that the boxes $V_1$ and $V_2$ in Fig. 6.10 have no net flux and can
shrink to zero without developing any. For this field then, $\div B = 0$,
and hence also for all superpositions of such fields. Again we postulate
that the principle can be extended to the field of any distribution
of currents, so that a companion to Eq. 12 is the condition
\begin{equation}
\boxed{
  \div\vc{B} = 0
}
\end{equation}

Equations 15 and 16 together very nearly determine $\vc{B}$ uniquely
if $\vc{J}$ is given. If we had two different fields $\vc{B}_1(x,y,z)$ and $\vc{B}_2(x,y,z)$
both satisfying Eqs. 15 and 16, their difference $\vc{B}_1-\vc{B}_2$ would be a
field with zero divergence and zero curl everywhere. Such a field is
simply a constant vector $\vc{B}_0$, the same at all points in space. So except
for the possible addition of a constant field pervading all space, the
conditions $\curl\vc{B}=4\pi\vc{J}/c$, and $\div \vc{B} = 0$, uniquely determine the
% p. 195
magnetic field of a given distribution of currents. It is interesting
to compare these with their counterparts in the case of the electrostatic
field. There we had the conditions
\begin{equation}
  \div\vc{E} = 4\pi\rho \qquad \text{and} \qquad \curl\vc{E}=0
\end{equation}

In the case of the electric field, however, we began with Coulomb's
law, which expressed directly the contribution of each charge to the
electric field at any point. Here we shall have to work our way back
to some relation of that type.\footnote{The student may wonder why we couldn't have started from some equivalent of
Coulomb's law for the interaction of currents. The answer is that a piece of a current
filament, unlike an electric charge, is not an independent object that can be physically
isolated. You cannot perform an experiment to determine the field from part of a
circuit: if the rest of the circuit isn't there. the current can't be steady without violating
the continuity condition.} We shall do so by means of a \emph{potential
function}.

\section{Vector potential}

We found that the scalar potential function $\pot(x,y,z)$ gave us a
simple way to calculate the electrostatic field of a charge distribution.
If there is some charge distribution $\rho(x,y,z)$, the potential at any
point $(x_1,y_1,z_1)$ is given by the volume integral
\begin{equation}
  \pot(x_1,y_1,z_1) = \int \frac{\rho(x_2,y_2,z_2)\;\der v_2}{r_{12}}
\end{equation}
The integration is extended over the whole charge distribution, and
$r_{12}$ is the magnitude of the distance from $(x_2,y_2,z_2)$ to $(x_1,y_1,z_1)$. The
electric field $\vc{E}$ is obtained as the negative of the gradient of $\pot$:
\begin{equation}
  \vc{E} = -\grad \pot
\end{equation}

The same trick won't work here, because of the essentially different
character of $\vc{B}$. The curl of $\vc{B}$ is not necessarily zero, so $\vc{B}$ can't, in
general, be the gradient of a scalar potential. However, we know
another kind of vector derivative, the curl. It turns out that we can
usefully represent $\vc{B}$, not as the gradient of a scalar function, but as
the curl of a vector function, like this:
\begin{equation}
\boxed{
  \vc{B} = \curl\vc{A}
}
\end{equation}

By obvious analogy, we call $\vc{A}$ the \intro{vector potential}. It is \emph{not}
obvious, at this point, why this tactic is helpful. That will have to
% p. 196
emerge as we proceed. It is encouraging that Eq. 16 is automatically
satisfied, since $\div \curl \vc{A} = 0$, for any 
$\vc{A}$.\footnote{If you are not familiar with this fact, refer back to Prob. 2.15.}
Or to put it another way,
the fact that $\div\vc{B} = 0$ presents us with the opportunity to represent
$\vc{B}$ as the curl of another vector function. Our job now is to discover
how to calculate $\vc{A}$, when the current distribution $\vc{J}$ is given, so that
Eq. 20 will indeed yield the correct magnetic field. In view of Eq. 15,
the relation between $\vc{J}$ and $\vc{A}$ is
\begin{equation}
  \curl(\curl\vc{A}) = \frac{4\pi\vc{J}}{c}
\end{equation}

Equation 21, being a vector equation, is really three equations.
We shall work out one of them, say the $x$-component equation. The
$x$ component of $\curl\vc{B}$ is $\partial B_z/\partial y-\partial B_y/\partial z$.
The $z$ and $y$ components
of $\vc{B}$ are
\begin{equation}
  B_z = \frac{\partial A_y}{\partial x} - \frac{\partial A_x}{\partial y}
  \qquad
  B_y = \frac{\partial A_x}{\partial z} - \frac{\partial A_z}{\partial x}
\end{equation}
Thus the $x$ component part of Eq. 21 reads
\begin{equation}
  \frac{\partial}{\partial y} \left(\frac{\partial A_y}{\partial x} - \frac{\partial A_x}{\partial y}\right)
 -\frac{\partial}{\partial z} \left(\frac{\partial A_x}{\partial z} - \frac{\partial A_z}{\partial x}\right)
 = \frac{4\pi J_x}{c}
\end{equation}
We assume our functions are such that the order of partial differentiation
can be interchanged. Taking advantage of that and rearranging
a little, we can write Eq. 23 in this way:
\begin{equation}
  -\frac{\partial^2 A_x}{\partial y^2} - \frac{\partial^2 A_x}{\partial z^2}
  +\frac{\partial}{\partial x} \left(\frac{\partial A_y}{\partial y}\right)
  +\frac{\partial}{\partial x} \left(\frac{\partial A_z}{\partial z}\right)
 = \frac{4\pi J_x}{c}
\end{equation}
To make the thing more symmetrical, 1et's add and subtract the same
term, $\partial^2 A_x/\partial x^2$, on the left:
\begin{equation}
  -\frac{\partial^2 A_x}{\partial x^2}
  - \frac{\partial^2 A_x}{\partial y^2}
  - \frac{\partial^2 A_x}{\partial z^2}
  +\frac{\partial}{\partial x} \left(
    \frac{\partial A_x}{\partial x}
   +\frac{\partial A_y}{\partial y}
   +\frac{\partial A_z}{\partial z}
  \right)
 = \frac{4\pi J_x}{c}
\end{equation}
We can now recognize the first three terms as the negative of the
Laplacian of $A_x$. The quantity in parentheses is the divergence of $\vc{A}$.
Now we have a certain latitude in the construction of $\vc{A}$. All we care
% p. 197
about is its curl; its divergence can be anything we like. Let us
\emph{require} that\footnote{To see why we are free to do this,
suppose we had an $\vc{A}$ such that $\curl\vc{A}=\vc{B}$, but
$\div \vc{A} = f(x,y,z) \ne 0$. Treating $f$ like the charge density $\rho$ in an electrostatic field, we
obviously can find a field $\vc{F}$, the analog of the electrostatic $\vc{E}$, such that $\div\vc{F} = f$.
But we know that the curl of such a field will be zero. Hence we could add $-\vc{F}$ to $\vc{A}$,
making a new field with the correct curl and zero divergence.}
\begin{equation}
  \div \vc{A} = 0
\end{equation}
In other words, among the various functions which might satisfy our
requirement that $\curl\vc{A} = \vc{B}$, let us consider as candidates only those
which also have zero divergence. Then that part of Eq. 25 drops
away and we are left simply with
\begin{equation}
  \frac{\partial^2 A_x}{\partial x^2}
  + \frac{\partial^2 A_x}{\partial y^2}
  + \frac{\partial^2 A_x}{\partial z^2}
 = -\frac{4\pi J_x}{c}
\end{equation}

$J_x$ is a known scalar function of $x$, $y$, and $z$. Let us compare Eq. 27 with
Poisson's equation, Eq. 2.70, which read:
\begin{equation}
  \frac{\partial^2 \pot}{\partial x^2}
  + \frac{\partial^2 \pot}{\partial y^2}
  + \frac{\partial^2 \pot}{\partial z^2}
 = -4\pi \rho
\end{equation}
The two equations are identical in form. We already know how to
find a solution to Eq. 28. The volume integral in Eq. 18 is the 
prescription. Therefore a solution to Eq. 27 must be given by
\begin{equation}
  A_x(x_1,y_1,z_1) = \frac{1}{c}\int \frac{J_x(x_2,y_2,z_2)\;\der v_2}{r_{12}}
\end{equation}
The other components must satisfy similar formulas. They can all
be combined neatly in one vector formula:
\begin{equation}
\boxed{
  \vc{A}(x_1,y_1,z_1) = \frac{1}{c}\int \frac{\vc{J}(x_2,y_2,z_2)\;\der v_2}{r_{12}}
}
\end{equation}

There is only one snag. We stipulated that $\div\vc{A} = 0$, in order to
get Eq. 27. How do we know the $\vc{A}$ given by Eq. 30 will have this
special property? Fortunately, it can be shown that it does.

As an example of a vector potential, consider a long straight wire
carrying a current $I$. In Fig. 6.11 we see the current coming toward
us out of the page, flowing along the positive $z$ axis. We know what
the magnetic field of the straight wire looks like. The field lines are
circles, as sketched already in Fig. 6.5. A few are shown in Fig. 6.11.
% p. 198
The magnitude of $\vc{B}$ is $2I/cr$. Using a unit vector (in in the ``circum-
ferential'' direction we can write the vector $\vc{B}$ as
\begin{equation}
  \vc{B} = \frac{2I\phihat}{cr}
\end{equation}
Noting that the unit vector $\phihat$ is $-\sin\pot\xhat+\cos\pot\yhat$, we can write this
in terms of $x$ and $y$ as follows:
\begin{equation}
  \vc{B} = \frac{2I(-\sin\pot\xhat+\cos\pot\yhat)}{c\sqrt{x^2+y^2}}
         = \frac{2I}{c} \left(\frac{-y\xhat+x\yhat}{x^2+y^2}\right)
\end{equation}
One vector function $\vc{A}(x,y,z)$ that will satisfy $\grad\times\vc{A}=\vc{B}$ is the
following:
\begin{equation}
  \vc{A} = -\zhat \frac{I}{c} \ln(x^2+y^2)
\end{equation}
To verify this, we calculate the components of $\grad\times\vc{A}$:
\begin{align}
\begin{split}
  (\grad\times\vc{A})_x &= \frac{\partial A_z}{\partial y} - \frac{\partial A_y}{\partial z}
                         = \frac{-2Iy}{c(x^2+y^2)} \qquad (=B_x) \\
  (\grad\times\vc{A})_y &= \frac{\partial A_x}{\partial z} - \frac{\partial A_z}{\partial x}
                         = \frac{+2Ix}{c(x^2+y^2)} \qquad (=B_y) \\
  (\grad\times\vc{A})_z &= \frac{\partial A_y}{\partial x} - \frac{\partial A_x}{\partial y}
                         = 0 \qquad\qquad\qquad (=B_z) \\
\end{split}
\end{align}
Of course, this is not the only function that could serve as the vector
potential for this particular $\vc{B}$. To the $\vc{A}$ of Eq. 33 could be added any
vector function with zero curl. This all holds for the space outside
the wire. Inside the wire, $\vc{B}$ is different, so $\vc{A}$ must be different also.
It is not hard to find the appropriate vector potential function for
the interior of a solid round wire---see Prob. 6.13.

Incidentally, the $\vc{A}$ for our particular example above could not
have been obtained by Eq. 30. The integral would diverge owing to
the infinite extent of the wire. This may remind you of the difficulty
we encountered in Chap. 2 in setting up a scalar potential for the
electric field of a charged wire. Indeed the two problems are very
closely related, as we should expect from their identical geometry
and the similarity of Eqs. 30 and 18. We found (Eq. 2.19) that a suitable
scalar potential for the line charge problem
is $-\lambda\ln(x^2+y^2)+\text{\emph{arbitrary constant}}$.
This assigns zero potential to some arbitrary
point which is neither on the wire nor at infinite distance away. Both
that scalar potential and the vector potential of Eq. 33 are singular
at the origin and at infinity.

% p. 199

\section{Field of any current-carrying wire}

Figure 6.12 shows a loop of wire carrying current $I$. The vector
potential $\vc{A}$ at the point $(x_1,y_1,z_1)$ is given according to Eq. 30 by the
integral over the loop. For current confined to a thin wire we may
take as the volume element $\der v_2$ a short section of the wire of length $\der\bell$.
The current density $J$ is $I/a$, where $a$ is the cross-section area, and
$\der v_2 = a \der\bell$. Hence $\vc{J}\: \der v_2 = I\: \der\bell$, and if 
we make the vector $\der\bell$ point
in the direction of positive current, we can simply replace $\vc{J}\: \der v_2$ by
$I\: \der\bell$. Thus for a thin wire or filament, we can write Eq. 30 as a line
integral over the circuit:
\begin{equation}
  \vc{A} = \frac{1}{c} \int \frac{\der\bell}{r_{12}}
\end{equation}

To calculate $\vc{A}$ everywhere and then find $\vc{B}$ by taking the curl of $\vc{A}$
might be a long job. It will be more useful to isolate one contribution
to the line integral for $\vc{A}$, the contribution from the segment of
wire at the origin, where the current happens to be flowing in the
$x$ direction (Fig. 6.13). We shall denote the length of this segment
by $\der \ell$. Let $\der\vc{A}$ be the contribution of this part of the integral to $\vc{A}$.
Then at the point $(x,y,0)$ in the $xy$ plane, $\der\vc{A}$, which points in the
positive $x$ direction, is
\begin{equation}
  \der\vc{A} = \xhat \frac{(I/c)\der\ell}{\sqrt{x^2+y^2}}
\end{equation}
% p. 200
It is clear from symmetry that the contribution of this part of $\vc{A}$ to
$\curl \vc{A}$ must be perpendicular to the $xy$ plane. Denoting the corresponding
part of $\vc{B}$ by $\der\vc{B}$ we have
\begin{align}
\begin{split}
  \der\vc{B} = \curl(\der\vc{A}) = \zhat\left(-\frac{\partial A_x}{\partial y}\right)
        &= \zhat\frac{(I/c)\der\ell\cdot y}{(x^2+y^2)^{3/2}} \\
        &= \zhat\frac{(I/c)\:\der\ell\:\sin\pot}{r^2} 
\end{split}
\end{align}

With this result we can free ourselves at once from a particular
coordinate system. Obviously all that matters is the relative orientation
of the element $\der\bell$ and the radius vector $\vc{r}$ from that element to
the point where the field $\vc{B}$ is to be found. The contribution to $\vc{B}$
from any short segment of wire $\der\bell$ can be taken to be a vector
perpendicular to the plane containing $\der\bell$ and $\vc{r}$, of magnitude
$I \der\ell \sin \pot/r^2c$, where $\pot$ is the angle between $\der\bell$ and $\vc{r}$. This can be
written compactly using the cross-product, and is illustrated in
Fig. 6.14.
\begin{equation}
  \der\vc{B} = \frac{I\:\der\bell\times \rhat}{cr^2}
\end{equation}
% p. 201
If you are familiar with the rules of the vector calculus, you can take
a short cut from Eq. 35 to Eq. 38. Writing $\der\vc{B} = \grad \times \der\vc{A}$, with
$\der\vc{A} = I\: \der\bell/cr$, we treat $\grad$ as a vector, reversing the order of the 
cross-product and changing the sign. Here $\der\bell$ is a constant, so that $\grad$
operates only on $1/r$, otherwise we couldn't get away with this! We
recall that $\grad( 1 /r) = -\rhat/r^2$ (as in going from the Coulomb potential
to the Coulomb field). Thus:
\begin{align}
\begin{split}
  \der\vc{B} &= \grad\times \frac{I\:\der\bell}{cr} = -\frac{I}{c}\der\bell\times\grad\left(\frac{1}{r}\right) \\
             &= -\frac{I}{c} \der\bell\times\left(-\frac{\rhat}{r^2}\right)
                   = \frac{I\:\der\bell\times\rhat}{cr^2}
\end{split}
\end{align}

Historically, Eq. 38 is known as the Biot-Savart law.\index{Biot-Savart law} The meaning
of Eq. 38 is that if $\vc{B}$ is computed by integrating over the \emph{complete
circuit}, taking the contribution from each element to be given by this
formula, the resulting $\vc{B}$ will be correct. As we remarked in the footnote
at the end of Sec. 6.2, the contribution of part of a circuit is not
physically identifiable. In fact, Eq. 38 is not the only formula that
could be used to get a correct result for $\vc{B}$---to it could be added any
function which would give zero when integrated around a closed
path.

We seem to have discarded the vector potential as soon as it performed
one essential service for us. Indeed, it is often easier, as a
practical matter, to calculate the field of a current system directly,
now that we have Eq. 38, than to find the vector potential first. We
shall practice on some examples in the next section. However, the
vector potential is important for deeper reasons. For one thing, it
has revealed to us a striking parallel between the relation of the electrostatic
field $\vc{E}$ to its sources, electric charges. and the relation of the
magnetic field $\vc{B}$ to steady currents. Its greatest usefulness lies ahead,
in the study of time-varying fields, and electromagnetic radiation.

\section{Fields of rings and coils}

A current filament in the form of a circular ring of radius $b$ is shown
in Fig. 6.15a. We could predict without any calculation that the
magnetic field of this source must look something like Fig. 6.15b,
where we have sketched some field lines in a plane through the axis
of symmetry. The field as a whole must be rotationally symmetrical
about this axis, the $z$ axis in Fig. 6.l5a, and the field lines themselves
must be symmetrical with respect to the plane of the loop, the
% p. 202
$xy$ plane. Very close to the filament the field will resemble that near
a long straight wire, since the distant parts of the ring are there relatively
unimportant.

It is easy to calculate the field on the axis, using Eq. 38. Each element
of the ring of length $\der\ell$ contributes a $\der\vc{B}$ perpendicular to $\vc{r}$. We
need only include the $z$ component of $\der\vc{B}$, for we know the total field
on the axis must point in the $z$ direction,
\begin{equation}
  \der B_z = \frac{I\:\der\ell}{cr^2} \cos\theta = \frac{I\:\der\ell}{cr^2} \frac{b}{r}
\end{equation}
Integrating over the whole ring, we have simply $\int\:\der\ell = 2\pi b$, so the
field on the axis at any point $z$ is
\begin{equation}
  B_z = \frac{2\pi b^2I}{cr^3} = \frac{2\pi b^2I}{c(b^2+z^2)^{3/2}}
           \qquad (\text{field on axis})
\end{equation}
At the center of the ring, $z=0$, the magnitude of the field is
\begin{equation}
  B_z = \frac{2\pi I}{cb}
           \qquad (\text{field at center})
\end{equation}

The cylindrical coil of wire shown in Fig. 6.16a is usually called a
solenoid. We assume the wire is closely and evenly spaced so that
% p. 203
the number of turns in the winding, per centimeter length along the
cylinder, is a constant, $n$. Now the current path is actually helical,
but if the turns are many and closely spaced we can ignore this and
regard the whole solenoid as equivalent to a stack of current rings.
Then we can use Eq. 41 as a basis for calculating the field at any point
such as the point z, on the axis of the coil. Take first the contribution
from the current ring included between radii from the point $z$ making
angles $\theta$ and $\theta+\der\theta$ with the axis. The length of this segment of the
solenoid, shaded in Fig. 6.16b, is $r\der\theta/\sin\theta$, and it is therefore equivalent
to a ring carrying a current $Inr \der\theta/ \sin \theta$. Since $r = b/ \sin\theta$, we
have, for the contribution of this ring to the axial field:
\begin{equation}
  \der B_z = \frac{2\pi b^2}{cr^3}\frac{Inr\:\der\theta}{\sin\theta}
            = \frac{2\pi I n}{c} \sin\theta\:\der\theta
\end{equation}
Carrying out the integration between the limits $\theta_1$ and $\theta_2$ gives
\begin{equation}
  B_z = \frac{2\pi I n}{c} \int_{\theta_1}^{\theta_2} \sin\theta\:\der\theta
      = \frac{2\pi I n}{c}\:(\cos\theta_1 - \cos\theta_2)
\end{equation}

We have used Eq. 44 to make a graph, in Fig. 6.17, of the field
strength on the axis of a coil the length of which is four times its
diameter. The ordinate is the field strength $B_z$ relative to the field
% p. 204
strength in a coil of infinite length with the same number of turns
per centimeter and the same current in each turn. For the infinite
coil, $\theta_1 = 0$ and $\theta_2 = \pi$, so
\begin{equation}
  B_z = \frac{2\pi I n}{c} \qquad (\text{infinitely long solenoid})
\end{equation}
At the center of the ``four-to-one'' coil the field is very nearly as large
as this, and it stays pretty nearly constant until we approach one of
the ends.

Figure 6.18 shows the magnetic field lines in and around a coil of
these proportions. Note that some field lines actually penetrate the
% p. 205
winding. The cylindrical sheath of current is a surface of discontinuity
for the magnetic field. Of course, if we were to examine the
field Very closely in the neighborhood of the wires, we would not find
any infinitely abrupt kinks, but we would find a very complicated,
ripply pattern around and through the individual wires.

It is quite possible to make a long solenoid with a \emph{single} turn of a
thin wide ribbonlike conductor, as in Fig. 6.19. To this our calculation
and the diagram in Fig. 6.18 apply exactly, the quantity $nI$ being
merely replaced by the current per centimeter flowing in the sheet.
Now the change in direction of a field line that penetrates the wall
occurs entirely within the thickness of the sheet, as suggested in the
inset in Fig. 6.19.

We could have found the field of the infinitely long solenoid without
going through the analysis leading to Eq. 45. In the infinite
% p. 206
solenoid, clearly, nothing can vary with $z$, the coordinate in the axial
direction. The field must be parallel to $\zhat$ everywhere. Consider the
line integral of $\vc{B}$ around a rectangular path like the path $ABCDA$ in
Fig. 6.20. The horizontal sides give nothing. The side $CD$ had better
give zero also, for if the line integral on $CD$ has a finite value, the
integral along any other such leg, such as $C'D'$ would have to give
the same, and we would have magnetic field filling all space outside
the coil with constant intensity. We conclude that the field outside
must be zero.\footnote{Why couldn't such a solenoid produce a uniform field in the entire space outside?
After all, the infinite flat current sheet we are about to consider has a uniform field
filling the half space on each side of it. The solenoid, however, can be made as slender
as we please, and it would be strange indeed if a solenoid of vanishing diameter could
still cause a finite field everywhere. Perhaps you can think of a more conclusive
argument.} That leaves the line integral of $\vc{B}$ along $AB$ which is
$B_z\ell$ and the entire line integral must equal $4\pi/c$ times the enclosed
current. Hence $B_z\ell = (4\pi/c)nI\ell$, or $B_z=4\pi nI/c$, in agreement
with Eq. 45.

% p. 207
\section{Change in $\vc{B}$ at a current sheet}

In the example of Fig. 6.19 we had a solenoid constructed from a
single curved sheet of current. Let's look at something even simpler,
a flat. unbounded current sheet. You may think of this as a sheet
of copper of uniform thickness in which a current flows with constant
density and direction everywhere within the metal. In order to refer
to directions, let us locate the sheet in the $xz$ plane, and let the current
flow in the $x$ direction. As the sheet is supposed to be of infinite extent
with no edges, it is hard to draw a picture of it! We show a
broken-out fragment of the sheet in Fig. 6.21, in order to have something
to draw; you must imagine the rest of it extending over the
whole plane. The thickness of the sheet will not be very important,
finally, but we may suppose that it has some definite thickness $t$. If
the current density inside the metal is $J$ in $(\esu/\sunit)/\cmunit^2$ then every
centimeter of height, in the $z$ direction, includes a ribbon of current
amounting to $Jt$ $\esu/\sunit$. We call this 
the ``surface current density''\index{surface current density}\index{current density!surface}
or ``sheet current density,'' and use the symbol $\scrj$ to distinguish it from
the volume current density $\vc{J}$. The units of $\scrj$ are $(\esu/\sunit)/\cmunit$. If
we are not concerned with what goes on inside the sheet itself, $\scrj$ is
a useful quantity. It is $\scrj$ that determines the \emph{change} in the magnetic
field from one side of the sheet to the other, as we shall see.

The field in Fig. 6.21 is not merely that due to the sheet alone.
Some other field in the $z$ direction was present, from another source.
The total field, including the effect of the current sheet, is represented
by the $\vc{B}$ vectors drawn in front of and behind the sheet.

Consider the line integral of $\vc{B}$ around the rectangle 12341 in
Fig. 6.21. One of the long sides is in front of the surface, the other
behind it, with the short sides piercing the sheet. Let $B_z^+$ denote the
$z$ component of the magnetic field immediately in front of the sheet,
$B_z^-$ the $z$ component of the field immediately behind the sheet. We
mean here the field of \emph{all} sources that may be around, including the
sheet itself. The line integra1 of $\vc{B}$ around the long rectangle is simply
$\ell(B_z^+-B_z^-)$. (Even if there were some other source which caused
a field component parallel to the short legs of the rectangle, these
legs themselves can be kept much shorter than the long sides, since
we assume the sheet is thin, in any case, compared to the scale of any
field variation.) The current enclosed by the rectangle is just $\ell\scrj$.
Hence we have the relation $\ell(B_z^+-B_z^-)=4\pi\scrj\ell/c$, or
\begin{equation}
  B_z^+-B_z^- = \frac{4\pi\scrj}{c}
\end{equation}
% p. 208
A current sheet of density $\bscrj$ gives rise to a jump in that component
of $\vc{B}$ which is parallel to the surface and perpendicular to $\bscrj$. This
may remind you of the change in electric field at a sheet of charge.
There, the \emph{perpendicular} component of $\vc{E}$ is discontinuous, the magnitude
of the jump depending on the density of surface charge.

If the sheet is the only current source we have, then of course the
field is symmetrical about the sheet. $B_z^+$ is $2\pi\scrj/c$, and $B_z^-$ is
$-2\pi\scrj/c$. This is shown in Fig. 6.22a. Some other situations, in
which the effect of the current sheet is superposed on a field already
present from another source, are shown in Fig. 6.221) and c. Suppose
there are two sheets carrying equal and opposite surface currents, as
shown in cross section in Fig. 6.23, with no other sources around.
The direction of current flow is perpendicular to the plane of the
paper, out on the left and in on the right. The field between the sheets
is $4\pi\scrj/c$ and there is no field at all outside. Something like this is
found when current is carried by two parallel ribbons or slabs, close
together compared to their width, as sketched in Fig. 6.24. Often \emph{bus
bars}\index{bus bar} for distributing heavy currents in power stations are of this form.

The change in $\vc{B}$ takes place within the sheet, as we already remarked
in connection with Fig. 6.19. For the same $\scrj$, the thinner
the sheet, the more abrupt the transition. We looked at a situation
very much like this in Chaps. 1 and 2 when we examined the discontinuity
in the perpendicular component of $\vc{E}$ that occurs at a sheet
of surface charge. It was instructive then to ask about the force on
the surface charge, and we shall ask a similiar question here.

Consider a square portion of the sheet, 1 cm on a side. The current
included is equal to $\scrj$, the length of current path is 1 cm, and the
average field that acts on this current, assuming the current is uniformly
distributed through the thickness of the sheet, is $\frac{1}{2}(B_z^++B_z^-)$.
Therefore the force on this portion of the current distribution is
\begin{equation}
  \text{Force on 1 $\cmunit^2$ of sheet} = \frac{1}{2}(B_z^++B_z^-)\frac{\scrj}{c}
\end{equation}
In view of Eq. 46, we can substitute $(B_z^+-B_z^-)/4\pi$ for $\scrj/c$, so that
the force per square centimeter can be expressed in this way:
\begin{align}
\begin{split}
  \text{Force per $\cmunit^2$}
       & = \left(\frac{B_z^++B_z^-}{2}\right)\left(\frac{B_z^+-B_z^-}{4\pi}\right) \\
       &= \frac{1}{8\pi} \left[(B_z^+)^2-(B_z^-)^2\right]
\end{split}
\end{align}
% p. 209
The force is perpendicular to the surface and proportional to the
area, like the stress caused by hydrostatic pressure. To make sure
of the sign, we can figure out the direction of the force in a particular
case, such as that in Fig. 6.23. The force is \emph{outward} on each conductor.
It is as if the high field region were the region of high pres-
sure. We must remember, though, that it is only the component of
B parallel to the surface that counts in determining this force.

We have been considering an infinite flat sheet, but things are very
much the same in the immediate neighborhood of any curving
surface. Wherever the component of $\vc{B}$ parallel to the surface
changes from $B_1$ to $B_2$, from one side of the surface to the other, we
% p. 210
may conclude not only that there is a sheet of current flowing in the
surface, but that the surface must be under a perpendicular stress of
$(B_1^2-B_2^2)/8\pi$, measured in $\zu{dyne}/\cmunit^2$. This is one of the controlling
principles in \intro{magnetohydrodynamics}, the study of electrically
conducting fluids, a subject of interest both to electrical engineers
and to astrophysicists.

\section{How the fields transform}

A sheet of surface charge, if it is moving parallel to itself, constitutes
a surface current. If we have a uniform charge density of $\sigma$
on the surface, with the surface itself sliding along at speed $v$, the
surface current density is just $\scrj=\sigma v$. This simple idea will help us
to see how the electric and magnetic field quantities must change
when we transform from one inertial frame of reference to another.

Let's imagine two plane sheets of surface charge, parallel to the
$xz$ plane as in Fig. 6.25. Again, we show fragments of surfaces only,
in the sketch; the surfaces are really infinite in extent. In the inertial
frame $F$, with coordinates $x$, $y$, and $z$, the density of surface charge
is $\sigma$ on one sheet and $-\sigma$ on the other. In this frame the uniform
electric field $\vc{E}$ points in the positive $y$ direction, and Gauss's law
assures us, as usual, that its strength is:
\begin{equation}
  E_y = 4\pi\sigma
\end{equation}
In this frame $F$ the sheets are both moving in the positive $x$ direction
with speed $v_0$, so that we have a pair of current sheets. The
density of surface current is $\scrj_x=\sigma v_0$ in one sheet, the negative of
that in the other. As in the arrangements in Fig. 6.21, the field between
two such current sheets is
\begin{equation}
  B_z = \frac{4\pi\scrj_x}{c} = \frac{4\pi\sigma v_0}{c} 
\end{equation}

The inertial frame $F'$ is one that moves, as seen from $F$, with a speed
$v$ in the positive $x$ direction. \emph{What fields will an observer in $F'$
measure?} To answer this we need only find out what the sources look
like in $F'$.

In $F'$ the $x'$ velocity of the charge-bearing sheets is $v_0'$, given by
the velocity addition formula:
% p. 211
\begin{equation}
  v_0' = \frac{v_0-v}{1-v_0v/c^2} = c\frac{\beta_0-\beta}{1-\beta_0\beta}
\end{equation}
There is a different Lorentz contraction of the charge density, in this
frame, exactly as in our earlier example of the moving line charge,
in Sec. 5.9. Repeating the argument we used then, the density in
the rest frame of the charges themselves is $\sigma(1 - v_0^2/c^2)^{1/2}$, or $\sigma/\gamma_0$,
and therefore the density in the frame $F'$ is
\begin{equation}
  \sigma' = \sigma \frac{\gamma_0'}{\gamma_0}
\end{equation}
As usual, $\gamma_0'$ stands for $(1 - v_0'^2/c^2)^{-1/2}$. By means of Eq. 51 we
can eliminate $\gamma_0'$, expressing it in terms of $\beta_0$ and $\beta$, or $\gamma_0$ and $\gamma$.
When we do this, the result is:
\begin{equation}
  \sigma' = \sigma\gamma(1-\beta_0\beta)
\end{equation}
% p. 212
The surface current density in the frame $F'$ is therefore
\begin{equation}
  \scrj' = \sigma' v_0' = \sigma\gamma(1-\beta_0\beta)c\frac{\beta_0-\beta}{1-\beta_0\beta}
        = \sigma\gamma(v_0-v)
\end{equation}
We now know how the sources appear in frame $F'$, so we know what
the fields in that frame must be. In saying this, we are again invoking
the postulate of relativity. The laws of physics must be the same in
all inertial frames, and that includes the formulas connecting electric
field with surface charge density, and magnetic field with surface
current density. It follows then that:
\begin{align}
  E_y' &= 4\pi\sigma'
        = \gamma\left[4\pi\sigma-\left(\frac{4\pi\sigma v_0}{c}\right)\left(\frac{v}{c}\right)\right] \\
  B_z' &= \frac{4\pi}{c}\scrj'
        = \gamma\left[\frac{4\pi\sigma v_0}{c}-4\pi\sigma\left(\frac{v}{c}\right)\right] 
\end{align}
If we look back at the values of $E_y$, and $B_z$, in Eqs. 49 and 50, we see
that our result can be written as follows:
\begin{align*}
  E_y' &= \gamma(E_y - \beta B_z) \\
  B_z' &= \gamma(B_z - \beta E_y)
\end{align*}

If the sandwich of current sheets had been oriented parallel to the
$xy$ plane, instead of the $xz$ plane, we would have obtained relations
connecting $E_z'$ with $E_z$ and $B_y$, and $B_y'$ with $B_y$ and $E_z$. Of course they
would have the same form as the relations above, but if you trace
the directions through, you will find that there are differences of sign,
following from the rules for the direction of $\vc{B}$.

It remains to find how the field components in the direction of
motion change. We had already discovered in Sec. 5.5 that a longitudinal
component of $\vc{E}$ has the same magnitude in the two frames.
That this is true also of a longitudinal component of $\vc{B}$ can be seen
as follows. Suppose a longitudinal component of $\vc{B}$, a $B_x$ component
in the arrangement in Fig. 6.25, is produced by a solenoid around the
$x$ axis in frame $F$. The field strength inside a solenoid, as we know,
depends only on the current in the wire, $I$, which is charge per second,
and $n$, the number of turns of wire per centimeter of axial length.
In the frame $F'$ the solenoid will be Lorentz-contracted so the number
of turns per centimeter in that frame will be greater. But the current,
as reckoned by the observer in $F'$, will be reduced, since from his point
of view, the $F$ observer who measured the current by counting the
number of electrons passing a point on the wire, per second, was
using a slow-running watch. The time dilation just cancels the length
% p. 213
contraction in the product $nI$. Indeed any quantity of the dimensions
$(\text{longitudinal length}^{-1})\times(\text{time}^{-1})$ is unchanged in a Lorentz 
transformation. So $B_x' = B_x$.

Remember the point made early in Chap. 5, in the discussion
following Eq. 5.6: The transformation properties of the field are \emph{local}
properties. The values of $\vc{E}$ and $\vc{B}$ at some space-time point in one
frame must uniquely determine the field components observed in
any other frame at that same space-time point. Therefore the fact
that we have used an especially simple kind of source (the parallel
uniformly charged sheets) in our derivation in no way compromises
the generality of our result. We have in fact arrived at the general
laws for the transformation of all components of the electric and
magnetic field, of whatever origin or configuration.

We give below the full list of transformations. All primed
quantities are measured in the frame $F'$, which is moving in the positive
$x$ direction with speed $v$ as seen from $F$. Unprimed quantities
are the numbers which are the results of measurement in $F$. As usual,
$\beta$ stands for$v/c$ and $\gamma$ for $(1-\beta^2)^{-1/2}$.

% In the following, the split environment causes it not to compile.
\begin{framed}
\begin{align}
%\begin{split}
  E_x'&=E_x    &    E_y'&=\gamma(E_y-\beta B_z)   &    E_z'&=\gamma(E_z+\beta B_y) \\
  B_x'&=B_x    &    B_y'&=\gamma(B_y-\beta E_z)   &    B_z'&=\gamma(B_z+\beta E_y) 
%\end{split}
\end{align}
\end{framed}

The equations in the box confront us with an astonishing fact, their
symmetry with respect to $\vc{E}$ and $\vc{B}$. If the printer had mistakenly
interchanged $E$'s with $B$'s, and $y$'s with $z$'s, the equations would come
out exactly the same! Yet our previous View had been that magnetism
is a kind of ``second-order'' effect arising from relativistic
changes in the electric fields of moving charges. Certainly magnetic
phenomena as we find them in Nature are distinctly different from
electrical phenomena. The world around us is by no means symmetrical
with respect to electricity and magnetism. Nevertheless,
with the sources out of the picture, we find that the fields themselves,
$\vc{E}$ and $\vc{B}$, are connected to one another in a highly symmetrical way.

It appears too that the electric and magnetic fields are in some
sense aspects, or components, of a single entity. We can speak of the
\emph{electromagnetic} field,\index{electromagnetic field}
and we may think of $E_x$, $E_y$, $E_z$, $B_x$, $B_y$, and $B_z$ as
six components of the electromagnetic field. The \emph{same} field viewed
in different inertial frames will be represented by different sets of
values for these components, somewhat as a vector is represented by
different components in different coordinate systems rotated with
respect to one another. However, the electromagnetic field so
conceived is not a vector, mathematically speaking, but rather
% p. 214
something called a \intro{tensor}. The totality of the equations in the box
forms the prescription for transforming the components of such a
tensor when we shift from one inertial frame to another. We are not
going to develop that mathematical language here. In fact, we shall
return now to our old way of talking about the electric field as a vector
field, and the magnetic field as another vector field coupled to the
first in a manner to be explored further in Chap. 7. To follow up on
this brief hint of the unity of the electromagnetic field as represented
in four-dimensional space-time, you will have to wait for a more
advanced course.

The transformations of Eq. 58 predict a strikingly simple relation
in a certain class of cases. Suppose that in one frame, say the ``unprimed''
coordinates, the magnetic field $\vc{B}$ is zero everywhere. Then
the fields observed in the other frame are:
\begin{align}
%\begin{split}
    E_x'&=E_x   &   E_y'&=\gamma E_y         &   E_z'&=\gamma E_z \\
    B_x'&=0     &   B_y'&=\beta\gamma E_z    &   B_z'&=-\beta\gamma E_y
%\end{split}
\end{align}
This implies a certain relation between the electric and the magnetic
field everywhere in the ``primed'' frame, namely:
\begin{equation}
  B_x'=0 \qquad B_y'=\beta E_z' \qquad B_z'=-\beta E_y'
\end{equation}
Remembering that in this instance the velocity of the unprimed frame
is a vector in the $-\xhat'$ direction, as seen in the primed frame, we can
express the relation above as a vector cross-product, and thus obtain
a more general rule:
\begin{equation}
\boxed{
  \vc{B}' = \left(\frac{\vc{v}'}{c}\right) \times \vc{E}'
    \qquad (\text{if $\vc{B}=0$ everywhere in some frame})
}
\end{equation}
Here $\vc{v}'$ means the velocity, as observed from the primed frame, of
that special frame in which $\vc{B}$ happens to be everywhere zero.

In exactly the same way, we deduce from Eq. 58 that if $\vc{E} = O$
everywhere in one frame, which we shall call the unprimed frame.
then in another frame
\begin{equation}
\boxed{
  \vc{E}' = -\left(\frac{\vc{v}'}{c}\right) \times \vc{B}'
    \qquad (\text{if $\vc{E}=0$ everywhere in some frame})
}
\end{equation}
Here, as in Eq. 61, $\vc{v}'$ is the velocity of the unprimed frame (in this
case the one in which $\vc{E}$ is everywhere zero) as seen from the primed
% p. 215
frame. The parenthetical restrictions on Eqs. 61 and 62 are pretty
severe, of course. Often there will be \emph{no} frame in which $\vc{B}$ is zero
everywhere, and \emph{no} frame in which the electric charge density, and
therefore $\vc{E}$, is zero everywhere.

Because Eq. 61 involves only quantities measured in the same
frame of reference, it is easy to apply, whenever the restriction is
met, to fields that vary in 
space.\footnote{For spatially varying fields, the
meaning of Eq. 58 is $E'(x',y',z',t') = E(x,y,z,t)$, etc.
Thus if we want to compute the fields that will be observed everywhere in the primed
frame at the instant $t'$, we have to use for each point $(x',y',z')$ the time $t$ that goes with
$x'$, $y'$, $z'$, $t'$, as well as the $x$, $y$, and $z$ that
go with $x'$, $y'$, $z'$, $t'$. For instance, it is the
values of $B_z$ and $B_y$ at that time $t$, at that point $(x,y,z)$, that go on 
the right in the last of the equations in the box.}

A good example is the field of a point charge $q$ moving with constant
velocity, the problem studied in Chap. 5. Take the unprimed
frame to be the frame in which the charge is at rest. In this frame,
of course, there is no magnetic field. Equation 61 tells us that in the
``lab'' frame, where we find the charge moving with speed 12, there
must be a magnetic field perpendicular to the electric field and to the
direction of motion. We have already worked out the exact form of
the electric field in this frame: We know the field is radial from the
instantaneous position of the charge, with a magnitude given by
Eq. 5.12. The magnetic field lines must be circles around the direction
of motion, as indicated crudely in Fig. 6.26. When the velocity
of the charge is high, so that $\gamma\gg 1$, the radial ``spokes'' which are the
electric field lines are folded together into a thin disk. The circular
magnetic field lines are likewise concentrated in this disk. The magnitude
of $\vc{B}$ is then nearly equal to the magnitude of $\vc{E}$. That is, the
magnitude of the magnetic field in gauss is almost exactly the same
as the magnitude of the electric field, at the same point and instant
of time, in statvolts/cm.

We have come a long way from Coulomb's law in the last two
chapters. Yet with each step we have only been following out consistently
the requirements of relativity and of the invariance of electric
charge. We can begin to see that the existence of the magnetic
field and its curiously symmetrical relationship to the electric field
is a necessary consequence of these general principles. We remind
the reader again that this was not at all the historical order of discovery
and elucidation of the laws of electromagnetism. One aspect
of the coupling between the electric and magnetic fields which is
implicit in Eq. 58 came to light in Michael Faraday's experiments
with changing electrical currents. That was seventy-five years before
anyone thought of writing down equations like those in the box.

% p. 217
\section{Rowland's experiment}

As we remarked in Sec. 5.9, it was not obvious a hundred years
ago that a current flowing in a wire and a moving electrically charged
object are essentially alike as sources of magnetic field. The unified
view of electricity and magnetism which was then emerging from
Maxwell's work suggested that any moving charge ought to cause a
magnetic field, but experimental proof was hard to come by.

That the motion of an electrostatically charged sheet produces a
magnetic field was first demonstrated by Henry Rowland, the great
American physicist renowned for his perfection of the diffraction
grating. Rowland made many ingenious and accurate electrical
measurements, but none that taxed his experimental virtuosity as
severely as the detection and measurement of the magnetic field of
a rotating charged disk. The field to be detected was something like
$10^{-5}$ of the earth's field in magnitude---a formidable experiment,
even with today's instruments! In Fig. 6.27, you will see a sketch
of Rowland's apparatus and a reproduction of the first page of the
paper in which he described his experiment. Ten years before Hertz'
discovery of electromagnetic waves, Rowland's result gave 
independent, if less dramatic, support to Maxwell's theory of the electromagnetic
field.

\iffalse

\section{Electric conduction in a magnetic field: the Hall effect}

When a current flows in a conductor in the presence of a magnetic
field, the force (q/c)v x B acts directly on the moving charge car-
riers. Yet we observe a force on the conductor as a whole. Let's
see how this comes about. Figure 6.28a shows a section of a metal
bar in which a steady current is flowing. Driven by a field E, electrons
are drifting to the left with average speed 6, which has the same
meaning as the 17 in our discussion of conduction in Chap. 4. The
conduction electrons are indicated, very schematically, by the white
dots. The black dots are the positive ions which form the rigid framework
of the solid metal bar. Since the electrons are negative, we have
a current in the y direction. The current density J and the field E
are related by the conductivity of the metal, a, as usual: J = (IE.
There is no magnetic field in Fig. 6280 except that of the current
itself, which we shall ignore. Now an external field $\vc{B}$ in the x direction
is switched on. The state of motion immediately thereafter is

% p. 218
% p. 219
\/ll/I-41.1: u .llII(/ Lvnusnvuvuu 1. vvvuw 54!]

shown in Fig. 6.28b. The electrons are being deflected downward.
But since they cannot escape at the bottom of the bar, they simply
pile up there, until the surplus of negative charge at the bottom of the
bar and the corresponding excess of positive charge at the top creates
an electric field E, in which the upward force, of magnitude eE,,
exactly balances the downward force (e/c)t'>B. In the steady state
(which is attained very quickly!) the average motion is horizontal
again, and there existsin the interior of the metal this transverse
electric field E,, as observed in coordinates fixed in the metal lattice
(Fig. 6.28c). This field causes a downward force on the positive
ions. That is how the force, (- e/c)v x B, on the electrons is passed
on to the solid bar. The bar, of course, pushes on whatever is holding
it---or, if nothing is holding it, accelerates downward.

The existence of the transverse field E, can be demonstrated electrically
in a very direct way (Fig. 6.29). Wires are connected to
points $P_1$ and $P_2$ on opposite edges of the bar, the junction points
being carefully located so that they are at the same potential when
current is flowing in the bar and the field $\vc{B}$ is zero. The wires are
connected to a galvanometer. After the magnetic field $\vc{B}$ is switched
on, a steady current flows in this circuit, showing that $P_1$ and $P_2$ are
no longer at the same potential. In fact, $P_1$ is positive, relative to $P_2$
in the system we have just described.

This effect was discovered in 1879 by E. H. Hall, who was studying
under Rowland at Johns Hopkins. In those days no one understood
the mechanism of conduction in metals. The electron itself was un-
known. The Hall effect proved to be a most instructive phenomenon.

 

% p. 220
For modern research on electric conduction, especially in semicon-
ductors, Hall-effect measurements are indispensable.

We have seen that the magnetic field of a current, as well as the
force on a current-carrying conductor in an external field, are quite
independent of the details of the conduction process. The Hall
effect, however, reveals something about the charge carriers. Notice
that if the current in the bar in Fig. 6.28 had been due to positive
charges moving to the right, a transverse field E, of opposite direction
would have developed. Thus the sign of the ``Hall potential differ-
ence'' between $P_1$ and $P_2$ tells us whether the charge carriers are positive
or negative. Quantitatively, the magnitude of the transverse
field E, is determined by the equality

\begin{equation}
\end{equation}

qEt : q B 01' E5 : B 

ole:
ole:

On the other hand, the average carrier velocity 6 is related to the
current density J by

\begin{equation}
\end{equation}

J : nqfi (64)

where n is 'the number of charge carriers per unit volume, with charge
q in each. Combining Eqs. 63 and 64, we can eliminate 6:

\begin{equation}
\end{equation}

E, = JB (65)

E,, J, and $\vc{B}$ can be measured in an arrangement like that in
Fig. 6.29. E, is just the potential difference between $P_1$ and $P_2$ divided
by the width of the bar; J the total current divided by the cross-
sectional area. Thus we can infer (1 /nqc). This factor is called the
``Hall coefficient'' for the substance. For many metals the Hall
coefficient has about the value you would expect if there is roughly
one conduction electron per atom, with the sign of the effect indicating
that the charge carriers are indeed negative. But some metals
have Hall coefficients of the opposite sign! This remained a baffiing

paradox until it was explained by the quantum theory of electrons
in metals.

\fi
