\chapter{Magnetic fields in matter}

\section{How various substances respond to a magnetic field}

Imagine doing some experiments with a very intense magnetic
field. To be definite, suppose we have built a solenoid of 10-cm inside
diameter, 40 cm long, like the one shown in Fig. 10.1 Its outer
diameter is 40 cm, most of the space being filled with copper 
windings. This coil will provide a steady field of 30,000 gauss at its center
if supplied with 400 kilowatts of electrical power --- and something
like 30 gallons of water per minute, to carry off the heat. We mention
these practical details to show that our device, though nothing 
extraordinary, is a pretty respectable laboratory magnet. The field
strength at the center is nearly 105 times the earth's field, and probably
5 or 10 times stronger than the field near any iron bar magnet
or horseshoe magnet you may have experimented with. The field
will be fairly uniform near the center of the solenoid, falling, on the
axis at either end, to roughly half its central value. It will be rather
less uniform than the field of the solenoid in Fig. 6.18. since our coil
is equivalent to a ``nested'' superposition of solenoids with 
length-diameter ratio varying from 4/1 to 1/1. In fact, if we analyze our
coil in that way and use the formula (Eq. 6.44) which we derived for
the field on the axis of a solenoid with a single-layer winding, it is
not hard to calculate the axial field exactly. A graph of the field
strength on the axis, with the central field taken as 30 kilogauss, is
included in Fig. 10.1. The intensity just at the end of the coil is
18,000 gauss, and in that neighborhood the field is changing with a
gradient of approximately 1700 gauss/cm.

Let's put various substances into this field and see if a force acts
on them. Generally, we do detect a force. It vanishes when the
current in the coil is switched off. We soon discover that the force
is strongest not when our sample of substance is at the center of the
coil where the magnetic field $B_z$ is strongest, but when it is located
near the end of the coil where the gradient, $\der B_z/\der z$, is large. From
now on let us support each sample just inside the upper end of the
coil. Figure 10.2 shows one such sample, contained in a test tube
suspended by a spring which can be calibrated to indicate the extra
force caused by the magnetic field. Naturally we have to make a
``blank'' experiment with the test tube and suspension alone, to allow
for the magnetic force on everything other than the sample.

We find in such an experiment that the force on a particular
substance --- meta1lic aluminum, for instance --- is proportional to the
mass of the sample and independent of its shape, as long as the
sample is not too large. (Experiments with a small sample in this
coil show that the force remains practically constant over a region a
few centimeters in extent, inside the end of the coil; if we use samples
no more than 1 to $2\ \cmunit^3$ in volume they can be kept well within this
% p. 354
region.) We can express our quantitative results, for a given 
substance, as so many dynes force per gram of sample, under the conditions
$B_z = 18,000\ \zu{gauss}$, $\der B_z/\der z = 1700\ \zu{gauss}/\cmunit$.

But first the qualitative results, which are a bit bewildering: For
a large number of quite ordinary pure substances the force observed,
although easily measurable, seems after all our effort to provide an
intense magnetic field, ridiculously small. It is ten or twenty dynes
per gram, typically, not more than a few percent of the weight of the
sample. It is upward for some samples, downward for others. This
has nothing to do with the \emph{direction} of the magnetic field, as we can
verify by reversing the current in the coil. Instead, it appears that
some substances are always pulled in the direction of \emph{increasing} field
intensity, others in the direction of \emph{decreasing} field intensity, irrespective
of the field direction.

\begin{table} % makes it floating
\caption{Force on 1 g sample in magnetic field, with $B_z=18,000\ \zu{gauss}$, $\der B_z/\der z = 1700\ \zu{gauss}/\cmunit$}\label{table:magnetic-forces}
\begin{tabular}{lllr@{}l} % final col is to align on decimal point; see http://tex.stackexchange.com/a/44769/6853
\hline
\quad & \emph{Substance} & \emph{Formula} & \emph{Force (dyne)\footnote{Direction of force:
downward $+$, upward $-$. All measurements made at temperature of $20\degunit$ except for
liquid nitrogen, $78\ \zu{K}$, and liquid oxygen, $90\ \zu{K}$.
}} \\
\hline
\multicolumn{2}{l}{\emph{Diamagnetic}} \\
  & Water          & $\zu{H}_2\zu{O}$      & $-22$ \\
  & Copper         & Cu                    & $-2$&$.6$ \\ % align on decimal point
  & Lead           & Pb                    & $-37$ \\
  & Sodium chloride & NaCl                 & $-15$ \\
  & Quartz         & $\zu{SiO}_2$          & $-16$ \\
  & Sulfur         & S                     & $-16$ \\
  & Diamond        & C                     & $-16$ \\
  & Graphite       & C                     & $-110$ \\
  & Liquid nitrogen & $\zu{N}_2$           & $-10$ \\
\multicolumn{2}{l}{\emph{Paramagnetic}} \\
  & Sodium         & Na                    & $+20$ \\
  & Aluminum       & Al                    & $+17$ \\
  & Copper chloride & $\zu{CuCl}_2$        & $+280$ \\
  & Nickel sulfate & $\zu{NiSO}_4$         & $+830$ \\
  & Liquid oxygen  & $\zu{O}_2$            & $+7,500$ \\
\multicolumn{2}{l}{\emph{Ferromagnetic}} \\
  & Iron           & Fe                    & $+400,000$ \\
  & Magnetite      & $\zu{Fe}_3\zu{O}_4$   & $+120,000$
\end{tabular}
\end{table}



% p. 355
We do find some substances that are attracted into the coil with
considerably greater force. For instance, copper chloride crystals
are pulled downward with a force of 280 dynes per gram of sample.
Liquid oxygen behaves spectacularly in this experiment; it is pulled
into the coil with a force nearly eight times its weight. In fact, if we
were to bring an uncovered flask of liquid oxygen up to the bottom
end of our coil, the liquid would be lifted right out of the flask.
(Where do you think it would end up?) Liquid nitrogen, on the
other hand, proves to be quite unexciting; a gram of liquid nitrogen
is pushed away from the coil with the feeble force of 10 dynes. In the
table we have listed some results that one might obtain in such an
experiment. The substances, including those already mentioned,
have been chosen to suggest, as best one can with a sparse sampling,
the wide range of magnetic behavior we find in ordinary materials.

As you know, a few substances, of which the most familiar is
metallic iron, seem far more ``magnetic'' than any others. In the
table we give the force that would act on a 1 g piece of iron put in
the same position in the field as the other samples. The force is
nearly a pound! (We would not have been so naive as to approach
our magnet with several grams of iron suspended in a test tube from
a delicate spring --- a different suspension would have to be used.)
Note that there is a factor of more than $10^5$ between the force that
acts on a gram of iron and the force on a gram of copper, elements
not otherwise radically different. Incidentally, this suggests that
reliable magnetic measurements on a substance like copper may
not be easy. A few parts per million contamination by metallic iron
particles would utterly falsify the result.

There is another essential difference between the behavior of the
iron and the magnetite and that of the other substances in the table.
Suppose we make the obvious test, by varying the field strength of
the magnet, to ascertain whether the force on a sample is proportional
to the field. For instance, we might reduce the solenoid current
by half, thereby halving both the field intensity $B_z$ and its gradient
$\der B/\der z$. We would find, in the case of every substance above
iron in the table, that the force is reduced to \emph{one-fourth} its former
value, whereas the force on the iron sample, and that on the 
magnetite, would be reduced only to one-half or perhaps a bit less.
Evidently the force, under these conditions at least, is proportional
to the square of the field strength for all the other substances listed,
but nearly proportional to the field strength itself for Fe and $\zu{Fe}_3\zu{O}_4$.

% p. 356

It appears that we may be dealing with several different phenomena
here, and complicated ones at that. As a small step toward
understanding, we can introduce some classification.

First, those substances which are feebly repelled by our magnet,
water, sodium chloride, quartz, etc., are called \emph{diamagnetic}.\index{diamagnetism}\index{magnetic properties of materials!diamagnetism} The
majority of inorganic compounds and practically all organic compounds
are diamagnetic. It turns out, in fact, that diamagnetism
is a property of \emph{every} atom and molecule. When the opposite behavior
is observed, it is because the diamagnetism is outweighed by
a different and stronger effect, one that leads to attraction.

Substances which are attracted toward the region of stronger field
are called \emph{paramagnetic}.\index{paramagnetism}\index{magnetic properties of materials!paramagnetism}
In some cases, notably the metals such as
Al, Na, and many others, the paramagnetism is not much stronger
than the common diamagnetism. In other materials such as the
$\zu{NiSO}_4$ and the $\zu{CuCl}_2$ on our list, the paramagnetic effect is much
stronger. In these substances also, it increases as the temperature
is lowered, leading to quite large effects at temperatures near absolute
zero. The increase of paramagnetism with lowering temperature
is responsible in part for the large force recorded for liquid
oxygen. If you think all this is going to be easy to explain, observe
that copper is diamagnetic while copper chloride is paramagnetic.
but sodium is paramagnetic while sodium chloride is diamagnetic.

Finally, substances that behave like iron and magnetite are called
\emph{ferromagnetic}.\index{ferromagnetism}\index{magnetic properties of materials!ferromagnetism}
In addition to the common metals of this class, iron,
cobalt, and nickel, quite a number of ferromagnetic alloys and
crystalline compounds are known. Indeed current research in 
ferromagnetism is steadily lengthening the list.

In this chapter we have two tasks. One is to develop a treatment
of the large-scale phenomena involving magnetized matter, in which
the material itself is characterized by a few parameters and the
experimentally determined relations among them. It is like a treatment
of dielectrics based on some observed relation between electric
field and bulk polarization. We sometimes call such a theory 
\emph{phenomenological}. Our second task is to try to understand, at least in
a general way, the atomic origin of the various magnetic effects.
Even more than dielectric phenomena the magnetic effects, once
understood, reveal some basic features of atomic structure.

One general fact stands out in the table. Very little energy, on the
scale of molecular energies, is involved in diamagnetism and 
paramagnetism. Take the extreme example of liquid oxygen. To pull
% p. 357
1 g of liquid oxygen away from our magnet would require an
amount of work of the order of magnitude of 7,500 dynes times a
distance of a few centimeters (since the field strength falls off substantially
in a few centimeters' distance). Let us say, roughly,
50,000 ergs. Now it takes about 300 times that much energy (0.4
calories, or 1.6 joules) to raise the temperature of 1 gram of liquid $\zu{O}_2$
by one degree, and it takes more than 30,000 times as much energy
to vaporize the liquid, that is, to separate the molecules one from
another. Whatever may be happening in liquid oxygen at the molecular
level as a result of the magnetic field, it is apparently a very minor
afl``air in terms of energy.

We know that even strong magnetic fields have practically no
influence on chemical processes, and that goes for biochemistry too.
You can put your hand and forearm (not the one with the wrist
watch!) into our 30 kilogauss solenoid, without any significant
sensation or consequence. It is hard to predict whether your arm
will be paramagnetic or diamagnetic, but the force on it will be no
more than a fraction of an ounce, in any case. Generations of mice
have been bred and raised in strong magnetic fields without significant
changes. Nor, as of this writing, have other biological experiments
turned up anything remarkable in the way of magnetic effects
on chemical processes.\footnote{
This is not to say that small effects are always inconsequential. After all, an argument
like the one we just used would show that gravity is energetically unimportant
on a molecular scale, yet trees on a hillside grow vertically. This presumably involves
the gross force on a biologicalunit larger than molecular size. Indeed, a somewhat
similar ``tropism" has been experimentally demonstrated in the case of seedlings growing
in a strongly inhomogeneous magnetic field. We are not suggesting, either, that
the magnetic properties of molecules are uninteresting to the biochemist. On the
contrary. the intermediates in chemical reactions are sometimes detectable, even 
identifiable, by their magnetic behavior. But this is quite a different thing from the influence
of an external field \emph{on} a chemical process. By the way, if you put your head in a
strong magnetic field and shake it, you will ``taste'' electrolytic currents in your mouth
--- simply evidence of the induced electromotive force.} This is not surprising. In its interaction
with matter, the magnetic field plays a role utterly different from that
of the electric field. Because atoms and molecules are made of
slowly moving electric charges, electric forces overwhelmingly
dominate the molecular scene.

% p. 358

\section{The absence of magnetic ``charge''}

The magnetic field outside a magnetized rod such as a compass
needle looks very much like the electric field outside an electrically
polarized rod, a rod that has an excess of positive charge at one end,
negative charge at the other (Fig. 10.3). It is conceivable that the
magnetic field has sources which are related to it the way electric
charge is related to the electric field. Then the ``north'' pole of the
compass needle would be the location of an excess of one kind of
``magnetic charge'' and the south pole would be the location of an
excess of the opposite kind. We might call ``north charge'' positive
and ``south charge'' negative, with magnetic field directed from positive
to negative, a rule like that adopted for electric field and electric
charge. Historically, that is how our convention about the positive
direction of magnetic field was established\footnote{
In Chap. 6, remember, we established the positive direction of $\vc{B}$ by reference to a
current direction (direction of motion of positive charge) and a right-hand rule. Now
``north pole'' means ``north-seeking pole'' of the compass needle. To this day we do
not understand why the earth's magnetic polarity should be one way rather than the
other. Franklin's designation of ``positive'' electricity had nothing to do with any of
this. So the fact that it takes a right-hand rule rather than a left~hand rule to make
this all consistent is the purest accident.
} What we have called
magnetic charge has usually been called magnetic pole strength.

This idea is perfectly sound as far as it goes. It becomes even more
plausible when we recall that the fundamental equations of the electromagnetic
field are quite symmetrical in $\vc{E}$ and $\vc{B}$. Why, then,
should we not expect to find symmetry in the sources of the field?
With magnetic charge as a possible source of the static magnetic
field $\vc{B}$, we would have $\div\vc{B} = 4\pi\eta$, where $\eta$ stands for the density
of magnetic charge, in complete analogy to the electric charge
density $\rho$. Two positive magnetic charges (or north poles) of unit
strength, 1 cm apart, would repel one another with a force of l dyne,
and so on.\index{charge!magnetic}

The trouble is, that is not the way things are. Nature for some
reason has not made use of this opportunity. The world around us
appears totally asymmetrical in the sense that we find \emph{no magnetic
charges} at all. No one has ever observed an isolated excess of one
kind of magnetic charge --- an isolated north pole for example. If
such an object existed it could be recognized in several ways. It
would give rise to a magnetic field directed radially outward from
the object and decreasing as $1/r^2$ at large distances. Perhaps even
more striking, such an object would experience a force if placed in
% p. 360
a \emph{uniform} magnetic field. Put in a solenoid, the magnetic pole would
experience the maximum force at the center of the coil rather than
at the end. And unlike the force on an \emph{electrically} charged particle
\emph{moving} in a magnetic field, the force on a magnetic pole at rest would
be parallel to the field instead of perpendicular to it.

From a region of space containing an isolated magnetic pole there
would be a net flux of $\vc{B}$. The apparent total absence of such 
objects can thus be summarized by the statement that, instead of
$\div\vc{B} = 4\pi\eta$, we find simply
\begin{equation}
\boxed{
  \div\vc{B} = 0 \qquad \text{everywhere}
}
\end{equation}

Really \emph{everywhere}? Could it be that there are north and south
poles inseparably united in pairs of equal and opposite strength, and
so close together that we cannot physically explore the region between
the members of a pair? We have no reason to think so, but
it wouldn't make any difference, for the statement of Eq. 1 would
still hold wherever $\vc{B}$ itself had any meaning. There has been serious
speculation, however, that pairs of poles, like pairs of elementary
particles, might be created and fly apart in very energetic nuclear
events. Several recent searches for such particles, termed magnetic
monopoles, have detected none.\footnote{For an account of
the search for magnetic monopoles, see the article by Kenneth
Ford, ``Magnetic Monopoles,'' Sci.~American 209, 30 (December, 1963). The 
``asymmetry'' evident in the absence of magnetic charge is quite different from the electrical
asymmetry we are familiar with, the difference in character between negative and
positive particles. Electrons are the stable negative particles; positive charge is found
in the form of the proton, a much heavier particle. But we know this need not be
universal, for the \emph{antiparticles} are known to exist. All the evidence points to the possibility
of matter made up of positive electrons and negative protons; this ``antimatter''
would be the twin of the matter we find in our part of the universe. We have seen the
ingredients of an antimatter world in the laboratory --- but the ingredients of a 
``magnetic twin'' have not been seen. There is serious doubt that they exist at all; and
there is a strong argument that if they do exist they must be quite different, in certain
other respects, from electrically charged particles.} Whether they \emph{cannot} exist, and
if so why not, remains an open question. Should anyone ever discover
a monopole he will be entitled to add triumphantly after Eq. 1
the qualification ``. . . except at the location of this north (or south)
magnetic monopole that appears in my photographic plate (or
bubble-chamber picture, or counter record, etc.).'' But even that
would not affect the main conclusion: Ordinary matter is made up
of electric charges, and not magnetic charges.

% p. 361

We are forced to conclude that the only sources of the magnetic
field are electric currents. This takes us back to the hypothesis of
Ampere, his idea that magnetism in matter is to be accounted for by
a multitude of tiny rings of electric current distributed through the
substance.

\section{The field of a current loop}

A closed conducting loop lies in the $xy$ plane encircling the origin,
as in Fig. l0.4a. A steady current $I$, measured in esu/sec, flows
around the loop. We are interested in the magnetic field this current
creates --- not near the loop, but at distant points like $P_1$ in the figure.
We shall assume that $r_1$, the distance to $P_1$, is much larger than any
dimension of the loop. To simplify the diagram we have located $P_1$
in the $yz$ plane; it will turn out that this is no restriction. This is a
good place to use the vector potential. We shall compute first the
vector potential $\vc{A}$ at the location $P_1$, that is, $\vc{A}(0,y_1,z_1)$. From this
it will be obvious what the vector potential is at any other point
$(x,y,z)$ far from the loop. Then by taking the curl of $\vc{A}$ we shall get
the magnetic field $\vc{B}$.

For a current confined to a wire, we had, as Eq. 6.35:
\begin{equation}
  \vc{A}(0,y_1,z_1) = \frac{I}{c}\int_\text{Loop} \frac{\der\bell_2}{r_{12}}
\end{equation}
At that time we were concerned only with the contribution of a small
segment of the circuit; now we have to integrate around the entire
loop. Consider the variation in the denominator $r_{12}$ as we go around
the loop. If $P_1$ is far away, the first-order variation in $r_{12}$ depends
only on the coordinate $y_2$ of the segment $\der\bell_2$, and not on $x_2$. This
should be clear from the side view sketched in Fig. 10.4b. Thus,
neglecting quantities proportional to $(x_2/r_{12})^2$, we may treat $r_{12}$
and $r'_{12}$, which lie on top of one another in the side view, as equal.
And in general, to first order in the ratio 
$(\text{loop dimension}/\text{distance to $P_1$})$ we have
\begin{equation}
  r_{12} \approx r_1 - y_2\sin\theta
\end{equation}

Look now at the two elements of the path $\der\bell_2$ and $\der\bell_2'$, shown in
Fig. 10.4a. For these the $\der y_2$'s are equal and opposite, and as we
have already pointed out, the $r_{12}$'s are equal to first order. To this
% p. 363
order then, their contributions to the line integral will cancel, and
this will be true for the whole loop. Hence $\vc{A}$ at $P_1$ will not have a
$y$ component. Obviously it will not have a $z$ component, for the
current path itself has nowhere a $z$ component. The $x$ component
of the vector potential comes from the $\der x$ part of the path integral.
Thus
\begin{equation}
  \vc{A}(0,y_1,z_1) = \xhat\frac{I}{c}\int\frac{\der x_2}{r_{12}}
\end{equation}
Without spoiling our first-order approximation, we can turn Eq. 3
into
\begin{equation}
  \frac{1}{r_{12}} \approx \frac{1}{r_1}\left(1+\frac{y_2\sin\theta}{r_1}\right)
\end{equation}
and using this for the integrand we have
\begin{equation}
  \vc{A}(0,y_1,z_1) = \xhat\frac{I}{cr_1}\int \left(1+\frac{y_2\sin\theta}{r_1}\right)\der x_2
\end{equation}
In the integration $r_1$ and $\theta$ are constants. Obviously $\int\der x_2$ around
the loop vanishes. Now $\int y_2\der x_2$ around the loop is just the area
of the loop, regardless of its shape (see Fig. l0.4c). So we get finally
\begin{equation}
  \vc{A}(0,y_1,z_1) = \xhat\frac{I\sin\theta}{cr_1^2}\times(\text{area of loop})
\end{equation}

Here is a simple but crucial point: Since the \emph{shape} of the loop
hasn't mattered, our restriction of $P_1$ to the $yz$ plane can't make any
essential difference. Therefore we must have in Eq. 7 the general
result we seek, if only we \emph{state} it generally: The vector potential of
a current loop of any shape, at a distance $r$ from the loop which is
much greater than the size of the loop, is a vector perpendicular to
the plane containing $\vc{r}$ and the normal to the plane of the loop, of
magnitude
\begin{equation}
  A = \frac{Ia\sin\theta}{cr^2}
\end{equation}
where $a$ stands for the area of the loop.

This vector potential is symmetrical around the axis of the loop,
which implies that the field $\vc{B}$ will be symmetrical also. The explanation
is that we are considering regions so far from the loop that the
details of the shape of the loop have negligible influence. All loops
% p. 364
with the same $\text{\emph{current}} \times \text{\emph{area}}$ product produce the same far field.
We call the product $Ia/c$ the \intro{magnetic dipole moment} of the current
loop, and denote it by $\vc{m}$. The magnetic dipole moment is evidently
a vector, its direction being that of the normal to the loop, or that
of the vector $\vc{a}$, the directed area of the patch surrounded by the loop
\begin{equation}
  \vc{m} = \frac{I}{c}\vc{a}
\end{equation}
As for sign, we agree that the direction of $\vc{m}$ and the sense of positive
current flow in the loop are to be related by a right-hand-screw rule,
illustrated in Fig. 10.5. (The dipole moment of the loop in Fig. 10.4a
points downward, according to this rule.) The vector potential for
the field of a magnetic dipole $\vc{m}$ can now be written neatly with
vectors:
\begin{equation}
  \vc{A} = \frac{\vc{m}\times\rhat}{r^2}
\end{equation}
where $\rhat$ is a unit vector in the direction \emph{from} the loop \emph{to} the point for
which $\vc{A}$ is being computed. You can check that this agrees with
our convention about sign. Note that the direction of $\vc{A}$ must always
be that of the current in the \emph{nearest} part of the loop.

Figure 10.6 shows a magnetic dipole located at the origin, with the
dipole moment vector $\vc{m}$ pointed in the positive $z$ direction. To express
the vector potential at any point $(x,y,z)$, we observe that
$r^2 = x^2 + y^2 + z^2$, and $\sin\theta=\sqrt{x^2 + y^2}/r$. The magnitude $A$ of
the vector potential at that point is
\begin{equation}
  A = \frac{m\sin\theta}{r^2} = \frac{m\sqrt{x^2 + y^2}}{r^3}
\end{equation}
% p. 365
Since $\vc{A}$ is tangent to a horizontal circle around the $z$ axis, its components
are:
\begin{align}
\begin{split}
  A_x &= A \left(\frac{-y}{\sqrt{x^2+y^2}}\right) = \frac{-my}{r^3} \\
  A_y &= A \left(\frac{x}{\sqrt{x^2+y^2}}\right) = \frac{mx}{r^3} \\
  A_z &= 0
\end{split}
\end{align}

Let's evaluate $\vc{B}$ for a point in the $xz$ plane, by finding the components
of $\curl\vc{A}$ and then (not before!) setting $y = 0$.
\begin{align}
\begin{split}
  B_x &= (\grad\times\vc{A})_x 
       = \frac{\partial A_z}{\partial y} - \frac{\partial A_y}{\partial z} 
       = -\frac{\partial}{\partial z} \frac{mx}{(x^2+y^2+z^2)^{3/2}}
       = \frac{3mxz}{r^5} \\
  B_y &= (\grad\times\vc{A})_y
       = \frac{\partial A_x}{\partial z} - \frac{\partial A_z}{\partial x} 
       = \frac{\partial}{\partial z} \frac{-mx}{(x^2+y^2+z^2)^{3/2}}
       = \frac{3myz}{r^5} \\
  B_z &= (\grad\times\vc{A})_z
       = \frac{\partial A_y}{\partial x} - \frac{\partial A_x}{\partial y} \\
       &= m\left[
                 \frac{-2x^2+y^2+z^2}{(x^2+y^2+z^2)^{5/2}}
                +\frac{x^2-2y^2+z^2}{(x^2+y^2+z^2)^{5/2}}
          \right]\\
       &= \frac{m(3z^2-r^2)}{r^5}
\end{split}
\end{align}
In the $xz$ plane, $y = 0$, $\sin \theta = x/r$, and $\cos \theta = z/r$. The field components
at any point in that plane are thus given by:
\begin{align}
\begin{split}
  B_x &= \frac{3m\sin\theta\cos\theta}{r^3} \\
  B_y &= 0 \\
  B_z &= \frac{m(3\cos^2\theta-1)}{r^3} 
\end{split}
\end{align}

Now turn back to Sec. 9.3, where in Eq. 9.14 we had expressed
the components in the $xz$ plane of the field $\vc{E}$ of an electric dipole $\vc{p}$,
which was situated exactly like our magnetic dipole $\vc{m}$. The expressions
are identical. We have thus found that the magnetic field of a
small current loop has at remote points the same form as the electric
field of two separated charges. We already know what that field,
the electric dipole field, looks like. Figure 10.7 is an attempt to
suggest the three-dimensional form of the magnetic field $\vc{B}$ arising
from our current loop with dipole moment $\vc{m}$.

% p. 366

The magnetic field \emph{close} to a current loop is entirely different from
the electric field close to a pair of separated positive and negative
charges, as the comparison in Fig. 10.8 shows. Notice that between
the charges the electric field points down, while inside the current

ring the magnetic field points up, although the far fields are alike.
This reflects the fact that our magnetic field satisfies $\grad\cdot\vc{B}$
everywhere, even \emph{inside the source}. The magnetic field lines don't end.
By near and far we mean, of course, relative to the size of the current
loop or the separation of the charges. If we imagine the current ring
shrinking in size, the current meanwhile increasing so that the dipole
moment $m = Ia/c$ remains constant, we approach the infinitesimal
magnetic dipole, the counterpart of the infinitesimal electric dipole
described in Chap. 9.

\section{The force on a dipole in an external field}

Consider a small circular current loop of radius $r$, placed in the
magnetic field of some other current system, such as a solenoid. In
Fig. 10.9, a field $\vc{B}$ is drawn that is generally in the $z$ direction. It is
not a uniform field. Instead, it gets weaker as we proceed in the
$z$ direction; that is evident from the fanning out of the field lines.
Let us assume, for simplicity, that the field is symmetric about the
$z$ axis. Then it resembles the field near the upper end of the solenoid
in Fig. 10.1. The field represented in Fig. 10.9 does \emph{not} include the
magnetic field of the current ring itself. We want to find the force
on the current ring caused by the other field, which we shall call, for
want of a better name, the external field. The net force on the current
ring due to its \emph{own} field is certainly zero, so we are free to ignore
its own field in this discussion.

If you study the situation in Fig. 10.9 you will soon conclude that
there is a net force on the current ring. It arises because the external
field $\vc{B}$ has an \emph{outward} component $B_r$, everywhere around the ring.
Therefore if the current flows in the direction indicated, each element
of the loop, $\der\bell$, must be experiencing a downward force of magnitude
$I B_r\der\ell/c$. If $B_r$ has the same magnitude at all points on the
ring, as it must in the symmetrically spreading field assumed, the
total downward force will have the magnitude
\begin{equation}
  F = \frac{2\pi r I B_r}{c}
\end{equation}

% p. 368

Now $B_r$ can be directly related to the gradient of $B_z$. Since
$\div\vc{B} = O$ at all points, the net flux of magnetic field out of any
volume is zero. Consider the little cylinder of radius $r$ and height $\Delta z$
(Fig. 10.10). The outward flux from the side is $2\pi r(\Delta z)B_r$, and the
net outward flux from the end surfaces is 
\begin{equation}
  \pi r^2[-B_z(z)+B_z(z+\Delta z)]
\end{equation}
which to the first order in the small distance $\Delta z$ is $\pi r^2(\partial B_z/\partial z)\Delta z$.
Setting the total flux equal to zero:
$0 = \pi r^2(\partial B_z/\partial z)\Delta z+2\pi r B_r \Delta z$,
which gives us the relation
\begin{equation}
  B_r = -\frac{r}{2} \frac{\partial B_z}{\partial z}
\end{equation}
As a check on the sign, notice that according to Eq. 16, $B_r$ is positive
when $B_z$ is decreasing upward; a glance at the figure shows that to
be correct.

The force on the dipole can now be expressed in terms of the
gradient of the component $B_z$ of the external field:
\begin{equation}
  F = \frac{2\pi rI}{c}\cdot\frac{r}{2} \frac{\partial B_z}{\partial z}
    = \frac{\pi r^2 I}{c}\cdot \frac{\partial B_z}{\partial z}
\end{equation}
In the factor $\pi r^2 I/c$ we recognize the magnitude $m$ of the magnetic
dipole moment of our current ring. So the force on the ring can be
expressed very simply in terms of the dipole moment:
\begin{equation}
  F = m \frac{\partial B_z}{\partial z}
\end{equation}
We haven't proved it, but you will not be surprised to hear that for
small loops of any other shape the force depends only on the 
current-area product, that is, on the dipole moment. The shape doesn't
matter. Of course, we are discussing only loops small enough so
that only the first-order variation of the external field, over the span
of the loop, is significant.

Our ring in Fig. 10.9 has a magnetic dipole moment $\vc{m}$ pointing
upward and the force on it is downward. Obviously, if we could
reverse the current in the ring, thereby reversing $\vc{m}$, the force would
reverse its direction. The situation can be summarized this way:

\begin{itemize}
\item Dipole moment \emph{parallel} to the external field: The force acts in the direction
of \emph{increasing} field strength.

\item Dipole moment \emph{antiparallel} to the external field: The force acts in the
direction of \emph{decreasing} field strength.

\item \emph{Uniform} external field: \emph{Zero} force.
\end{itemize}

% p. 369

This is not the most general situation. The
moment $\vc{m}$ could be pointing at some odd angle with respect to the
field $\vc{B}$, and the different components of $\vc{B}$ could be varying, spatially,
in different ways. It is not hard to develop a formula for the force $\vc{F}$
that is experienced in the general case. It would be exactly like the
general formula we gave, as Eq. 9.22, for the force on an electric
dipole in a nonuniform electric field. That is, the $x$ component of
force on any magnetic dipole $\vc{m}$ is given by
\begin{equation}
  F_x = \vc{m}\cdot\operatorname{grad} B_x
\end{equation}
with corresponding formulas for $F_y$ and $F_z$.

In Eqs. 18 and 19 the force is in dynes, with the magnetic field
gradient in gauss/cm and the magnetic dipole moment $m$ given by
Eq. 9, $m = Ia/c$ where $I$ is in esu/s, $a$ in square centimeters, and $c$
in cm/s. There are several equivalent ways to express the units
of $m$. We shall adopt ``erg/gauss.'' As you can see from Eq. 18,
\begin{equation}
  m = \frac{\text{dyne}}{(\text{gauss}/\cmunit)}
    = \frac{\text{dyne}\unitdot\cmunit}{\text{gauss}}
    = \frac{\text{erg}}{\text{gauss}}
\end{equation}

Now we can begin to see what must be happening in the experiments
described at the beginning of this chapter. A substance
located at the position of the sample in Fig. 10.2 would be attracted
\emph{into} the solenoid if it contained magnetic dipoles \emph{parallel} to the field
$\vc{B}$ of the coil. It would be pushed \emph{out} of the solenoid if it contained
dipoles pointing in the opposite direction, antiparallel to the field.
The force would depend on the gradient of the axial field strength,
and would be zero at the midpoint of the solenoid. Also, if the total
strength of dipole moments in the sample were proportional to the
field strength $\vc{B}$, then in a given position the force would be proportional
to $\vc{B}$ times $\partial B/\partial z$, and hence to the square of the solenoid 
current. That is the observed behavior in the case of the diamagnetic
and the paramagnetic substances. It looks as if the ferromagnetic
samples must have possessed a magnetic moment nearly independent
of field strength, but we must set them aside for a special discussion
anyway.

How does the application of a magnetic field to a substance evoke
in the substance magnetic dipole moments with total strength proportional
to the applied field? And why should they be parallel to
the field in some substances, and oppositely directed in others? If
we can answer these questions we shall be on the way to understanding
the physics of diamagnetism and paramagnetism.

% p. 370
 
\section{Electric currents in atoms}

We know that an atom consists of a positive nucleus surrounded
by negative electrons. To describe it fully we would need the concepts
of quantum physics that you will be studying later in this
course. Fortunately, a simple and easily visualized model of an
atom can explain diamagnetism very well. It is a planetary model
with the electrons in orbits around the nucleus, like the model in
Bohr's first quantum theory of the hydrogen atom.

We begin with one electron moving at constant speed on a circular
path. Since we are not attempting here to explain atomic structure,
we shall not inquire into the reasons why the electron has this particular
orbit. We ask only, if it does move in such an orbit, what
magnetic effects are to be expected? In Fig. 10.1] we see the 
electron, visualized as a particle carrying a concentrated electric charge
$-e$, moving with speed $v$ on a circular path of radius $r$. In the middle
is a positive nuclear charge, making the system electrically neutral,
but the nucleus, because of its relatively great mass, moves so slowly
that its magnetic effects can be neglected.

At any instant, the electron and the positive charge would appear
as an electric dipole, but on the time average the electric dipole
moment is zero, producing no steady electric field at.a distance. We
discussed this point in Sec. 9.5. The \emph{magnetic} field of the system,
far away, is \emph{not} zero on the time average. Instead, it is just the field
of a current ring. For, as concerns the time average, it can't make
any diflerence whether we have all the negative charge gathered into
one lump, going around the track, or distributed in bits, as in
Fig. l0.11b, to make a uniform endless procession. The current is
the amount of charge that passes a given point on the ring, per
second. Since the electron makes $v/2\pi r$ revolutions per second, the
current, in esu/s if $e$ is in esu, is
\begin{equation}
  I = \frac{ev}{2\pi r}
\end{equation}
The orbiting electron is equivalent to a ring current of this magnitude
with the direction of positive flow opposite to $v$, as shown in
Fig. l0.llc. Its far field is therefore that of a magnetic dipole. of
strength
\begin{equation}
  m = \frac{\pi r^2 I}{c} = \frac{evr}{2c}
\end{equation}

Let us note in passing a simple relation between the magnetic
moment $\vc{m}$ associated with the electron orbit, and the orbital angular
% p. 371
momentum $\vc{L}$. The angular momentum is a vector of magnitude
$L = m_evr$, where $m_e$ denotes the mass of the 
electron,\footnote{We shall be dealing with
speeds $v$ much less than c, so $m_e$ stands for the rest mass,
$9.0\times10^{-28}\ \zu{g}$. Our choice of the symbol $\vc{m}$ for magnetic moment makes it 
necessary, in this chapter, to use a different symbol for the electron mass. For angular
momentum we choose the symbol $\vc{L}$, rather than $\vc{J}$ as used in Vol. 1, Chap. 6, because
$\vc{L}$ is traditionally used in atomic physics for orbital angular momentum, which is what
we here consider, and because we have been using $\vc{J}$ for 
current density.} and it points
downward if the electron is revolving in the sense shown in
Fig. 10.1la. Notice that the product $vr$ occurs in both $m$ and $L$.
With due regard to direction, we can write:
\begin{equation}
  \vc{m} = \frac{-e}{2m_ec}\vc{L}
\end{equation}
This relation involves nothing but fundamental constants, which
should make you suspect that it holds quite generally. Indeed that
is the case, although we shall not prove it here. It holds for elliptical
orbits, and it holds even for the rosettelike orbits that occur in a
central field that is not inverse-square. Remember the important
property of any orbit in a central field: Angular momentum is a constant
of the motion. It follows then, from the general relation expressed
by Eq. 22 (derived by us only for a special case), that where-
ever angular momentum is conserved, the magnetic moment also
remains constant in magnitude and direction. The factor
\begin{equation}
  \frac{-e}{2m_ec}
\end{equation}
is called the \intro{orbital magnetomechanical ratio}\index{magnetomechanical ratio}\index{gyromagnetic ratio|see{orbital magnetomechanical ratio}}
for the electron.\footnote{Many people use the term gyromagnetic ratio for this quantity. We prefer 
magnetomechanical ratio, as used in Vol. 1, Chap. 8.} The
intimate connection between magnetic moment and angular momentum
is central to any account of atomic magnetism.

Why don't we notice the magnetic fields of all the electrons orbiting
in all the atoms of every substance? The answer must be that
there is a mutual cancellation. In an ordinary lump of matter there
must be as many electrons going around one way as the other. This
is to be expected, for there is nothing to make one sense of rotation
intrinsically easier than another, or otherwise to distinguish any
unique axial direction. There would have to be something in the
structure of the material to single out not merely an axis, but a \emph{sense
of rotation around that axis!}

% p. 372

We may picture a piece of matter, in the absence of any external
magnetic field, as containing revolving electrons with their various
orbital angular momentum vectors and associated orbital magnetic
moments distributed evenly over all directions in space. Consider
those orbits which happen to have their planes approximately parallel
to the $xy$ plane, of which there will be about equal numbers with
$\vc{m}$ up and $\vc{m}$ down. Let's find out what happens to one of these orbits
when we switch on an external magnetic field in the $z$ direction.

We'll analyze first an electromechanical system that doesn't look
much like an atom. In Fig. 10.12 there is an object of mass $M$ and
electric charge $q$, tethered to a fixed point by a cord of fixed length $r$.
This cord provides the centripetal force that holds the object in its
circular orbit. The magnitude of that force $F_0$ is given by
\begin{equation}
  F_0 = \frac{M v_0^2}{r}
\end{equation}
In the initial state, Fig. 10.l2a, there is no external magnetic field.
Now, by means of some suitable large solenoid, we begin creating
a field $\vc{B}$ in the negative $z$ direction, uniform over the whole region
at any given time. While this field is growing at the rate $\der B/\der t$, there
will be an induced electric field $\vc{E}$ all around the path, as indicated in
Fig. 10.12b. To find the magnitude of this field $\vc{E}$ we note that the
rate of change of flux through the circular path is
\begin{equation}
  \frac{\der\Phi}{\der t} = \pi r^2\frac{\der B}{\der t}
\end{equation}
This determines the line integral of the electric field, which is really all
that matters (we only assume for symmetry and simplicity that it
is the same all around the path).
\begin{equation}
  \int \vc{E}\cdot\der\bell = \frac{\pi r^2}{c} \frac{\der B}{\der t} = 2\pi r E
\end{equation}
Thus we find that
\begin{equation}
  E = \frac{r}{2c}\frac{\der B}{\der t}
\end{equation}
% p. 373
We have ignored signs so far, but if you apply to Fig. 10.12 your
favorite rule for finding the direction of an induced electromotive
force, you will see that $\vc{E}$ must be in a direction to accelerate the body,
if $q$ is a positive charge. The acceleration along the path, $\der v/\der t$, is
determined by the force $qE$:
\begin{equation}
  M \frac{\der v}{\der t} = qE = \frac{qr}{2c} \frac{\der B}{\der t}
\end{equation}
so that we have a relation between the change in $v$ and the change
in $B$:
\begin{equation}
  \der v = \frac{qr}{2Mc} \der B
\end{equation}
The radius $r$ being fixed by the length of the cord, the factor $(qr/2Mc)$
is a constant. Let $\Delta v$ denote the net change in $v$ in the whole process
of bringing the field up to the final value $B_1$. Then:
\begin{equation}
  \Delta v = \int_{v_0}^{v_0+\Delta v}\der v = \frac{qr}{2Mc} \int_0^{B_1} = \frac{qrB_1}{2Mc}
\end{equation}
Notice that the time has dropped out --- the final velocity is the same
whether the change is made slowly or quickly.

The increased speed of the charge in the final state means an increase
in the upward-directed magnetic moment $\vc{m}$. A \emph{negatively}
charged body would have been \emph{de}celerated under similar 
circumstances, which would have \emph{decreased} its \emph{downward} moment. In
either case, then, the application of the field $\vc{B}_1$ has brought about
a change in magnetic moment opposite to the field. The magnitude
of the change in magnetic moment $\Delta m$, is
\begin{equation}
  \Delta m = \frac{qr}{2c} \Delta v = \frac{q^2r^2}{4Mc^2} B_1
\end{equation}

Likewise for charges, either positive or negative, revolving in the
other direction, the induced change in magnetic moment is opposite
to the change in applied magnetic field. Figure 10.13 shows this for
% p. 374
a positive charge. It appears that the following relation holds for
either sign of charge and either direction of revolution:
\begin{equation}
  \Delta\vc{m} = -\frac{q^2r^2}{4Mc^2} \vc{B}_1
\end{equation}

In this example we forced $r$ to be constant by using a cord of fixed
length. Let us see how the tension in the cord has changed. We
shall assume that $B_1$ is small enough so that $\Delta v \ll v_0$. In the final
state we require a centripetal force of magnitude:
\begin{equation}
  F_1 = \frac{M(v_0+\Delta v)^2}{r} \approx \frac{Mv_0^2}{r} + \frac{2Mv_0\Delta v}{r}  
\end{equation}
neglecting the term proportional to $(\Delta v)^2$. But now the magnetic
field itself provides an inward force on the moving charge, given by
$q(v_0 + \Delta v)B_1/c$. Using Eq. 29 to express $B_1$ in terms of $\Delta v$, we find
that this extra inward force has the magnitude
$\frac{q(v_0+\Delta v)}{c}\cdot\frac{2Mc\Delta v}{qr}$
which, to first order in $\Delta v/v_0$, is $2Mv_0 \Delta v/r$. That is just what is
needed, according to Eq. 32, to avoid any extra demand on our cord!
Hence the tension in the cord \emph{remains unchanged, at the value $F_0$}.

This points to a surprising conclusion: our result, Eq. 31 must be
valid for \emph{any} kind of tethering force that varies with radius. Our
cord could be replaced by an elastic spring without affecting the
outcome --- the radius would still be unchanged in the final state.
Or to go at once to a system we are interested in, it could be replaced
by the Coulomb attraction of a nucleus for an electron. Or it could
be the effective force that acts on one electron in an atom containing
many electrons, which has a still diflerent dependence on radius.
% p. 375
From a relation as \emph{general} as Eq. 31 has turned out to be, we can hope
for some meaningful results even without a good theory of atomic
structure. The only feature of the atom that appears explicitly is $r^2$.
Of course, we must observe the restriction $\Delta v/v_0 \ll 1$, in effect a
restriction on $B_1$, which allowed this general application.

The effect on electron orbits of switching on a magnetic field $\vc{B}$ can
be visualized in this way: Every electron continues to revolve at the
same radius, but its angular velocity, which had been $\pm v_0/r$, depending
on its sense of revolution, has added to it a small increment
$\Delta\omega=\Delta v/r$. According to Eq. 29 the value of this increment is
\begin{equation}
  \Delta\omega = \frac{\Delta v}{r} = \frac{eB}{2m_ec}
\end{equation}
an angular velocity depending only on the strength of the applied
field and the charge-mass ratio of the electron. All revolutions one
way are speeded up by the same amount, expressed in radians per
second, those in the other direction slowed down by the same
amount. The new system looks just like the old system Viewed from
a \emph{rotating frame of reference}. The angular velocity $eB/2m_ec$ in Eq. 33
is called the ``Larmor angular velocity,'' or ``Larmor frequency.''\index{Larmor frequency}
Sir Joseph Larmor, the British mathematical physicist, proved this
general theorem in 1895, before anyone really knew how an atom
is constructed.

We have considered only orbits perpendicular to $\vc{B}$. Our conclusion
should apply, roughly speaking, to a third of the electron
orbits in a substance, there being three mutually perpendicular
directions. What happens to orbits lying parallel to the $xz$ planes
and $yz$ planes is interesting, and you can find out by working
Prob. 10.22. They also contribute an induced moment opposite to
the field and proportional to the square of the orbit radius. The
effect of all the orbits can be summarized in an equation like Eq. 31,
with $r^2$ replaced by $\langle r^2 \rangle$, the average of the squares of the orbit radii,
and with some numerical factor to take care of averaging over orbit
orientation.

Without going into this refinement, 1et's use Eq. 31 as it stands
for all the electrons, putting in some reasonable estimate for an orbit
radius, and see if we can approximately account for some of the data
in the table. The number of electrons per gram is about the same
in most substances, because there is one proton in the nucleus for
every electron in the atom, and roughly one neutron per proton. So
the number of electrons per gram, $n$, is about the same as in a substance
% p. 376
of atomic weight 2 and atomic number 1, namely:
\begin{equation}
  n \approx \frac{6\times10^{23}}{2} = 3\times10^{23}
\end{equation}
For $r$ we shall put in $0.5 \times 10^{-8}\ \cmunit$, a distance you will become very
familiar with later as a typical atomic ``radius.'' In many-electron
atoms, naturally, some electrons have large orbits and some small.
For $M$ we substitute the electron mass $m_e$. The magnetic field
strength at the location of the sample was 18,000 gauss. Then the
total magnetic moment induced in one gram of almost anything
should be, roughly:
\begin{align}
\begin{split}
  n\Delta m &= \frac{ne^2r^2B}{4m_ec^2} \\
      &= \frac{(3\times10^{23})(4.8\times10^{-10})^2(0.5\times10^{-8})^2(1.8\times10^4)}
              {4(9\times10^{-28})(3\times10^{10})^2} \\
      &= 0.95\times10^{-2}
\end{split}
\end{align}
The gradient of the field $\partial B_z/\partial z$ was 1700 gauss/cm. Using Eq. 18
to calculate the force, we would predict a force of magnitude
$1700 \times 0.95 \times 10^{-2}$, or about 16 dynes. This is close to the figure
given for a number of substances in the table. Indeed, it is closer
than we had any right to expect, so the agreement is to that extent
accidental.\footnote{The exact formula, obtained by averaging over orbits isotropically oriented, replaces
the factor $\frac{1}{4}$ in Eq. 31 by $\frac{1}{6}$, while replacing $r^2$ by $\langle r^2 \rangle$.
A rigorous quantum
mechanical theory leads to exactly the same result, and the agreement with 
experiment is excellent. Indeed, for many diamagnetic atoms the magnetic measurements
provide the most accurate way to determine $\langle r^2 \rangle$.
}

We had better make sure that our assumption $\Delta v\ll v_0$ is fulfilled
in this situation. Putting the same numbers into Eq. 29, we can
estimate $\Delta v$:
\begin{align}
\begin{split}
  \Delta v &= \frac{erB}{2m_ec}
\end{split}
\end{align}
You don't need to know much atomic physics to realize that
$10^3\ \cmunit/\sunit$ is small compared to the speed of an electron in an atom.
A person can run that fast! A typical velocity for an atomic electron
is more like $10^3\ \cmunit/\sunit$, or higher. This shows that even our rather
powerful magnet applies a field that is very weak, from the point of
view of an atomic electron, causing only a slight change in the speeds
of revolution.

% p. 377

We can see now why diamagnetism is a universal phenomenon,
and a rather inconspicuous one. It is about the same in molecules
as in atoms. The fact that a molecule can be a much larger structure
than an atom --- it may be built of hundreds or thousands of atoms-
does not generally increase the effective mean-square orbit radius.
The reason is that in a molecule any given electron is pretty well
localized on an atom. There are some interesting exceptions and
we included one in the table --- graphite. The anomalous 
diamagnetism of graphite is due to an unusual structure which permits some

electrons to circulate rather freely within a planar group of atoms
in the crystal lattice.

\iffalse

\section{Electron spin and magnetic moment}

The electron possesses angular momentum that has nothing to
do with its orbital motion. It behaves in many ways as if it were
continually rotating around an axis of its own. This property is
called spin. When the magnitude of the spin angular momentum
is measured, the same result is always obtained: h/47-r, where h is
Planck's constant. Electron spin is a quantum phenomenon. You
will hear more about its discovery and its implications in Vol. IV of
this course. Its significance for us now lies in the fact that there is
associated with this intrinsic, or ``built-in,'' angular momentum a
nuzgnetic moment, likewise of invariable magnitude. This magnetic
moment points in the direction you would expect if you visualize
the electron as a ball of negative charge spinning around its axis.
That is, the magnetic moment vector points antiparallel to the spin
angular momentum vector, as indicated in Fig. 10.14. The magnetic
moment, however, is twice as large, relative to the angular 
momentum, as is the case in orbital motion.

There is no point in trying to devise a classical model of this object;
its properties are essentially quantum mechanical. We need not
even go so far as to say it is a current loop. What matters is only that
it behaves like one in the following respects: (i) it produces a magnetic
field which, at a distance, is that of a magnetic dipole; (ii) in an
external field B it experiences a torque equal to that which would
act on a current loop of equivalent dipole moment; (iii) within the
source, div B = O everywhere, as in the ordinary sources of magnetic
field with which we are already familiar.

Since the magnitude of the spin magnetic moment is always the
same, the only thing an external field can influence is its direction.
A magnetic dipole in an external field experiences a torque. If you
% p. 378
worked through Prob. 6.22, you proved that the torque N on a current
loop of any shape, with dipole moment m, in a field B, is given by

For those who have not been through that demonstration, let's take
time out to calculate the torque in a simple special case. In Fig. 10.15
we see a rectangular loop of wife carrying current I. The loop has
a magnetic moment in, of magnitude m = Iab/c. The torque on
the loop arises from the forces F1 and F2 that act on the horizontal
wires. Each of these forces has the magnitude F = IbB/c, and its
moment arm is"the distance (a/ 2) sin 0. We see that the magnitude

of the torque on the loop is
\begin{equation}
\end{equation}

N: 2'''7B-%sino = (%)13sino = mBsin0 (38)

The torque acts in a direction to bring in parallel to B; it is represented
by a vector N in the positive x direction, in the situation
shown. All this is consistent with the general formula, Eq. 37.
Notice that Eq. 37 corresponds exactly to the formula we derived
in Chap. 9 for the torque on an electric dipole p in an external field E,
namely, N = p x E. The orientation with m in the direction of B,
like that of the electric dipole parallel to E, is the position of lowest
energy. Similarly, the work required to rotate a dipole in from parallel
to antiparallel is 2mB. (See Eq. 9.18; we can simply take over
this result for the magnetic case.)

If the electron spin moments in a substance are free to orient them-
selves, we expect them to prefer the orientation in the direction of
any applied field B, the orientation of lowest energy. Suppose every
electron in a gram of material takes up this orientation. We had
already calculated that there are roughly 3 X 1023 electrons in a gram
of anything. The spin magnetic moment of an electron, ms, is given
in Fig. 10.14 as 0.93 X 1O‘2° erg/gauss. The total magnetic moment
of our lined-up spins will be (3 )< 1023) X (0.9 )( 10-20) or 2700
ergs/gauss. The force on such a sample, in our coil where the field
gradient was 1700 gauss/cm, would be 4.6 x 106 dynes, or a little
over ten pounds!

Obviously this is much greater than the force recorded for any of
the paramagnetic samples. The explanation is that the alignment
of the electron moments is very far from perfect. Thermal agitation
% p. 379
tends always to create a chaotic, or random, distribution of spin-axis
directions. The degree of alignment that actually prevails represents
a compromise between the preference for the direction of lowest
energy and the disorienting influence of thermal motion. It turns
out that the total magnetic moment is generally proportional to the
applied field B, and inversely proportional to the absolute temperature
T. How that works out we must leave for Vol. V of this course,
the theme of which might be described as the competition involving
energy and disorder. The paramagnetism of electron spins will
provide, at that time, an instructive example. The quantum physics
you will learn between now and then will make the problem simpler
than it appears from our present viewpoint.

Why isn't everything paramagnetic? The reason is that in most
atoms and molecules, the electrons are grouped in pairs, with the
spins in each pair constrained to point in opposite directions regardless
of the applied field. As a result, the magnetic moments of an
electron pair exactly cancel each other. All that is left is the 
diamagnetism we had already explored. A few molecules contain an odd
number of electrons, and in them total cancellation in pairs is obviously
impossible. Nitric oxide, NO, with 15 electrons in the molecule
is an instance: it is paramagnetic. The oxygen molecule 02
contains an even number of electrons, but its electronic structure
happens to favor noncancellation of two of the electron spins. In
certain groups of the elements, notably the elements around gadolinium
in the periodic table, and also those around iron, the atoms
contain unpaired electron spins which are relatively free to orient
in a magnetic field. (The magnetic moment of such an atom often
includes some contribution from the orbital motion as well.) In
metallic conductors, the ``free'' electrons that roam through the metal
have a weakly paramagnetic behavior all their own. All of this is
basically quantum physics.

Even diamagnetism, fundamentally, involves quantum mechanics.
Consider two electrons circling in opposite directions in an atom.
We explained that diamagnetism arises because an applied field B
causes one electron to speed up slightly and the other to slow down.
But why don't both orbits eventually shift around so their orbital
magnetic moments are pointing in the same direction, parallel to
the field? The answer is that the two electrons, in most cases, are
required by quantum mechanical laws to maintain opposite direc-

tions of orbital revolution, much as the electron spins are paired ofi``
two by two.

% p. 380
\section{Magnetic susceptibility}

We have seen that both diamagnetic and paramagnetic substances
develop a magnetic moment proportional to the applied field. At
least, that is true under most conditions. At very low temperatures,
in fairly strong fields, the induced paramagnetic moment can be
observed to approach a limiting value as the field strength is 
increased. Setting this ``saturation'' effect aside, the relation between
moment and applied field is quite linear, so that we can characterize
the magnetic properties of a substance by the ratio of induced
moment to applied field. The ratio is called the magnetic 
susceptibility. Depending on whether we choose the moment of 1 g of
material, of 1 cm3 of material, or of 1 mole, we define the specific
susceptibility, the volume susceptibility, or the molar susceptibility.
Our discussion in Sec. 10.5 suggests that for diamagnetic substances
the specific susceptibility, based on the induced moment per gram,
should be most nearly the same from one substance to another..
However, the volume susceptibility, based on the induced magnetic
moment per cubic centimeter, is more relevant to our present
concerns.

The magnetic moment per unit volume we shall call the magnetic
polarization, or the magnetization, using for it the symbol M. Now
magnetization M and magnetic field B have similar dimensions.T
To verify that, recall that the field B of a magnetic dipole is given by
magnetic dipole moment

\begin{equation}
\end{equation}
(distance)3
magnetic dipole moment

volume '
magnetic susceptibility, denoted by x,,,, through the relation

,while M, as we have just defined it, has the

dimensions If we now define the volume

M = X,,,B (warning: see remarks below) (39)

the susceptibility will be a dimensionless number, negative for 
diamagnetic substances, positive for paramagnetic. This is exactly
analogous to the procedure, expressed in Eq. 9.38, by which we
defined the electric susceptibility Xe as the ratio of electric polarization
P to electric field E. We shall presently see that the analogy goes
even deeper because the macroscopic field B inside matter will turn
out to be the average of the microscopic B, just as the macroscopic E
turned out to be the average of the microscopic E.

TWhile the dimensions of M and B are the same, it would be confusing to express
them in the same units, because of a factor 47r that will turn up presently. If and
when a name for the units of M is called for, we shall use ergs/gauss-cm3.

% p. 381
Unfortunately, Eq. 39 is not the customary definition of volume
magnetic susceptibility. In the usual definition another field H,
which we shall meet in due course, appears instead of B (see Eq. 55).
Although illogical, the definition in terms of H has a certain practical
justification, and the tradition is so well established that we shall
eventually have to bow to it. But in this chapter we want to follow
as long as we can a path that naturally and consistently parallels the
description of the electric fields in matter.

The difference in definition is of no practical consequence as long
as X," is a number very small compared to one. The values of xm
for purely diamagnetic substances, solid or liquid, lie typically between
-0.5 X 10‘6 and -1.0 )< 10*'. Even for oxygen under the
conditions given in the table, the paramagnetic susceptibility is less
than 10-3. This means that the magnetic field caused by the dipole
moments in the substance, at least as a large-scale average, is very
much weaker than the applied field B. That gives us some confidence
that in such systems we may assume the field that acts on the
atomic dipole to orient them is the same as the field that would exist
there in the absence of the sample. However, we shall be interested
in other systems in which the field of the magnetic moments is not
small. Therefore we must study, just as we did in the case of electric
polarization, the magnetic fields that magnetized matter itself 
produces, both inside and outside the material.

\section{The magnetic field caused by magnetized matter}

A block of material which contains, evenly distributed through its
volume, a large number of atomic magnetic dipoles all pointing in
the same direction, is said to be uniformly magnetized; The magnetization
vector M is simply the product of the number of oriented
dipoles per unit volume and the magnetic moment In of each dipole.
We don't care how the alignment of these dipoles is maintained.
There may be some field applied from another source, but we are

not interested in that. We want to study only the field produced
by the dipoles themselves.

Consider first a slab of material of thickness dz, sliced out perpendicular
to the direction of magnetization, as shown in Fig. 10. 16a.
The slab can be divided into little tiles. One such tile, which has a
top surface of area da, contains a total dipole moment amounting to
M da dz, since M is the dipole moment per unit volume (Fig. 10.1619).
The magnetic field this tile produces at all distant points --- distant
% p. 382
compared to the size of the tile --- is just that of any dipole with the
same magnetic moment. We could construct a dipole of that
strength by bending a conducting ribbon of width dz into the shape
of the tile, and sending around this loop a current I = M0 dz (Fig.
10.160). That will give the loop a dipole moment:
\begin{equation}
\end{equation}

m=Lxarea= Mcdz
c a

da. : M da dz (40)

which is the same as that of the tile.

Let us substitute such a current loop for every tile in the slab, as
indicated in Fig. 10.l6d. The current is the same in all of these and
therefore, at every interior boundary we find equal and opposite
currents, equivalent to zero current. Our ``egg-crate'' of loops is
therefore equivalent to a single ribbon running around the outside,
carrying the current Mc dz (Fig. 10.l6e). Now these tiles can be
made quite small, so long as we don't subdivide clear down to molecular
size. They must be large enough so that their magnetization
does not vary appreciably from one tile to the next. Within that
limitation, we can state that the field at any external point, even close
to the slab, is the same as that of the current ribbon.

It remains only to reconstruct a whole block from such 
laminations, or slabs, as in Fig. 10.l7a. The entire block is then equivalent
to the wide ribbon in Fig. 10. 1712 around which flows a current Mc dz,

in esu/ sec, in every strip dz, or, stated more simply, a surface current
of density <9, in (esu/sec)/cm, given by
\begin{equation}
\end{equation}

é] = M0 (41)

The magnetic field B at any point outside the magnetized block in
Fig. l0.l7a, and even close to the block provided we don't approach
within molecular distances, is the same as the field B' at the corresponding
point in the neighborhood of the wide current ribbon in
Fig. 10.1719.

But what about the field inside the magnetized block? Here we
face a question hke the one we met in Chap. 9. Inside matter the
magnetic field is not at all uniform if we observe it on the atomic
scale which we have been calling ``microscopic.'' It varies sharply
in both magnitude and direction between points only a few angstroms
apart. This microscopic field B is simply a magnetic field in vacuum,
for from the microscopic viewpoint, as we emphasized in Chap. 9,
matter is a collection of particles and electric charge in otherwise
% p. 383
empty space. The only large-scale field that can be uniquely defined
inside matter is the spatial average of the microscopic field.

Because of the absence of effects attributable to magnetic charge,
we believe that the microscopic field itself satisfies div B = 0. If that
is true, it follows quite directly that the spatial average of the internal
microscopic field in our block is the same as the field B' inside the
equivalent current ribbon.

To demonstrate this, consider the long rod uniformly magnetized
parallel to its length, shown in Fig. l0.18a. We have just shown that
the external field will be the same as that of the long cylinder of current
(practically equivalent to a single-layer solenoid) shown in
Fig. 10.1819. S in Fig. lO.l8a indicates a closed surface which includes
a portion of S1 passing through the interior of the rod.
Because div B = 0 for the internal microscopic field, as well as for
the external field, div B is zero throughout the entire volume enclosed
by S. It then follows from Gauss's theorem that the surface integral
of B over S must be zero. The surface integral of B' over the closed
surface S' is zero also. Over the portions of S and S' external to the
cylinders, B and B' are identical. Therefore the surface integral of B
over the internal disk S1 must be equal to the surface integral of B'
over the internal disk Si. This must hold also for any one of a closely
spaced set of parallel disks, such as S2, S3, etc., indicated in
Fig. 10.18c, because the field outside the cylinder in this neighborhood
is negligibly small, so that the outside parts don't change any-
thing. Now taking the surface integral over a series of equally spaced
planes like that is a perfectly good way to compute the volume
average of the field B in that neighborhood, for it samples all volume
elements impartially. It follows that the spatial average of the microscopic
field B inside the magnetized rod is equal to the field B' inside
the current sheath of Fig. 10.1%.

It is instructive to compare the arguments we have just developed
with our analysis of the corresponding questions in Chap. 9. Figure
10.19 displays these developments side by side. You will see that
they run logically parallel, but that at each stage there is a difference
which reflects the essential asymmetry epitomized in the observation
that electric charges are the source of electric fields, while moving
electric charges are the source of magnetic fields. For example, in
the arguments about the average of the microscopic field, the key
to the problem in the electric case is the assumption that curl E = 0
for the microscopic electric field. In the magnetic case, the key is
the assumption that div B = 0 for the microscopic magnetic field.

% p. 384
is equivalent to:
\begin{equation}
\end{equation}

     

q = -P da '51‘ = M dz
/ ---  Charge because a bit of magnetized matter, d

I 55:31 P da volume da'dz, has dipole moment ,: --- -a~)_

ik '_jdz equal to that of: fCurr(eint
\ --- Charge CM Z

_ P da 

A uniformly magnetized block can
be divided into such layers. Hence

the block has the
same external  
field as the wide  3-
ribbon of surface 
current with 3 = 0M.

[More generally, for nonuniform magnetization,

rn, polarized magnetized matter is equivalent to a current distribution
1p=-divP.] J=ccurlM]

      

{AT THE EQUIVALENCE EXTENDS TO
.L AVERAGE OF THE INTERNAL FIELDS

ed slab and its Consider a long uniformly magnetized column and its
he middle the equivalent cylinder of surface
current. Near the middle the

   
  

   

E'!

_ external field is slight and ;  ‘
'V><E= O B'is uniform. If V-B = 0 
E = E' on for the internal field, thenfsB- d a = 0. But B = B' on the
"dlr for all surface external to the column. Hence 5 B- da :1; , B'-da'

1 1
over any interior portion of surface like S1, S2, etc.

1aV§Tage Of CONCLUSION: (B) = B' ; the spatial average of
d E that the internal magnetic field is equal to the field B' that
space by the would be produced at that point in empty space by the
bove (together equivalent current distribution described above (together

with any external sources).

:clectric (a) and magnetic (b) cases compared.

% p. 385
If the magnetization M within a volume of material is not uniform

but instead varies with position as M(x,y,z), the equivalent current
distribution is given simply by
\begin{equation}
\end{equation}

J = c curl M (42)

Let's see how this comes about in one situation. Suppose there is a
magnetization in the z direction, which gets stronger as we proceed
in the y direction. This is represented in Fig. 10.20a, which shows a

% p. 386
small region in the material subdivided into little blocks. The blocks
are supposed to be so small that we may consider the magnetization
uniform within a single block. Then we can replace each block by
a current ribbon, with surface current density 67 = cMz. The current
I carried by such a ribbon, if the block is Az in height, is é] Az
or cM2 Az. Now each ribbon has a bit more current density than
the one to the left of it. The current in each loop is greater than the
current in the loop to the left by

BMZ
3y
At every interface in this row of blocksthere is a net current in the
x direction of magnitude AI (Fig. 10.20c). To get the current per

unit area flowing in the x direction we have to multiply by the number
of blocks per unit area, which is l/(Ag Az). Thus
\begin{equation}
\end{equation}

AI: cAzAMz: cAz

Ag (43)

1 8M
J = A1( ) 2 6 44
3'' Ag Az 0 0y ( )

\begin{equation}
\end{equation}

Another way of getting an x-directed current is to have a y component
of magnetization that varies in the z direction. If you trace
through that case, using a vertical column of blocks, you will find
that the net x-directed current density is given by
\begin{equation}
\end{equation}

8M,
4, 2  ---  45
J c az ( )
in general then, by superposition of these two situations,
\begin{equation}
\end{equation}
J,, = c  _ 33:'') = c(curl M), (46)

which is enough to establish Eq. 42.

\section{The field of a permanent magnet}

The uniformly polarized spheres and rods we talked about in
Chap. 9 are seldom seen, even in the laboratory. Frozen-in electric
polarization can occur in some substances, although it is usually
disguised by some accumulation of free charge. To make Fig. lO.3a,
which shows how the field of a polarized rod would look, it was necessary
to use two charged disks. On the other hand, materials with
permanent magnetic polarization, that is, permanent magnetization,
are familiar and useful. Permanent magnets can be made from many

% p. 387
alloys and compounds of the ferromagnetic substances. What
makes this possible is a question we'll leave for Sec. 10.11, where we
dip briefly into the physics of ferromagnetism. In this section, taking
the existence of permanent magnets for granted, we want to study
the magnetic field B of a uniformly magnetized cylindrical rod, and
compare it carefully with the electric field E of a uniformly polarized
rod of the same shape.

Figure 10.21 shows each of these solid cylinders in cross section.
The polarization, in each case, is parallel to the axis, and it is 
uniform. That is, the polarization P and the magnetization M have the
same magnitude and direction everywhere within their respective
cylinders. In the magnetic case this implies that every cubic millimeter
of the permanent magnet has the same number of lined-up
electron spins, pointing in the same direction. (A very good approximation
to this can be achieved with modern permanent magnet
materials.)

By the field inside the cylinder we mean, of course, the macroscopic
field defined as the space average of the microscopic field.
With this understanding, we show in Fig. 10.2] the field lines both
inside and outside the rods. By the way, these rods are not supposed
to be near one another; we only put the diagrams together for convenient
comparison. Each rod is isolated in otherwise field-free

space. (Which do you think would more seriously disturb the field
of the other, if they were close together?)

Outside the rods the fields E and B look alike. In fact the field
lines follow precisely the same course. That should not surprise
you if you recall that the electric dipole and the magnetic dipole have
similar ``far fields.'' Each little chunk of the magnet is a magnetic
dipole, each little chunk of the polarized rod (sometimes called an
``electret'') is an electric dipole, and the field outside is the superposition
of all their far fields.

The field B, inside and out, is the same as that of a cylindrical
sheath of current. In fact if we were to wind very evenly, on a cardboard
cylinder, a single-layer solenoid of fine wire, we could hook
a battery up to it and duplicate the exterior and interior field B of
the permanent magnet. (The coil would get hot and the battery
would run down; electron spins provide the current free and friction-
less!) The electric field E, both inside and outside the polarized rod,
is that of two disks of charge, one at each end of the cylinder.

Observe that the interior fields E and B are essentially different
in form: B points to the right, is continuous at the ends of the
% p. 388
cylinder, and suffers a sharp change in direction at the cylindrical
surface. E points to the left, passes through the cylindrical surface
as if it weren't there, but is discontinuous at the end surfaces. These
differences arise from the essential difference between the ``inside''
of the physical electric dipole, and the ``inside'' of the physical magnetic
dipole, seen in Fig. 10.8. By physical, we mean the ones
Nature has actually provided us with.

If the external field were our only concern, we could use either
picture to describe the field of our magnet. We could say that the
magnetic field of the permanent magnet arises from a layer of positive
magnetic charge --- a surface density of ``north'' magnetic poles --- 
on the right-hand end of the magnet, and a layer of negative magnetic
charge, ``south'' poles, on the other end. We could adopt a scalar
potential function cpmag, such that B =  --- grad (pmas. The potential
function tpmag would be related to the fictitious pole density as the
electric potential is related to charge density. The simplicity of the
scalar potential compared to the vector potential is rather appealing.
Moreover, the magnetic scalar potential can be related in a very neat
way to the currents that are the real source of B, and thus one can use
the scalar potential without any explicit use of the fictitious poles.
You may want to use this device if you ever have to design magnets
or calculate magnetic fields.

We must abandon the magnetic pole fiction, however, if we want
to understand the field inside the magnetic material. That the
macroscopic magnetic field inside a permanent magnet is, in a very
real sense, like the field in Fig. 10.2119 rather than the field in
Fig. lO.2la has been demonstrated experimentally by deflecting
energetic charged particles in magnetized iron, as well as by the magnetic
effects on slow neutrons, which pass even more easily through
the interior of matter.

Figure 10.22a shows a small disk-shaped permanent magnet, in
which the magnetization is parallel to the axis of symmetry. You
are probably more familiar with permanent magnets in the shape of
long bars. However, flat disk magnets of considerable strength can
be made with certain new materials. The dimensions in this example
were chosen the same as the dimensions of the electrically polarized
disk in Fig. 9.21. The magnetization M is given as 150 in
CGS units. The magnetic moment of the electron is 0.93 x l0'2°
erg/gauss, so this value of M corresponds to 1.6 )< 102? lined-up
electron spins, per cubic centimeter. The disk is equivalent to a
band of current around its rim, of surface density <3 = cM. The rim

% p. 389
being 0.3 cm wide, the current I amounts to
0.3cM = (0.3) (3 x 101°)(15O)

or 1.35 x 1012 esu/sec. In practical units this is 450 amperes --- 
rather more current than you draw by short-circuiting an automobile
battery! The field B at any point in space, including points inside
the disk, is simply the field of this band of current. For instance,
near the center of the disk, B is approximately:
\begin{equation}
\end{equation}

_ 27rI _ 277(0.3cM) _ 277(O.3)(l50)
_  --- rc_ _ rc _ (1.0)

B = 280 gauss (47)

The approximation consists in treating the 0.3-cm-wide band of 
current as if it were concentrated in a simple ring. (In the corresponding
approximation in the electrical example we treated the equivalent
charge sheets as large compared to their separation.) As for
the field at a distant point, it would be easy to compute it for the
ring current, but we could also, for an approximate calculation,
proceed as we did in the electrical example. That is we could find
the total magnetic moment of the object, and find the distant field
of a single dipole of that strength.

\section{Free currents, and the field $H$}

It is often useful to distinguish between bound currents and free
currents. Bound currents are currents associated with molecular
or atomic magnetic moments, including the intrinsic magnetic
moment of particles with spin. These are the molecular current
loops envisioned by Ampere, the source of the magnetization we
have just been considering. Free currents are ordinary conduction
currents flowing on macroscopic paths --- currents that can be started
and stopped with a switch and measured with an ammeter.

The current density J in Eq. 42 is the macroscopic average of the
bound currents, so let us henceforth label it Jbound:
\begin{equation}
\end{equation}

Jbound : c curl M (48)

At a surface where M is discontinuous, such as the side of the magnetized
block in Fig. 10.17, we have a surface current density :7 which
also represents bound current.

We found that B, both outside matter and, as a space average,
inside matter, is related to Jbmmd just as it is to any current density.

% p. 390
That is, curl B = (471/c)Jb0.md. But that was in the absence of free
currents. If we bring these into the picture, the field they produce

simply adds on to the field caused by the magnetized matter and
we have
\begin{equation}
\end{equation}

Curl B = i(:‘7T_(Jbound ‘l' Jfree) ,: 4777 Jtotal 

Let us express Jbound in terms of M, through Eq. 48. Then Eq. 49
becomes
\begin{equation}
\end{equation}

curl B = 47''(c curl M) + 4T''J,,,, (50)
which can be rearranged as
\begin{equation}
\end{equation}
curl (B  ---  477M) = 4T''J,,,, (51)

If we now define a vector function H(x,y,z) at every point in space
by the relation
\begin{equation}
\end{equation}

H = B  ---  47rM (52)
Eq. 51 can be written
\begin{equation}
\end{equation}
curl H = 4T7TJ;,ee (53)

In other words, the vector H, defined by Eq. 52, is related to the
free current in the way B is related to the total current, bound plus
free. The parallel is not complete however, for we always have
div B = 0, whereas our vector function H does not necessarily have
zero divergence.

This surely has reminded you of the vector D which we introduced,
a bit grudgingly, in the last chapter. D, remember, was related to
the free charge as E is related to the total charge. Although we rather
disparaged D, the vector H is really useful, for a practical reason that
is worth understanding. In electrical systems, what we can easily
control and measure are the potential differences of bodies, and not
the amounts of free charge on them. Thus we control the electric
field E directly. D is out of our direct control, and since it is not a
fundamental quantity in any sense, what happens to it is not of much
concern. In magnetic systems, however, it is precisely the free currents
that we can most readily control. We lead them through wires,
measure them with ammeters, channel them in well-defined paths

% p. 391
with insulation, and so on. We have much less direct control, as a

rule, over magnetization, and hence over B. So the auxiliary vector
H is useful, even if D is not.

The integral relation equivalent to Eq. 53 is
\begin{equation}
\end{equation}
. _ .41 . _ fl
fCH d1. 0 SJfree da ---  0 Ifree (54)

where Ifm is the total current enclosed by the path C. Suppose we
wind a coil around a piece of iron and send through this coil a certain
current I which we can measure by connecting an ammeter in
series with the coil. This is the free current, and it is the only free
current in the system. Therefore one thing we know for sure is the
line integral of H around any closed path, whether that path goes
through the iron or not. The integral depends only on the number
of turns of our coil that are linked by the path, and not on the magnetization
in the iron. The determination of M and B in this system
may be rather complicated. It helps to have singled out one quantity
that we can determine quite directly.

Figure 10.23 illustrates this property of H by an example, and is
a reminder of the units we may use in a practical case. H has the

% p. 392
same dimensions as B; in the Gaussian CGS system they are related
in exactly the same way to current in esu/sec. As you know, the
unit of magnetic field strength B in this system is named the gauss.
There was no compelling need for a different name for the unit of H.
Nevertheless, people who like to name things have given the unit
of H a name all its own, the oerstedj‘ Because you will find this
name used elsewhere, we have introduced it in Fig. 10.23.

We consider B the fundamental magnetic field vector because the
absence of magnetic charge, which we discussed in Sec. 10.2, implies
div B = 0 everywhere, even inside atoms and molecules. From
div B = 0 it follows, as we showed in Sec. 10.8, that the average
macroscopic field inside matter is B, not H. The implications of
this have not always been understood or heeded in the past. Further-
more, H has the practical advantage we have already explained. In
some older books you will find H introduced as the primary magnetic
field. B is then defined as H + 477M, and given the name magnetic
induction. Even some modern writers who treat B as the primary
field feel obliged to call it the magnetic induction because the name
magnetic field was historically preempted by H. This seems clumsy
and pedantic. If you go into the laboratory and ask a physicist what
causes the pion trajectories in his bubble chamber to curve, he'll
probably answer ``magnetic field,'' not ``magnetic induction.'' You
will seldom hear a geophysicist refer to the earth's magnetic 
induction, or an astronomer talk about the magnetic induction in the
Galaxy. We propose to keep on calling B the magnetic field. As
for H, although other names have been invented for it, we shall call
it ``the field H,'' or even, ``the magnetic field H.''

It is only the names that give trouble, not the symbols. Everyone
agrees that in the Gaussian CGS system the relation connecting B,
M, and H is that stated in Eq. 52. In vacuum there is no essential
distinction between B and H, for M must be zero where there is no
matter. You will often see Maxwell's equations for the vacuum fields
written with E and H, rather than E and B.

These remarks about names and units are not relevant to the MKS
electrical unit system. In that system quantities corresponding to
our B and H, and denoted by the same symbols, are treated as
dimensionally different and their numerical values differ even in
vacuum. (See Appendix.)

TMeasuring B in gauss and H in oersteds is like expressing the radius of a circle in
centimeters, but reserving a special name, such as ``are centimeter,'' for the unit of distance
measured around the circumference!

% p. 393
We should note here too that the accepted definition of the volume

magnetic susceptibility X", is not the logically preferable one given
in Eq. 39, but rather:
\begin{equation}
\end{equation}

M = X,,,H (55)

The permanent magnet in Fig. 10.2119 is an instructive example
of the relation of H to B and M. To obtain H at some point inside
the magnetized material, we have to add vectorially to the magnetic
field B at that point the vector  --- 47rM. Figure 10.24 depicts this
for a particular point P. It turns out that the lines of H inside the
magnet look just like the lines of E inside the polarized cylinder of
Fig. 10.21a. That is as it should be, for if magnetic poles really were
the source of the magnetization, rather than electric currents, the
macroscopic magnetic field inside the material would be H, not B,
and the similarity of magnetic polarization to electric polarization
would be complete.

In the permanent magnet there are no free currents at all. 
Consequently, the line integral of H, according to Eq. 54, must be zero
around any closed path. You can see that it will be if the H lines
really look like the E lines in Fig. l0.2la, for we know the line integral
of that electrostatic field is zero around any closed path. In this
example of the permanent magnet, Eq. 55 does not apply. The
magnetization vector M is not proportional to H but is determined,
instead. by the previous treatment of the material. How this can
come about will be explained in the next section.

For any material in which M is proportional to H, so that Eq. 55
applies as well as the basic relation, Eq. 52, we have:
\begin{equation}
\end{equation}

B = H + 477M = (1 + 47rx,,,)H (56)

B is then proportional to H. The factor of proportionality,
(1 + 47rX,,,). is called the magnetic permeability and denoted usually
by it:
\begin{equation}
\end{equation}

B = pH (57)

The permeability ,u, rather than the susceptibility X, is customarily
used in describing ferromagnetism.

% p. 394
\section{Ferromagnetism}

Ferromagnetism has served and puzzled man for a long time.
The lodestone (magnetite) was known in antiquity, and the influence
on history of iron in the shape of compass needles was perhaps second
only to that of iron in the shape of swords. For nearly a century
our electrical technology has depended heavily on the circumstance
that one abundant metal happens to possess this peculiar property.
Nevertheless, it is only in recent years that anything like a fundamental
understanding of ferromagnetism has been achieved.

We have already described some properties of ferromagnets. In
a very strong magnetic field the force on a ferromagnetic substance
is in such a direction as to pull it into a stronger field, as for 
paramagnetic materials, but instead of being proportional to the product
of the field B and its gradient, the force is proportional to the gradient
itself. As we remarked at the end of Sec. 10.4, this suggests
that if the field is strong enough, the magnetic moment acquired by
the ferromagnet reaches some limiting magnitude. The direction
of the magnetic moment vector must still be controlled by the field,
for otherwise the force would not always act in the direction of increasing
field intensity.

In ``permanent'' magnets we observe a magnetic moment even in
the absence of any externally applied field, and it maintains its magnitude
and direction even when external fields are applied, if they
are not too strong. The field of the permanent magnet itself is always
present of course, and you may wonder whether it could not keep
its own sources lined up. However, if you look again at Fig. 10.2112
or Fig. 10.24, accepting the assurance that this does represent a real
magnet, you will notice that M is generally not parallel to either B
or H. This suggests that the magnetic dipoles must be clamped in
direction by something other than purely magnetic forces.

The magnetization observed in ferromagnetic materials is much
larger than we are used to in paramagnetic substances. Permanent
magnets quite commonly have fields in the range of a few thousand
gauss. A more characteristic quantity is the limiting value of the
magnetization, the magnetic moment per unit volume, which the
material acquires in a very strong field. This is called the saturation

magnetization. We can deduce the saturation magnetization of iron
from the data in the table (Sec. 10.11. In a field with a gradient of

% p. 395
1700 gauss/cm, the force on 1 g of iron was 4 x 105 dynes. From
Eq. 18, which relates the force on a dipole to the field gradient, we find
\begin{equation}
\end{equation}

m _ F _ 4 x 105 dynes
_ (dB/dz) _ 1700 gauss/cm
= 235 ergs/gauss (for 1 g) (58)

To get the moment per cubic centimeter we multiply m by the
density of iron, 7.8 g/cm3. The magnetization M is thus
\begin{equation}
\end{equation}

M = 235 )< 7.8 = 1830 ergs/gauss-cm3 (59)

It is 4¢rM, not M, that we should compare with field strengths in gauss.
For instance, if we had a very long rod of iron which had this amount
of magnetization as a permanent magnet, the field H inside would
be quite small (imagine the situation in Fig. 10.2119 stretched out
axially) so the field B in the iron would be approximately equal to
47rM, or some 23,000 gauss.

It is more interesting to see how many electron spin moments this
magnetization corresponds to. Dividing M by the electron moment
given in Fig. 10.14, 0.93 X l0‘20 erg/gauss, we get about 2 X 1023
spin moments per cubic centimeter. Now 1 cm3 of iron contains
about 1023 atoms. The limiting magnetization seems to correspond
to about two lined-up spins per atom. As most of the electrons in
the atom are paired off and have no magnetic effect at all, this indicates
that we are dealing with substantially complete alignment
of those few electron spins in the atom's structure that are at liberty
to point in the same direction.

A very suggestive fact about ferromagnets is this: A given ferromagnetic
substance, pure iron for example, loses its ferromagnetic
properties quite abruptly if heated to a certain temperature. Above
770°C. pure iron acts like a paramagnetic substance. Cooled below
770°C, it immediately recovers its ferromagnetic properties. This
transition temperature, called the Curie point after Pierre Curie who
was one of its early investigators. is different for different substances.
For pure nickel it is 358°C.

What is this ``ferromagnetic behavior'' which so sharply distinguishes
iron below 770°C from iron above 770°C, and from copper
at any temperature? It is the spontaneous lining up in one direction
of the atomic magnetic moments, which implies the alignment of
the spin axes of certain electrons in each iron atom. By spontaneous,
we mean that no external magnetic field need be involved. Over a
region in the iron large enough to contain millions of atoms, the spins
and magnetic moments of nearly all the atoms are pointing in the

% p. 396
 

same direction. Well below the Curie point --- at room temperature,
for instance, in the case of iron --- the alignment is nearly perfect.
If you could magically look into the interior of a crystal of metallic
iron and "see the elementary magnetic moments as vectors with
arrowheads on them, you might see sometlnflg like Fig. 10.25.

It is hardly surprising that a high temperature should destroy this
neat arrangement. Thermal energy is the enemy of order, so
to speak. A crystal, an orderly arrangement of atoms, changes to a
liquid, a much less orderly arrangement, at a sharply defined tem-
perature, the melting point. The melting point, like the Curie point,
is different for different substances. The analogy goes deeper, but it
involves ideas that are better left for your study of heat and statistical
physics in the latter part of the course. Let us concentrate here on
the ordered state itself. Two or three questions are obvious:

Question 1: What makes the spins line up and keeps them lined
up?

Question 2: How, if there is no external field present, can the
spins choose one direction rather than another? Why didn't all the
moments in Fig. 10.25 point down, or to the right, or to the left?

Question 3: If the atomic moments are all lined up, why isn't every
piece of iron at room temperature a strong magnet?

The answers to these three questions will help us to understand,
in a general way at least, the behavior of ferromagnetic materials
when an external field, neither very strong nor very weak, is applied.

That includes a very rich variety of phenomena which we haven't
even described yet.

Answer 1: For some reason connected with the quantum mechanics
of the structure of the iron atom, it is energetically favorable
for the spins of adjacent iron atoms to be parallel. This is not due
to their magnetic interaction. It is a stronger effect than that, and
moreover, it favors parallel spins whether like this M‘ or like this  --- > --- >
(dipole interactions don't work that way --- see Prob. 9.26). Now if
atom A (Fig. 10.26) wants to have its spin in the same direction as
that of its neighbors, atoms B, C, D, and E, and each of them prefers
to have its spin in the same direction as the spins of its neighbors,
including atom A, you can readily imagine that if a local majority
ever develops there will be a strong tendency to ``make it 
unanimous,'' and then the fad will spread.

% p. 397
Answer 2: Accident somehow determines which of the various
equivalent directions in the crystal is chosen, if we commence from
a disordered state --- as, for example, if the iron is cooled through its
Curie point without an';' external field applied. Pure iron consists of
body-centered cubic crystals. Each atom has eight nearest neigh-
bors. The symmetry .11‘ the environment imposes itself on every
physical aspect of the atom, including the coupling between spins.
In iron the cubic axes happen to be the axes of easiest magnetization.
That is, the spins like to point in the same direction, but they like
it even better if that direction is one of the six directions, :52, :9,
:2 (Fig. 10.27). This is important because it means that the spins
cannot easily swivel around en masse from one of the easy directions
to an equivalent one at right angles. To do so, they would have to
swing through less favorable orientations on the way. It is just this
hindrance that makes permanent magnets possible.

Answer 3: An apparently unmagnetized piece of iron is actually
composed of many domains, in each of which the spins are all lined
up one way, but in a direction different from that of the spins in
neighboring domains. On the average over the whole piece of
``unmagnetized'' iron, all directions are equally represented, so no
large-scale magnetic field results. Even in a single crystal the magnetic
domains establish themselves. The domains are usually microscopic
in the everyday sense of the word. In fact they can be made
visible under a low-power microscope. That is still enormous, of
course, on an atomic scale, so a magnetic domain typically includes
billions of elementary magnetic moments. Figure 10.28 depicts a
division into domains. The division comes about because it is
cheaper in energy than an arrangement with all the spins pointing
in one direction. The latter arrangement would be a permanent
magnet with a strong field extending out into the space around it.
The energy stored in this exterior field is larger than the energy
needed to turn some small fraction of the spins in the crystal, namely
those at a domain boundary, out of line with their immediate 
neighbors. The domain structure is thus the outcome of an 
energy-minimization contest.

If we wind a coil of wire around an iron rod, we can apply a magnetic
field to the material by passing a current through the wire. In
this field, moments pointing parallel to the field will have a lower
energy than those pointing antiparallel, or in some other direction.
This favors some domains over others; those that happen to have a

% p. 398
 

favorably oriented moment direction"? will tend to grow at the expense
of the others, if that is possible. A domain grows like a club,
that is, by expanding its membership. This happens at the 
boundaries. Spins belonging to an unfavored domain but located next to
the boundary with a favored domain, simply switch allegiance by
adopting the favored direction. That merely shifts the domain
boundary, which is nothing more than the dividing surface between
the two classes of spins. This happens rather easily in single crystals.
That is, a very weak applied field can bring about, through boundary
movement, a very large domain growth, and hence a large overall
change in magnetization. Depending on the grain structure of the
material however, the movement of domain boundaries can be
diflicult.

If the applied field does not happen to lie along one of the ``easy''
directions (in the case of a cubic crystal, for example,) the exhaustion
of the unfavored domains still leaves the moments not pointing
exactly parallel to the field. It may now take a considerably stronger
field to pull them into line with the field direction so as to create
finally, the maximum magnetization possible.

Let us look at the large-scale consequences of this, as they appear
in the magnetic behavior of a piece of iron under various applied
fields. A convenient experimental arrangement is an iron torus,
around which are wound two coils (Fig. 10.29). This afl"ords a
practically uniform field within the iron, with no end effects to complicate
matters. By measuring the voltage induced in one of the
coils we can determine changes in flux (I), and hence in B inside the
iron. If we keep track of the changes in B, starting from B = O, we
always know what B is. A current through the other coil establishes
H, which we take as the independent variable. If we know B and H,
we can always compute M. It is more usual to plot B rather than M,
as a function of H. A typical B --- H curve for iron is shown in
Fig. 10.30. Notice that the scales on abscissa and ordinate are vastly
different. If there were no iron in the coil, 1 oersted would be worth
exactly 1 gauss. Instead, when the field H is only a few oersteds.
B has risen to thousands of gauss. Of course B and H here refer to

an average throughout the whole iron ring; the fine domain structure
as such never exhibits itself.

TWe tend to use spins and moments almost interchangeably in this discussion. The
moment is an intrinsic aspect of the spin and if one is lined up so is the other. To be
meticulous, we should remind the reader that in the case of the electron the magnetic
moment and angular momentum vectors point in opposite directions (Fig. 10.14).

% p. 399
Starting with ``unmagnetized'' iron, B = 0 and H = 0, increasing
H causes B to rise in a conspicuously nonlinear way, slowly at
first, then more rapidly, then very slowly, finally flattening off. What
actually becomes constant in the limit is not B but M. In this graph
however, since M = (B  ---  H) /477, and H < B, the difference between
B and 47rM is not appreciable.

The lower part of the B-H curve is governed by the motion of
domain boundaries, that is by the growth of ``right-pointing'' domains
at the expense of ``wrong-pointing'' domains. In the flattening
part of the curve, the atomic moments are being pulled by ``brute
force'' into line with the field. The iron here is an ordinary 
polycrystalline metal, so only a small fraction of the microcrystals will
be fortunate enough to have an ``easy'' direction lined up with the
field direction.

If we now slowly decrease the current in the coil, thus lowering H,
the curve does not retrace itself. Instead, we find the behavior given
by the dashed curve in Fig. 10.31. This irreversibility is called
hysteresis. It is largely due to the domain boundary movements'
being partially irreversible. The reasons are not obvious from anything
we have said, but are well understood by physicists who work
on ferromagnetism. The irreversibility is a nuisance, and a cause
of energy loss in many technical applications of ferromagnetic
materials --- for instance, in alternating-current transformers. But it
is indispensable for permanent magnetization, and for such 
applications, one wants to enhance the irreversibility. Figure 10.31 shows
the corresponding portion of the B-H curve for a good permanent
magnet alloy. Notice that H has to become 600 oersteds in the
reverse direction before B is reduced to zero. If the coil is simply
switched OH and removed, we are left with B at 13,000 gauss, called
the remanence. Since H is zero, this is essentially the same as the
magnetization M, except for the factor 472-. The alloy has acquired
a permanent magnetization, that is, one that will persist indefinitely
if it is exposed only to. weak magnetic fields. All the information
that is stored on magnetic tapes, from music to computer programs,
owes its permanence to this physical phenomenon. Magnetic 
computer elements, magnetic core memories, and the like, involve the
same physics.

\fi
