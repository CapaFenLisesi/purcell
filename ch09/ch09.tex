\chapter{Electric fields in matter}

\section{Dielectrics}

The capacitor we studied in Chap. 3 consisted of two conductors,
insulated from one another, with nothing in between. The system
of two conductors was characterized by a certain capacitance $C$, a
constant relating the magnitude of the charge Q on the capacitor
(positive charge $Q$ on one plate, equal negative charge on the other)
to $V_{12}$, the difference in electrical potential between the two 
conductors:
\begin{equation}
  C = \frac{Q}{V_{12}}
\end{equation}
For the parallel-plate capacitor, two flat plates each of area $A\ \cmunit^2$
and separated by a distance $t$, we found that the capacitance is
given by
\begin{equation}
  C = \frac{A}{4\pi t}
\end{equation}
Capacitors like this can be found in some electrical apparatus. They
are called \emph{vacuum capacitors}\index{capacitor!vacuum}
and consist of plates enclosed in a highly
evacuated bottle. They are used chiefly where extremely high and
rapidly varying potentials are involved. Far more common, how-
ever, are capacitors in which the space between the plates is filled
with some nonconducting solid or liquid substance. Most of the
capacitors you have worked with in the laboratory are of that sort;
there are dozens of them in any television receiver. For conductors
embedded in a material medium, Eq. 2 doesn't agree with experiment.
Suppose we fill the space between the two plates shown in Fig. 9.1a
with a slab of plastic, as in Fig. 9.1b. Experimenting with this new
capacitor, we still find a simple proportionality between charge and
potential difierence, so that we can still \emph{define} a capacitance by Eq. 1.
But we find $C$ to be substantially \emph{larger} than Eq. 2 would 
predict.

Not only in the special devices called capacitors, but almost everywhere
in the world around us, the electric and magnetic fields exist
in the presence of matter rather than in a vacuum --- if not in dense
matter, then at least in a gas, namely air. All this is to remind us
that, except for our excursion in Chap. 4 into the subject of electrical
conduction, we have really been studying the electromagnetic field
in empty space populated only by certain point charges or smooth
% p. 299
charge distributions. Now we must seek to understand the interactions
of electric and magnetic fields with matter in bulk.

Two different approaches are open to us. Maintaining a large-scale,
or \emph{macroscopic}, point of view, we could see how the presence
of a block of homogeneous material like the plastic slab in Fig. 9.lb
affects the electric field in the space outside, where we can measure
the field. We could try to discover simple laws which would adequately
describe such effects in any system of conductors and in-
sulators. We would find that the macroscopic electric behavior of
homogeneous substances can indeed be characterized fairly simply
and completely. Equation 2, for example, needs only the insertion
on the right-hand side of a constant factor 5, characteristic of the particular
substance, to give correctly the capacitance of any capacitor
filled with that material. $\epsilon$ is called the \intro{dielectric constant} of that
substance, and the material itselfis generally referred to as a dielectric
when we are considering its behavior in an electric field. Dielectric
constants of some common substances are listed in Table \ref{table:dielectric-constants}. Once
the dielectric constant of a particular material has been determined,
perhaps by measuring the capacitance of one capacitor filled with it,
% p. 300
we are able to predict the behavior, not merely of two-plate 
capacitors, but of any electrostatic system made up of conductors and
pieces of that dielectric of any shape. That is, we can predict all
electric fields which will exist in the vacuum outside the dielectrics
for given charges or potentials on the conductors in the system.

\begin{table} % makes it floating
\caption{Dielectric constants of various substances}\label{table:dielectric-constants}
\begin{tabular}{lll}
\hline
\emph{Substance} & \emph{Conditions} & \emph{Dielectric constant} \\
\hline
Air                             & Gas, 0\degcunit, 1 atm & 1.00059 \\
Hydrogen chloride, HCl          & Gas, 0\degcunit, 1 atm & 1.0046 \\
Water, $\zu{H}_2\zu{O}$                      & Gas, 110\degcunit, 1 atm & 1.0126 \\
                                & Liquid, 20\degcunit & 80 \\
Benzene, $\zu{C}_6\zu{H}_6$               & Liquid, 20\degcunit        & 2.28 \\
Ammonia, $\zu{NH}_3$               & Liquid,  $-34\degcunit$    & 22 \\
Transformer oil               & Liquid, 20\degcunit      & 2.24 \\
Sodium chloride, NaCl           & Crystal, 20\degcunit               & 6.12 \\
Sulfur, S                 & Solid, 20\degcunit           & 4.0 \\
Quartz, $\zu{SiO}_2$                    & Crystal, 20\degcunit\ ($\perp$ optic axis)       & 4.34 \\
                                & Crystal, 20\degcunit\ ($\parallel$ optic axis)   & 4.27 \\
Polyethylene                 & Solid, 20\degcunit        & 2.25-2.3 \\
Neoprene                 & Solid, 20\degcunit            & 4.1 \\
Porcelain                 & Solid, 20\degcunit           & 6.0-8.0 \\
Paraffin wax                 & Solid, 20\degcunit        & 2.1-2.5 \\
Pyrex glass 7070                 & Solid, 20\degcunit    & 4.00
\end{tabular}
\end{table}

The theory which enables us to do this was fully worked out by
the physicists of the nineteenth century. Lacking a complete picture
of the atomic structure of matter, they were more or less obliged to
adopt a macroscopic description. From that point of view, the interior
of a dielectric is a featureless expanse of perfectly smooth
``mathematical jelly'' whose single electrical property distinguishing
it from a vacuum is a dielectric constant different from unity.

If we develop only a macroscopic description of matter in an
electric field, we shall find it hard to answer some rather obvious-
sounding questions --- or rather, hard to ask these questions in such
a way that they can be meaningfully answered. For instance, what
is the strength of the electric field \emph{inside} the plastic slab of Fig. 9.lb
when there are certain charges on the plates? Electric field strength
is defined by the force on a test charge. How can we put a test charge
inside a perfectly dense solid, without disturbing anything, and
measure the force on it? What would that force mean, if we did
measure it? You might think of boring a hole and putting the test
charge in the hole with some room to move around, so that you can
measure the force on it as on a free particle. But then you will be
measuring, not the electric field in the dielectric, but the electric field
in a cavity in the dielectric, which is quite a different thing.

Fortunately another line of attack is available to us, one that leads
up from the microscopic or \emph{atomic} level. We know that matter is
made of atoms and molecules; these in turn are composed of elementary
charged particles. We know something about the size and
structure of these atoms, and we know something about their arrangement
in crystals and fluids and gases. Instead of describing our
dielectric slab as a volume of structureless but nonvacuous jelly, we
shall describe it as a collection of molecules inhabiting a vacuum.
If we can find out what the electric charges in \emph{one} molecule do when
that molecule is all by itself in an electric field, we should be able to
understand the behavior of two such molecules a certain distance
apart in a vacuum. It will only be necessary to include the influence,
on each molecule, of any electric field arising from the other. This
is a vacuum problem. Now all we have to do is extend this to a population
of say $10^{20}$ molecules occupying a cubic centimeter or so of
% p. 301
vacuum, and we have our real dielectric. We hope to do this without
generating $10^{20}$ separate problems.

This program if carried through will reward us in two ways. We
shall be able at last to say something meaningful about the electric
and magnetic fields inside matter, answering questions such as the
one raised above. What is more valuable, we shall understand how
the macroscopic electric and magnetic phenomena in matter arise
from, and therefore reveal, the nature of the underlying atomic 
structure. We are going to study electric and magnetic effects separately.
We begin with dielectrics. Since our first goal is to describe the electric
field produced by an atom or molecule, it will help to make some
general observations about the electrostatic field external to any small
system of charges.

\section{The moments of a charge distribution}

An atom or molecule consists of some electric charges occupying
a small volume, perhaps \linebreak[4]$\sim (0.1\ \zu{nm})^3=10^{-24}\ \cmunit^3$ of space.
We are interested in the electric field outside that volume, which
arises from this rather complicated charge distribution. We shall
be particularly concerned with the field far away from the source,
by which we mean far away compared to the size of the source itself.
What features of the charge structure mainly determine the field at
remote points? To answer this, let's look at some arbitrary distribution
of charges and see how we might go about computing the field
at a point outside it. Figure 9.2 shows a charge distribution of some
sort located in the neighborhood of the origin of coordinates. It
might be a molecule consisting of several positive nuclei and quite
a large number of electrons. In any case we shall suppose it is described
by a given charge density function $\rho(x,y,z)$. $\rho$ is negative
where the electrons are and positive where the nuclei are. To find
the electric field at distant points we can begin by computing the
potential of the charge distribution. To illustrate, let's take some
point $A$ out on the $z$ axis. (Since we are not assuming any special
symmetry in the charge distribution, there is nothing special about
the $z$ axis.) Let $r$ be the distance of $A$ from the origin. The electric
potential at $A$, denoted by $\pot_A$, is obtained as usual by adding the
contributions from all elements of the charge distribution:
\begin{equation}
  \pot_A = \int \frac{\rho(x',y',z')\der v'}{R}
\end{equation}
% p. 302
In the integrand $\der v'$ is an element of volume within the charge 
distribution, $\rho(x',y',z')$ is the charge density there, and $R$ in the denominator
is the distance from $A$ to this particular charge element. The
integration is carried out in the coordinates $x'$, $y'$, $z'$, of course, and
is extended over all the region containing charge. We can express $R$
in terms of $r$ and the distance $r'$ from the origin to the charge element.
Using the law of cosines with $\theta$ the angle between $r'$ and the axis on
which $A$ lies:
\begin{equation}
  R = [r^2 + r'^2 - 2rr'\cos\theta]^{1/2}
\end{equation}
With this substitution for $R$ the integral becomes:
\begin{equation}
  \pot_A = \int \rho\der v'[r^2 + r'^2 - 2rr'\cos\theta]^{-1/2}
\end{equation}
Now we want to take advantage of the fact that for a distant point
like $A$, $r'$ is much smaller than $r$ for all parts of the charge distribution.
This suggests that we should expand the square root in Eq. 4 in
powers of $r'/r$. Writing
\begin{equation}
  [r^2 + r'^2 - 2rr'\cos\theta]^{-1/2} 
      = \frac{1}{r}\left[1+\left(\frac{r'^2}{r^2}-\frac{2r'}{r}\cos\theta\right)\right]^{-1/2}
\end{equation}
and using the expansion $(1+\delta)^{-1/2}=1-\frac{1}{2}\delta+\frac{3}{8}\delta^2\ldots$, we get,
after collecting together terms of the same power in $r'/r$:
\begin{equation}
  [r^2 + r'^2 - 2rr'\cos\theta]^{-1/2}
   = \frac{1}{r}\left[1+\frac{r'}{r}\cos\theta
                       +\left(\frac{r'}{r}\right)^2\frac{(3\cos^2\theta-1)}{2}
                       +\left(\begin{array}{c}\text{terms of}\\\text{higher power}\end{array}\right)\right]
\end{equation}
Now $r$ is a constant in the integration, so we can take it outside and
write the prescription for the potential at $A$ as follows:
\begin{equation}
  \pot_A =
        \frac{1}{r}  \underbrace{\int\rho\der v'}_{K_0}
       +\frac{1}{r^2}\underbrace{\int r'\cos\theta\:\rho\der v'}_{K_1}
       +\frac{1}{r^3}\underbrace{\int r'^2\frac{(3\cos^2\theta-1)}{2}\rho\der v'}_{K_2}
       +\ldots
\end{equation}
Each of the integrals above, $K_0$, $K_1$, $K_2$, and so on, has a value that
depends only on the structure of the charge distribution. Hence the
% p. 303
potential for all points along the $z$ axis can be written as a power series
in $1/r$ with constant coefficients:
\begin{equation}
  \pot_A = \frac{K_0}{r}+\frac{K_1}{r^2}+\frac{K_2}{r^3}+\ldots
\end{equation}

To finish the problem we would have to get the electric field at all
other points, in order to calculate the electric field as  $-\operatorname{grad}\pot$. We
have gone far enough, though, to bring out the essential point: The
behavior of the potential at large distances from the source will be
dominated by the first term in this series whose coefficient is not zero.

Let us look at these coefficients more closely. The coeflicient $K_0$
is $\int\rho\der v'$, which is nothing but the total charge in the distribution.
If we have equal amounts of positive and negative charge, as in a
neutral molecule, $K_0$ will be zero. For a singly ionized molecule $K_0$
will have the value $e$. If $K_0$ is not zero, then no matter how large
$K_1$, $K_2$, etc., may be, if we go out to a sufliciently large distance the
term $K_0/ r$ will win out. Beyond that, the potential will approach that
of a point charge at the origin and so will the field. This is hardly
surprising.

Suppose we have a neutral molecule, so that $K_0$ is zero. Our interest
now shifts to the second term, with coefficient $K_1 = \int r'\cos\theta\:\rho\der v'$.
Since $r' \cos \theta$ is simply $z'$, this term measures the relative 
displacement, in the direction toward $A$, of the positive and negative charge.
It has a nonzero value for the distributions sketched in Fig. 9.3, where
the densities of positive and of negative charge have been indicated
separately. In fact, all the distributions shown there have approximately
the same value of $K_1$.

It is worth noting that if the distribution is neutral the value of $K_1$
is independent of the position of the origin. That is, if we replace
$z'$ by $(z' + z_0')$, thus in effect shifting the origin, the value of the integral
is not changed: $\int(z' + z_0')\rho\der v' = \int z'\rho\der v' + z_0'\int\rho\der v'$ and the
latter integral is always zero for a neutral distribution.

Evidently if $K_0 = 0$ and $K_1\ne 0$, the potential along the $z$ axis will
vary asymptotically (that is, with ever-closer approximation as we
go out to larger distances) as $1/r^2$. We expect the electric field
strength, then, to behave asymptotically like $1/r^3$, in contrast to the
$1 /r^2$ dependence of the field from a point charge. Of course we have
discussed only the potential on the $z$ axis. We will return to the question
of the exact form of the field after getting a general view of the
situation.

% p. 304

If $K_0$ and $K_1$ are both zero, and $K_2$ is not, the potential will behave
like $1/r^3$ at large distances, and the field strength will fall off with
the inverse fourth power of the distance. Figure 9.4 shows a charge
distribution for which $K_0$ and $K_1$ are both zero (and would be zero
no matter what direction we had chosen for the $z$ axis) while $K_2$ is
not zero.

The quantities $K_0$, $K_1$, $K_2$, \ldots are related to what are called the
\emph{moments}\index{moments} of the charge distribution. Using this language, we call $K_0$,
which is simply the net charge, the \intro{monopole moment}, or \emph{monopole
strength}. $K_1$ is one component of the \intro{dipole moment} of the 
distribution. The dipole moment has the dimensions \emph{charge} times 
\emph{displacement}; it is a vector and our $K_1$ is its $z$ component. The third
constant $K_2$ is related to the \intro{quadrupole moment} of the distribution,
the next to the \intro{octupole moment}, and
so on.\footnote{It can be shown that decomposition of the source into various \intro{multipoles}, if carried
through completely, uniquely specifies the charge distribution. In other words, if we
know all the multipole strengths we can ``in principle'' deduce $\rho(x',y',z')$. This is not
very useful. The quadrupole and higher moments are not vectors, by the way, but
more complicated entities.}

The advantage to us of describing a charge distribution by this
hierarchy of moments is that it singles out just those features of the
charge distribution which determine the field at a great distance. If
we were concerned only with the field in the immediate neighborhood
of the distribution, it would be a fruitless exercise. For our main task,
understanding what goes on in a dielectric, it turns out that \emph{only} the
monopole strength (the net charge) and the dipole strength of the
molecular building blocks matter. We can ignore all other moments.
And if the building blocks are neutral, we have only their dipole
moments to consider.

\section{The potential and field of a dipole}

The dipole contribution to the potential at the point $A$, distance
$r$ from the origin, was given by $(1/r^2)\int r' \cos \theta\:\rho\der v'$. We can write
$r' \cos \theta$, which is just the projection of $\vc{r}'$ on the direction toward $A$,
as $\hat{\vc{r}}\cdot\vc{r}'$. Thus we can write the potential without reference to any
arbitrary axis as
\begin{equation}
  \pot_A = \frac{1}{r^2}\int \hat{\vc{r}}\cdot\vc{r}'\rho\der v'
      = \frac{\rhat}{r^2}\cdot\int \vc{r}'\rho\der v'
\end{equation}
which will serve to give the potential at any point. The integral on
the right in Eq. 9 is the \emph{dipole moment} of the charge distribution. It
% p. 305
is a vector, obviously, with the dimensions \emph{charge} times \emph{distance}. We
shall denote the dipole moment vector by $\vc{p}$:
\begin{equation}
  \vc{p} = \int \vc{r}'\rho\der v'
\end{equation}
Using the dipole moment $\vc{p}$, we can rewrite Eq. 9 as
\begin{equation}
  \pot(\vc{r}) = \frac{\rhat\cdot\vc{p}}{r^2}
\end{equation}
The electric field is the negative gradient of this potential. To see
what the dipole field is like, locate a dipole $\vc{p}$ at the origin, pointing
in the $z$ direction (Fig. 9.5). With this arrangement,
\begin{equation}
  \pot = \frac{p\cos\theta}{r^2}
\end{equation}
% p. 306
The potential and the field are, of course, symmetrical around the
$z$ axis. Let's work in the $xz$ plane, where $\cos \theta = z/ (x^2 + z^2)^{1/2}$. In
that plane,
\begin{equation}
  \pot = \frac{pz}{(x^2 + z^2)^{3/2}}
\end{equation}
The components of the electric field are:
\begin{align}
\begin{split}
  E_x = -\frac{\partial\pot}{\partial x} &= \frac{3pxz}{(x^2 + z^2)^{5/2}} = \frac{3p\sin\theta\cos\theta}{r^3} \\
  E_z = -\frac{\partial\pot}{\partial z} 
   &= p\left[\frac{3z^2}{(x^2 + z^2)^{5/2}}-\frac{1}{(x^2 + z^2)^{3/2}}\right] \\
   &= \frac{p(3\cos^2\theta-1)}{r^3}
\end{split}
\end{align}

Proceeding out in any direction from the dipole, we find the electric
field strength falling off as $1/r^3$, as we had anticipated. Along
the $z$ axis the field is parallel to the dipole moment $\vc{p}$, with magnitude
$2p/r^3$. In the equatorial plane the field points antiparallel to $\vc{p}$ and
has the value $-\vc{p}/r^3$.

This field may remind you of one we have met before. Remember
the point charge over the conducting plane, with its ``image charge.''
% p. 307
The simplest charge distribution with a dipole moment is two
point charges, $+q$ and  $-q$, separated by a distance $s$. For a system
of point charges Eq. 10 takes the form of a sum. The dipole moment
of our point-charge pair is just $qs$, and the vector points in the direction
from negative charge to positive. In Fig. 9.6 we have sketched
the field of this pair of charges, mainly to emphasize that the field
near the charges is \emph{not} a dipole field. This charge distribution has
many multipole moments, indeed infinitely many, so it is only the
``far field'' at distances $r \gg s$ that can be represented as a dipole field.

To generate a complete dipole field right in to the origin we would
have to let $s$ shrink to zero while increasing $q$ without limit so as to
keep $p = qs$ finite. This highly singular abstraction is not very 
interesting. We know that our molecular charge distribution will have
complicated near fields, so we could not easily represent the near
region in any case. Fortunately we shall not need to.

\section{The torque and the force on a dipole in an external field}

Suppose two charges $q$ and  $-	q$ are mechanically connected so
that $s$, the distance between them, is fixed. You may think of the
charges as stuck on the end of a short nonconducting rod of length $s$.
We shall call this object a dipole. Its dipole moment $p$ is simply $qs$.
Let us put the dipole in an external electric field, that is, the field from
some other source. The field of the dipole itself does not concern
us now. Consider first a uniform electric field, as in Fig. 9.7a. The
positive end of the dipole is pulled toward the right, the negative end
toward the left, by a force of strength $Eq$. The net force on the object
is zero and so is the torque, in this position. A dipole which makes
some angle $\theta$ with the field direction as in Fig. 9.7b obviously experiences
a torque. In general, torque $\btorque$ is $\vc{r} \times \vc{F}$ where $\vc{F}$ is the force
applied at a distance $\vc{r}$ from the origin (Vol. 1, Chap. 6). Taking the
origin in the center of the dipole, so that $r = s/2$, we have
\begin{equation}
  \btorque = \vc{r}\times\vc{F}_+ + (-\vc{r})\times\vc{F}_-
\end{equation}
$\btorque$ is a vector perpendicular to the figure, and its magnitude is:
\begin{equation}
  \torque = \frac{s}{2}Eq\sin\theta + \frac{s}{2}Eq\sin\theta = sqE\sin\theta = pE\sin\theta
\end{equation}
This can be written simply
\begin{equation}
  \btorque = \vc{p}\times\vc{E}
\end{equation}

% p. 308

The orientation of the dipole in Fig. 9.7a has the lowest energy.
Work has to be done to rotate it into any other position. Let us calculate
the work required to rotate the dipole from a position parallel
to the field, through some angle $\theta_0$, as shown in Fig. 9.7c. Rotation
through an infinitesimal angle $\der\theta$ requires an amount of work $\torque\der\theta$
Thus the total work done is
\begin{equation}
  \int_0^{\theta_0} \torque\der\theta 
    = \int_0^{\theta_0} pE\sin\theta\der\theta
    = pE(1-\cos\theta_0)
\end{equation}
To reverse the dipole, turning it end for end, corresponds to $\theta_0=\pi$
and requires an amount of work equal to $2pE$.

The net force on the dipole in any \emph{uniform} field is zero, obviously.
regardless of its orientation. In a nonuniform field the forces on the
two ends of the dipole will generally not be exactly equal and 
opposite, and there will be a net force on the object. A simple example
is a dipole in the field of a point charge $Q$. If the dipole is oriented
radially as in Fig. 9.8a, with the positive end nearer the positive
charge $Q$, the net force will be outward, and its magnitude will be
\begin{equation}
  F = (q)\frac{Q}{r^2}+(-q)\frac{Q}{(r+2)^2}
\end{equation}
For $s\ll r$, we need only evaluate this to first order in $s/r$, which we
do as follows:
\begin{equation}
  F = \frac{qQ}{r^2}\left[1-\frac{1}{\left(1+\frac{s}{r}\right)^2}\right]
    \approx \frac{qQ}{r^2}\left[1-\frac{1}{1+\frac{2s}{r}}\right]
    \approx \frac{2sqQ}{r^3}
\end{equation}
In terms of the dipole moment $p$, this is simply
\begin{equation}
  F = \frac{2pQ}{r^3}
\end{equation}

With the dipole at right angles to the field, as in Fig. 9.8b, there is
also a force. Now the forces on the two ends, though equal, are not
exactly opposite in direction.

It is not hard to work out a general formula for the force on a dipole
in a nonuniform electric field. The force depends essentially on the
gradients of the various components of the field. In general, the
$x$ component of the force on a dipole of moment $\vc{p}$ is
\begin{equation}
  F_x = \vc{p}\cdot\operatorname{grad} E_x
\end{equation}
with corresponding formulas for $F_y$ and $F_z$.

% p. 309

\section{Atomic and molecular dipoles; induced dipole moments}

In describing the charge distribution in an atom or molecule we
shall have to use classical terms to depict a quantum mechanical
system. Also, we shall be treating as static a structure in which the
particles are, in some sense, continually in motion. Later in the
course, in Vol. IV, you will see how quantum mechanics, far from discrediting
the picture we are about to sketch, reassuringly supports it.

Consider the simplest atom, the hydrogen atom, which consists of
a nucleus and one electron. If you imagine the negatively charged
electron revolving around the positive nucleus like a planet around
the sun --- as in the original atomic model of Niels Bohr\index{Bohr model} --- you will
conclude that the atom has, at any one instant of time, an electric
dipole moment. The dipole moment vector $\vc{p}$ points parallel to the
electron-proton radius vector, and its magnitude is $e$ times the
electron-proton distance. The direction of this vector is continually
and rapidly changing as the electron circles around its orbit. To be
sure, the \emph{time average} of $\vc{p}$ will be zero for a circular orbit, but we
should expect the periodically changing dipole moment components
to generate rapidly oscillating electric fields and electromagnetic
radiation. The absence of such radiation in the normal hydrogen
atom was one of the great paradoxes of early quantum physics.
Modern quantum mechanics tells us that it is better to think of the
hydrogen atom in its lowest energy state (the usual condition of most
of the hydrogen atoms in the universe) as a spherically symmetrical
structure with the electronic charge distributed, in the time average,
over a cloud surrounding the nucleus. Nothing is revolving or
oscillating. If we could take a snapshot with an exposure time
shorter than $10^{-16}$ seconds, we might discern an electron localized
some distance away from the nucleus. But for processes involving
times much longer than that we have, in effect, a smooth distribution
of negative charge surrounding the nucleus and extending out in all
directions with steadily decreasing density. The total charge in this
distribution is just  $-e$, the charge of one electron. Roughly half of
it lies within a sphere of radius 0.05 nm. The
density decreases exponentially outward; a sphere only 0.22 nm
in radius contains 99 percent of the charge.

A similar picture is the best one to adopt for other atoms and 
molecules. We can treat the nuclei in molecules as point charges; for our
present purposes their size is too small to matter. The entire electronic
structure of the molecule is to be pictured as a single cloud of
negative charge of smoothly varying density. The shape of this cloud
% p. 310
and the variation of charge density within it will of course be different
for different molecules. But at the fringes of the cloud the density
will always fall off exponentially, so that it makes some sense to talk
of the size and shape of the molecular charge distribution.

Figure 9.9 represents the charge distribution in the normal hydrogen
atom. It is a cross section through t-he spherically symmetrical
cloud, with the density suggested by shading. Obviously the dipole
moment of such a distribution is zero. The same is true of any atom
in its state of lowest energy, no matter how many electrons it 
contains, for in all such states the electron distribution has spherical
symmetry. It is also true of any ionized atom, though an ion, of
course, has a ``monopole moment,'' that is, a net charge.

So far we have nothing very interesting. But now let us put the
hydrogen atom in an electric field supplied by some external source,
as in Fig. 9.10. The electric field distorts the atom, pulling the negative
charge down and pushing the positive nucleus up. The distorted
atom will have an electric dipole moment because the ``center of
gravity'' of the positive charge and that of the negative charge no
longer coincide.

We can use a makeshift model of the hydrogen atom to estimate,
in order of magnitude, the amount of distortion to be expected.
Suppose that in the absence of an electric field the negative electric
charge $e$ is distributed with \emph{constant} density throughout a sphere of
radius $a$, outside which it is zero. Figure 9.1 1 shows this crude substitute
for the real distribution depicted in Fig. 9.9. Assume that
when the field $\vc{E}$ is applied, this ball of negative charge keeps its shape
and density and is merely displaced, relative to the nucleus, so that
the nucleus ends up some distance $b$ from the center of the sphere
(Fig. 9.12). In equilibrium, the force on the nucleus due to the electric
field $\vc{E}$, a force of $eE$ dynes acting upward, must be balanced by
the downward attraction exerted on the nucleus by the negative
charge cloud, which pulls the nucleus toward its center. To find the
magnitude of the latter force, we recall that inside a spherical charge
distribution, at a point $b$ cm from the center, the electric field is
simply that due to the charge inside a sphere of radius $b$. In this case
the amount of charge \emph{inside} the sphere of radius $b$ is $(b/a)^3e$, since $e$
is the amount of charge inside the sphere of radius $a$. At the location
of the nucleus, therefore, the field arising from the electron cloud is
just $(1/b^2)e(b/a)^3$ or $eb/a^3$. Setting this field strength equal to that
of the applied field $E$ gives the equilibrium condition:
\begin{equation}
  E = \frac{eb}{a^3} \qquad \text{from which} \qquad b=\frac{a^3E}{e}
\end{equation}

% p. 311

As a good round number let's put in 0.1 nm for $a$.
We have suggested that a radius of that magnitude would include
most of the charge in the true distribution. For $E$ we'll try 100 
statvolts/cm. That is 30,000 volts/cm, a pretty strong electric field as
fields go in the laboratory. With these assumptions, Eq. 23 yields
for $b$ the magnitude $2 \times10^{-13}\ \cmunit$.
The distortion is \emph{very slight}. The
displacement is about $10^{-5}$ of the atomic radius, not much more than
the radius of the nucleus. The resulting electric dipole moment is $eb$,
so that the relation between dipole moment and applied field, in this
model, is
\begin{equation}
  p = eb = a^3E
\end{equation}
The direction of the dipole moment vector is upward, that is, in the
same direction as the electric field.

Notice that the dipole moment is simply proportional to the applied
field. We can expect that this will be true in the real atom, at
least for small distortions, and our calculation strongly suggests that
any reasonable laboratory field perturbs an atom only very slightly.
Any atom can be polarized in this way. We say that the dipole
moment is \emph{induced}\index{induced dipole moment}\index{dipole moment!induced}
by the electric field $\vc{E}$. In every case we find that
$p$ is proportional to $\vc{E}$:
\begin{equation}
  \vc{p} = \alpha\vc{E}
\end{equation}
The constant $\alpha$ is a property of the atom called the \emph{atomic
polarizability}.\index{polarizability}\index{atomic polarizability}

\begin{table}[h] % makes it floating
\caption{Atomic polarizabilities, in units of $10^{-24}\ \cmunit^3$}\label{table:polarizabilities}
\begin{tabular}{llllllllll}
\hline
\emph{Element} & H & He & Li & Be & C & Ne & Na & A & K \\
$\alpha=$      & 0.66 & 0.21 & 12 & 9.3 & 1.5 & 0.4 & 27 & 1.6 & 34 \\
\hline
\end{tabular}
\end{table}

For our model of the hydrogen atom the polarizability $\alpha$ is equal
to $a^3$. Notice that $\alpha$ has the dimensions of volume. An exact
quantum mechanical calculation of the polarizability of the hydrogen
atom predicts $\alpha = (9/2)a_0^3$, where $a_0$ is the ``Bohr radius,''
$0.52 \times 10^{-8}\ \cmunit$, the characteristic distance in the H atom structure
in its normal state. The electric polarizabilities of several species of
atoms, experimentally determined, are given in Table \ref{table:polarizabilities}. The 
examples given are arranged in order of increasing numbers of 
electrons. Notice the wide variations in $\alpha$. If you are acquainted with
the Periodic Table of the elements, you may discern something systematic
here. Hydrogen and the alkali metals, lithium, sodium, and
% p. 312
potassium, which occupy the first column of the Periodic Table, have
large values of $\alpha$ and these increase steadily with increasing atomic
number, from hydrogen to potassium. The noble gases have much
smaller atomic polarizabilities, but these also increase as we proceed,
within the family, from helium to neon to krypton. Apparently the
alkali atoms, as a class, are easily deformed by an electric field,
whereas the electronic structure of a noble gas atom is much stiffer.
It is the loosely bound outer, or ``valence,'' electron in the alkali atom
structure that is responsible for the easy polarizability.

A molecule, too, develops an induced dipole moment when an
electric field is applied to it. The methane molecule depicted in
Fig. 9.13 is made from four hydrogen atoms arranged at the corners
of a tetrahedron around the central carbon atom. This object has
an electrical polarizability, determined experimentally, of
\begin{equation*}
  2.6\times 10^{-24}\ \cmunit^3
\end{equation*}
It is interesting to compare this with the sum of the polarizabilities of
a carbon atom and four isolated hydrogen atoms. Taking the data
from Table \ref{table:polarizabilities}, we find 
$\alpha_\zu{C}+4\alpha_\zu{H} = 4.1 \times 10^{-24}\ \cmunit^3$. Evidently
the binding of the atoms into a molecule has somewhat altered the
electronic structure. Measurements of atomic and molecular 
polarizabilities have long been used by chemists as clues to molecular
structure.

\section{The polarizability tensor}

Molecules are necessarily less symmetrical than atoms. This raises
the possibility of an induced dipole moment \emph{not} parallel to the electric
field that induced it. Consider the carbon dioxide molecule. It is
a linear ``cigar-shaped'' molecule with constituent atoms arranged
as shown in Fig. 9.14a. It would be surprising if this electronic structure
were equally stilf against longitudinal and transverse deforma-
tion. In general, we should expect that an electric field applied
parallel to the axis would cause an induced dipole moment different
in magnitude from that induced by a field of the same strength applied
at right angles to the molecular axis. Indeed, the observed
polarizability of the $\zu{CO}_2$ molecule is $4.05 \times 10^{-24}\ \cmunit^3$ for a field
applied parallel to the axis, and a little less than half that for a transverse
field. The molecule has two polarizabilities, which we might
label $\alpha_\parallel$ and $\alpha_\perp$. What happens if we apply a field in some other
direction, as in Fig. 9.14b? That can be easily predicted. Because
% p. 313
we are dealing with a linear\footnote{We
are also dealing with a linear molecule (atoms arranged in a straight line)!
Linear has, of course, totally different meanings in the two usages.}
phenomenon (effect directly proportional
to cause) the superposition principle holds. We can resolve
the field $\vc{E}$ into components parallel and perpendicular to the molecular
axis. $E_\parallel = E \cos \theta$ and $E_\perp = E \sin \theta$. We can imagine these
components applied separately, and then combine the resulting
moment vectors. $E_\parallel$ induces a moment along the molecular axis of
magnitude $p_\parallel = \alpha_\parallel E_\parallel = \alpha_\parallel E \cos \theta$.
$E_\perp$ causes a moment perpendicular
to the axis: $p_\perp = \alpha_\perp E \sin \theta$. These combine to make the dipole
moment $\vc{p}$ caused by the original field $\vc{E}$. The dipole moment
vector $\vc{E}$ is not parallel to $\vc{E}$ if $\alpha_\parallel\ne \alpha_\perp$. It points instead more nearly
in the direction of easy polarization. (Can you think of a mechanical
analog for this behavior?)

This example shows that the polarizability of a molecule is not a
simple number, a scalar, but rather a set of coeflicients that express
a linear dependence of the components of one vector, $\vc{p}$ in this 
example, on those of another, $\vc{E}$. Such a set of coefficients is called a
tensor. The most general relation of this sort involves
nine coeflicients, and might be written in this way:
\begin{align}
  p_x &= \alpha_{xx}E_x+\alpha_{xy}E_y+\alpha_{xz}E_z \\
  p_y &= \alpha_{yx}E_x+\alpha_{yy}E_y+\alpha_{yz}E_z \\
  p_z &= \alpha_{zx}E_x+\alpha_{zy}E_y+\alpha_{zz}E_z 
\end{align}
The nine $\alpha$'s defined in this way form what is known as the 
polarizability tensor.\index{polarizability tensor}

In the example of the $\zu{CO}_2$ molecule, if we orient the $x$ axis along
the axis of the molecule, the coefiicients become $\alpha_{xx}=\alpha_\parallel$;
$\alpha_{yy}=\alpha_{zz}=\alpha_\perp$; and the six other coefficients are zero. Had we
chosen some other direction for the coordinate axis, say at $30\degunit$ to the
molecular axis, a field $\vc{E}$ in the $x$ direction, as in Fig. 9.14c would cause
a dipole moment $\vc{p}$ having a component in the $z$ direction. So $\alpha_{zx}$
would not be zero. (You can find the value it would have by resolving
$\vc{E}$ into components parallel and perpendicular to the molecular
axis, finding the polarization induced by these, and then the $z$ component
of the resultant.) Thus the elements of the polarizability
tensor will depend on the orientation of the coordinate axes. They
must transform under a rotation of coordinate axes in such a way as
to preserve invariant the relation between the vectors $\vc{E}$ and $\vc{p}$. This
relation can only depend on the direction of $\vc{E}$ with respect to the
physical axis of the molecule, and not on how we happen to lay out
% p. 314
the $x$, $y$, and $z$ axes. We shall not work out here the rules by which
the tensor coefficients transform. They are analogous to the rules
for the transformation of the components of a vector. If you want
to see how it goes with minimum labor, you might work it out for the
two-dimensional case, as suggested in Prob. 9.23.

In the polarizability tensor oz only six of the nine coefficients are
independent. It can be proved that
$\alpha_{xy}=\alpha_{yx}$, $\alpha_{xz}=\alpha_{zx}$, $\alpha_{yz}=\alpha_{zy}$.
That is, the square array of nine numbers is always symmetrical
about the ``down-right'' diagonal. The symmetry of the
tensor expresses a most remarkable physical fact which deserves
some thought. It means that a field $\vc{E}$ applied in the $x$ direction
always causes a $z$ component of polarization exactly equal to the
$x$ component of polarization that would be caused by an equal field
applied along the $z$ direction. If you think this is obvious or trivial,
ponder the fact that it is true even for a molecule that has no symmetry
whatever, such as the molecule shown in Fig. 9.15. It is a kind
of ``reciprocity'' theorem which, like the equality of the mutual 
inductances we proved in Sec. 7.7, arises not from mere geometrical
symmetry but from something more general. If you wonder how it
can be proved, Prob. 9.22 will show you.

An important corollary of the symmetry of $\alpha$ is the fact that it is
always possible to orient the axes, relative to the molecular 
framework, so that the ``off-diagonal'' coefficients, $\alpha_{xy}$, etc., will be zero.
In these coordinates the polarizability of the molecule is completely
described by three numbers $\alpha_{xx}$, $\alpha_{yy}$, $\alpha_{zz}$.
And this is even true for
a molecule which has, itself, no symmetry at all. We shall not use
these facts in our limited study of dielectrics. They are extremely
important in the investigation of the optical properties of molecules
and are perhaps even more familiar to chemists, nowadays, than to
physicists. The chief purpose of this digression on the polarizability
tensor was to acquaint you, by means of this easily visualized ex-
ample, with the nature of a tensor.

\section{Permanent dipole moments}

Some molecules are so constructed that they have electric dipole
moments even in the absence of an electric field. They are unsymmetrical
already in their normal state. The molecule shown in
Fig. 9.15 is an example. A simpler example is provided by any
diatomic molecule made out of dissimilar atoms, such as hydrogen
chloride, HCl. There is no point on the axis of this molecule about
% p. 315
which the molecule is symmetrical fore and aft; the two ends of the
molecule are physically different. It would be a pure accident if the
``center of gravity'' of the positive charge and that of the negative
charge happened to fall at the same point along the axis. When
the HCl molecule is formed from the originally spherical H and Cl
atoms, the electron of the H atom shifts partially over to the Cl 
structure, leaving the hydrogen nucleus partially denuded. So there is
some excess of positive charge at the hydrogen end of the molecule
and a corresponding excess of negative charge at the chlorine end.
The magnitude of the resulting electric dipole moment, 
$1.03\times10^{-18}\ \zu{esu}\unitdot\cmunit$, is
equivalent to shifting one electron about one-fiftieth of a nanometer.
By contrast the hydrogen atom in a field of 
30 kilovolts/cm, with the polarizability listed in Table 9.2, acquires an induced
moment less than $10^{-22}\ \zu{esu}\unitdot\cmunit$. Permanent dipole moments,
when they exist, are as a rule enormously larger than any moment
that can be induced by ordinary laboratory electric 
fields.\footnote{
There is a good reason for this. The internal electric fields in atoms and molecules
are naturally of the order of $e/(10^{-8}\ \cmunit)^2$ which is roughly $10^9$ volts/cm! We
cannot apply such a field to matter in the laboratory for the closely related reason that
it would tear the matter to bits.} Because
of this, the distinction between \emph{polar} molecules, as molecules with
``built in'' dipole moments are called, and \emph{nonpolar} molecules is very
sharp.

We said at the beginning of Sec. 9.5 that the hydrogen atom had,
at any instant of time, a dipole moment. But then we dismissed it
as being zero in the time average, on account of the rapid motion of
the electron. Now we seem to be talking about molecular dipole
moments as if a molecule were an ordinary stationary object like a
baseball bat whose ends could be examined at leisure to see which
was larger! Molecules move more slowly than electrons, but their
motion is rapid by ordinary standards. Why can we credit them with
``permanent'' electric dipole moments? If this inconsistency was
bothering you, you are to be commended. The full answer can't be
given without some quantum mechanics, but the diflerence essentially
involves the time scale of the motion. The time it takes a molecule
to interact with its surroundings is generally \emph{shorter} than the
time it takes the intrinsic motion of the molecule to average out the
dipole moment smoothly." Hence the molecule \emph{really acts} as if it had
the moment we have been talking about. A very short time qualifies
as ``permanent'' in the world of one molecule and its neighbors.

% p. 316

Some common polar molecules are shown in Fig. 9.16, with the
direction and magnitude of the permanent dipole moment indicated
for each. The water molecule has an electric dipole moment because
it is bent in the middle, the O-H axes making an angle of about $105\degunit$
with one another. This is a structural oddity with the most far-reaching
consequences. The dipole momentof the molecule is largely
responsible for the properties of water as a solvent, and it plays a
decisive role in chemistry that goes on in an aqueous environment.
It is hard to imagine what the world would be like if the $\zu{H}_2\zu{O}$ 
molecule, like the $\zu{CO}_2$ molecule, had its parts arranged in a straight line;
probably we wouldn't be here to observe it. We hasten to add that
the shape of the $\zu{H}_2\zu{O}$ molecule is not a capricious whim of Nature.
Quantum mechanics has revealed clearly why a molecule made of
an eight-electron atom joined to two one-electron atoms must prefer
to be bent.

The behavior of a polar substance as a dielectric is strikingly different
from that of material composed of nonpolar molecules. The
dielectric constant of water is about 80, that of methyl alcohol 33,
while a typical nonpolar liquid might have a dielectric constant
around 2. In a nonpolar substance the application of an electric field
induces a slight dipole moment in each molecule. In the polar substance
dipoles are already present in great strength but, in the
absence of a field, are pointing in random directions so that they have
no large-scale effect. An applied electric field merely \emph{aligns} them to
a certain degree. In either process, however, the macroscopic elfects
will be determined by the net amount of polarization per unit volume.

\section{The electric field caused by polarized matter}

Suppose we build up a block of matter by assembling a very large
number of molecules in a previously empty region of space. Suppose
too that each of these molecules is polarized in the same direction.
For the present we need not concern ourselves with the nature of the
molecules or with the means by which their polarization is 
maintained. We are interested only in the electric field \emph{they} produce when
they are in this condition; later we can introduce any fields from other
sources that might be around. If you like, you can imagine that these
are molecules with permanent dipole moments that have been lined
up neatly and frozen in position. All we need to specify is $N$, the
number of dipoles per cubic centimeter, and the moment of each
dipole $\vc{p}$. We shall assume that $N$ is so large that any macroscopically
% p. 317
small volume $\der v$ contains quite a large number of dipoles. The total
dipole strength in such a volume is $\vc{p}N\der v$. At any point far away
from this volume element compared to its size, the electric field from
these particular dipoles will be practically the same if they were replaced
by a single dipole moment of strength $\vc{p}N\der v$. We shall call $\vc{p}N$
the density of polarization, and denote it by $\vc{P}$, a vector quantity with
the dimensions $\zu{charge}\unitdot\cmunit/\cmunit^3$, or 
$\zu{charge}/\cmunit^2$. Then $\vc{P}\der v$ is the
dipole moment to be associated with any small volume element $\der v$
for the purpose of computing the electric field at a distance. By the
way, our matter has been assembled from neutral molecules only;
there is no net charge in the system or on any molecule, so we have
\emph{only} the dipole moments to consider as sources of a distant field.

In Fig. 9.17 there is shown a slender column, or cylinder, of this
polarized material. Its cross section is $\der a$, and it extends vertically
from $z_1$ to $z_2$. The polarization density $\vc{P}$ within the column is uniform
over the length and points in the positive $z$ direction. We are
about to calculate the electrical potential, at some external point, of
this column of polarization. An element of the cylinder, of height $\der z$,
has a dipole moment $\vc{P} \der v = \vc{P} \der a \der z$. Its contribution to the potential
at the point $A$ can be written down by referring back to our
formula Eq. 12 for the potential of a dipole.
\begin{equation}
  \der\pot_A = \frac{P\der a\der z\:\cos\theta}{r^2}
\end{equation}
The potential due to the entire column is
\begin{equation}
  \pot_A = P\der a\:\int_{z_1}^{z_2} \frac{\der z\:\cos\theta}{r^2}
\end{equation}
This is simpler than it looks: $\der z\:\cos\theta$ is just $-\der r$, so that the 
integrand is a perfect differential, $\der(1 /r)$. The result of the integration is
then
\begin{equation}
  \pot_A = P\der a\left(\frac{1}{r_2}-\frac{1}{r_1}\right)
\end{equation}

Equation 29 is precisely the same as the expression for the potential
at $A$ that would be produced by two point charges, a positive
charge of magnitude $P\der a$ sitting on top of the column at a distance $r_2$
from $A$, and a negative charge of the same magnitude at the bottom
of the column. The source consisting of a column of uniformly
polarized matter is equivalent, at least so far as its field at all \emph{external}
points is concerned, to two concentrated charges.

% p. 318

We can prove this rigorously in another way without any 
mathematics. Consider a small section of the column of height $\der z$, containing
dipole moment in amount $P\der a\der z$. Let us make an 
imitation or substitute for this by taking an unpolarized insulator of the
same size and shape and sticking a charge $+P\der a$ on top of it, and a
charge $-P\der a$  on the bottom. This little block now has the same
dipole moment as that bit of our original column, and therefore it
will make an identical contribution to the field at any remote point A.
(The field inside our substitute, or very close to it, may be different
from the field of the original --- we don't care about that.) Now make
a whole set of such blocks and stack them up to imitate the polarized
column. They must give the same field at A as the whole column
does, for each block gave the same contribution as its counterpart
in the original (Fig. 9.17b). Now see what we have! At every joint
the positive charge on the top of one block coincides with the negative
charge on the bottom of the block above it, making charge zero.
The only charges left uncompensated are the negative charge $-P\der a$ da
on the bottom of the bottom block and the positive charge $+P\der a$ on
the top of the top block. Seen from a distant point such as $A$, these
look like point charges. We conclude, as before, that two such
charges produce at $A$ exactly the same field as does our whole column
of polarized material.

With no further calculation we can extend this to a slab, or right
cylinder, of any proportions uniformly polarized in a direction perpendicular
to its parallel faces (Fig. 9.18a). The slab can simply be
subdivided into a bundle of columns and the potential outside will
be the sum of the contributions of the columns, each of which can
be replaced by a charge at either end. The charges on the top, $P\der a$
on each column end of area $\der a$, make up a uniform sheet of surface
charge of density $\sigma=P$ esu per unit area. We conclude that the
potential everywhere \emph{outside} a uniformly polarized slab or cylinder
is precisely what would result from two sheets of surface charge
located where the top and bottom surfaces of the slab were located,
carrying the constant surface charge density $\sigma=+P$ and $\sigma=-P$
respectively (Fig. 9.18b).

We are not quite ready to say anything about the field \emph{inside} the
slab. However, we do know the potential at all points on the surface
of the slab, top, bottom, or sides. Any two such points, $A$ and $B$,
can be connected by a path running entirely through the external
field, so that the line integral $\int\vc{E}\cdot\der\vc{s}$ is entirely determined by the
external field. It must be the same as the integral along the path $A'B'$
% p. 319
in Fig. 9.18b. A point literally on the surface of the dielectric might
be within range of the intense molecular fields, the ``near field'' of
the molecule that we have left out of account. Let's agree to define
the boundary of the dielectric as a surface far enough out from the
outermost atomic nucleus --- 1 or 2 nm would be margin
enough --- so that at any point outside this boundary, the ``near fields''
of the individual atoms make a negligible contribution to the whole
line integral from $A$ to $B$.

With this in mind, let's look at a rather thin, wide plate of polarized
material, of thickness $t$ shown in cross section in Fig. 9.19a. Figure
9.19b shows, likewise in cross section, the equivalent sheets of charge.
For the system of two charge sheets we know the field, of course, in
the space both outside and between the sheets. The field strength
inside, well away from the edges, must be just $4\pi\sigma$, pointing down,
and the potential difference between points $A'$ and $B'$ is therefore
$4\pi\sigma t$ statvolts. The \emph{same potential difference} must exist between
corresponding points $A$ and $B$ on our polarized slab, because the
entire external field is the same in the two systems.

Is the field identical inside, too? Certainly \emph{not}, because the slab
is full of positive nuclei and electrons, with fields of millions of volts
per centimeter pointing in one direction here, another direction there.
But one thing is the same: The line integral of the field, reckoned over
any internal path from $A$ to $B$, must be just $\pot_B-\pot_A$, which as we have
seen is the same as $\pot_{B'}-\pot_{A'}$, which is equal to $4\pi\sigma t$, or $4\pi P t$. This
must be so because the introduction of atomic charges, no matter
what their distribution, cannot destroy the conservative property of
the electric field, expressed in the statement that $\int\vc{E}\cdot\der\vc{s}$ is independent
of path, or $\curl \vc{E} = 0$.

% p. 320

We know that in Fig. 9.19b the potential difference between the
top and bottom sheets is nearly constant, except near the edges, because
the interior electric field is practically uniform. Therefore in
the central area of our polarized plate the potential difierence between
top and bottom must likewise be constant. In this region the
line integral $\int_A^B\vc{E}\cdot\der\vc{s}$ taken from \emph{any} point $A$ on top of the slab to
\emph{any} point $B$ on the bottom, by \emph{any} path, must always yield the same
value $4\pi Pt$. Figure 9.20 is a ``magnified view'' of the central region
of the slab, in which the polarized molecules have been made to look
something like $\zu{H}_2\zu{O}$ molecules all pointing the same way. We have
not attempted to depict the very intense fields that exist between the
molecules, and inside them. (One nanometer distant from a water
molecule its field amounts to several hundred kilovolts/cm, as you
can discover from Table 9.1 and Eq. 14.) You must imagine some
rather complicated field configurations in the neighborhood of each
molecule. Now the $\vc{E}$ that stands in $\int\vc{E}\cdot\der\vc{s}$ means the \emph{total electric
field} at a given point in space, inside or outside a molecule; it includes
these complicated and intense fields just mentioned. We
have reached the remarkable conclusion that \emph{any} path through this
welter of charges and fields, whether it dodges molecules or penetrates
them, must yield the same value for the path integral, namely
the value we find in the system of Fig. 9.19b where the field is quite
uniform and has the strength $4\pi P$.

This tells us that the \emph{spatial average} of the electric field within our
polarized slab must be $-4\pi\vc{P}$. By the spatial average of a field $\vc{E}$
over some volume $V$, which we might denote by $\langle\vc{E}\rangle_V$, we mean
precisely this:
\begin{equation}
  \langle\vc{E}\rangle_V = \frac{1}{V}\int_V \vc{E}\der v
\end{equation}

One way to sample impartially the field in many equal small $\der v$'s
into which $V$ might be divided would be to measure the field along
each line in a ``fiber bundle'' of closely spaced parallel lines. We
have just seen that the line integral of $\vc{E}$ along any or all such paths
is the same as if we were in a constant electric field of strength $-4\pi\vc{P}$.
That is the justification for the conclusion that $\langle\vc{E}\rangle=-4\pi\vc{P}$.

This average field is a \emph{macroscopic} quantity. The volume over
which we take the average should be large enough to include very
many molecules, otherwise the average will fluctuate from one such
volume to the adjoining one. The average field $\langle\vc{E}\rangle$ defined by Eq. 30
is really the only kind of \emph{macroscopic} electric field in the interior of
% p. 321
a dielectric that we can talk about. It provides the only satisfactory
answer, in the context of a macroscopic description of matter, to the
question, ``What is the electric field inside a dielectric material?''

The $\vc{E}$ in the integrand on the right, in Eq. 30, we may call the
\emph{microscopic} field. If we send someone out to measure the field values
we need for the path integral, he will be measuring electric fields in
vacuum, in the presence, of course, of electric charge. He will need
very tiny instruments, for he may be called on to measure the field
at a particular point just inside one end of a certain molecule. Have
we any right to talk in this way about taking the line integral of $\vc{E}$
along some path that skirts the southwest corner of a particular
molecule and then tunnels through its neighbor? Yes. The justification
is the massive evidence that the laws of electromagnetism work
down to a scale of distances much smaller than atomic size. We can
even describe an experiment which would serve to measure the
average of the microscopic electric field along a path defined well
within the limits of atomic dimensions. All we have to do is shoot
an energetic charged particle, an alpha particle for example, through
the material. From the net change in its momentum the average
electric field that acted on it, over its whole path, could be inferred.

Let's work a numerical example involving polarized material.
Imagine a disk 1 cm in radius and 0.3 cm thick, as in Fig. 9.21a. Let
the density of molecules in the disk be $3\times10^{22}\ \text{molecules}/\cmunit^3$.
Suppose they are all polar molecules, each with a dipole moment
$1.8 \times 10^{-18}\ \zu{esu}\unitdot\cmunit$. Finally --- and this is a rather far-fetched
assumption --- suppose they are all lined up, like the molecules in
Fig. 9.20, with their dipole moments pointing in the same direction,
parallel to the axis of the disk. We'll discuss the field at a point $A$
inside the disk, a point $B$ just outside the disk, and a point $C$ 10 cm
away on the axis. The polarization $\vc{P}$ has the value
\begin{align}
\begin{split}
  P &= Np = (3\times10^{22}\ \cmunit^{-3})\times(1.8 \times 10^{-18}\ \zu{esu}\unitdot\cmunit) \\
    &= 5.4\times10^4\ \zu{esu}/\cmunit^2
\end{split}
\end{align}
The equivalent sheets of charge are shown in Fig. 9.21c. If these
sheets extended to infinity, the field at the point $A$ between them
would be simply $4\pi\sigma$, or $6.8 \times 10^5\ \zu{statvolts}/\cmunit$. That will be a
pretty good approximation in this case, because the separation of
the sheets is relatively small compared to their diameter. Actually,
the field will be slightly less. The field just outside, at point $B$, would
be zero for infinite sheets, but in the actual case will have a relatively
small value, pointing to the right. The discontinuity in $E$ at the
% p. 322
positive sheet will be exactly $4\pi\sigma$, or $4\pi P$. If we needed to calculate the
fields at $A$ or $B$ precisely, we could use the formula we worked out
in Chap. 2 for the field of a disk of surface charge, superposing the
fields of two such disks appropriately located. To estimate the field
at a remote point like $C$, all we need to know is the total dipole
moment of the object. Seen from $C$, it doesn't matter much how the
individual dipoles are distributed. The disk acts like a single dipole
of strength 
$p_{tot}=\text{volume}\times P=0.942\ \cmunit^3\times 5.4 \times 10^4\ \zu{esu}/\cmunit^2
=5.1\times10^4\ \zu{esu}\unitdot\cmunit$.
The field on the axis of such a dipole, 10 cm
away, is
\begin{equation}
  E_C = \frac{2p_tot}{r^3} = \frac{10.2\times10^4\ \zu{esu}\unitdot\cmunit}{(10\ \cmunit)^3}
       = 102\ \zu{statvolts}/\cmunit
\end{equation}

\section{The capacitor filled with dielectric}

We have been a long time getting around to the dielectric-filled
capacitor, but now we can bring to bear on this problem some understanding
of the dielectric itself. Consider first the two conducting
plates in vacuum with charge  $-Q$ on the upper plate, charge $+Q$
on the lower. Figure 9.22a is just Fig. 9.1a, with which we began
this chapter, in cross section. The field between the plates $\vc{E}_0$ is
equal to $4\pi Q/A$, and points upward. The potential difference between
the plates, $\pot_{12}$, is equal to $4\pi Qt/A$. The capacitance of the
empty capacitor, $C_0$, is given by the now familiar formula
\begin{equation}
  C_0 = \frac{Q}{\pot_{12}} = \frac{A}{4\pi t}
\end{equation}
Now bring between the plates a dielectric. The field will polarize
the atoms or molecules in the dielectric. We cannot, at this stage,
predict the magnitude of the induced dipole moment of each molecule
because the field that acts on the molecule in this situation is
not just the field $\vc{E}_0$ but includes a contribution from the other molecules
as well. The direction of the polarization, at any rate, will be
parallel to $\vc{E}_0$, for a dielectric that is isotropic. Let us denote the
magnitude of the polarization density, whatever it may be, by $\vc{P}$.

We now have the system shown in Fig. 9.22c, consisting of two real
sheets of charge plus a slab of polarized material. It is the \emph{superposition}
of the two charge distributions we have already analyzed,
that of Fig. 9.22a, and that of Fig. 9.19a, shown again in Fig. 9.22b.
The electric field will be the sum of the fields of those two 
distributions, the field $\vc{E}_0$ of two real charge sheets of surface charge density
% p. 323
$\sigma = Q/A$, plus the field $\vc{E}'$ of the two charge sheets of density $\sigma' = P$,
to which the polarized slab is equivalent. Notice that $\vc{E}'$ is directed
\emph{opposite} to $\vc{E}_0$ because $\vc{P}$ is in the same direction as $\vc{E}_0$; the sheet of
positive equivalent charge lies next to the negatively charged plate.
The reason for this, of course, is that the negative charge on the plate
polarized the atoms of the dielectric by pulling on their positive parts
and pushing on their negative parts, thus drawing positive charge
closer to that plate. In the interior of the capacitor, then, the electric
field $\vc{E}$ is
\begin{equation}
  \vc{E} = \vc{E}_0+\vc{E}' = \vc{E}_0-4\pi\vc{P}
\end{equation}
The magnitude of the potential difference between the plates has
become
\begin{equation}
  \pot_{12} = (E_0-4\pi P)t
\end{equation}
The charge on the capacitor is still the same. If the plates were to
be connected by a wire, the charge $Q$ would drain off, the dielectric
meanwhile relaxing back to its unpolarized condition. Because the
potential difference has been reduced by the factor $(E_0-4\pi P)/E_0$,
as compared to the vacuum capacitor with the same charge, the
capacitance, $C = Q/\pot_{12}$, has been increased by the reciprocal of that
factor:
\begin{equation}
  C = C_0\frac{E_0}{E_0-4\pi P}
\end{equation}
It is better to express this in terms of $\vc{E}$, the electric field (macroscopic,
or \emph{average}, field) that now exists within the capacitor. Since
$E_0 = E + 4\pi P$, from Eq. 34, we have
\begin{equation}
  C = C_0 \frac{E+4\pi P}{E} = C_0\left(1+4\pi\frac{P}{E}\right)
\end{equation}
The ratio of $P$ to $E$ is an intrinsic property of the dielectric material.
This ratio is called the \intro{electric susceptibility} of the material, and the
symbol $\chi_e$ is customarily used for it. The ratio is dimensionless.
The entire quantity in parenthesis in Eq. 37 is called the \intro{dielectric
constant} of the material denoted by $\epsilon$.
\begin{equation}
  P = \chi_e E \qquad \epsilon=1+4\pi\chi_e
\end{equation}
These are just definitions; the physics is in Eqs. 34 and 37.

Strictly speaking, filling the vacuum capacitor with dielectric material
increases the capacitance by precisely the factor $\epsilon$ only if we
% p. 324
fill the space all around it as well as the space between the plates.
In the example above we have tacitly assumed that the plate separation
$t$ is so small compared to the width of the plates that ``edge
effects,'' including the small amount of charge that is on the outside
of the plates near the edge (see Fig. 3.11b) are negligible. A quite
general statement can be made about a system of conductors of any
shape or arrangement which is entirely immersed in a homogeneous,
isotropic dielectric --- for instance in a large tank of oil. With any
charges whatever, $Q_1$, $Q_2$, etc., on the various conductors, the macroscopic
field $\vc{E}_{med}$ everywhere in the dielectric medium is just $1/\epsilon$ times
the field $\vc{E}_{vac}$ that would exist there with the same charges on the
same conductors in vacuum (Fig. 9.23). Of course all potential
differences are reduced by the same factor, $1/\epsilon$.

Our unfinished business consists of two problems of a quite different
nature:

\begin{enumerate}[(i)]
\item We need to understand the behavior of any system of insulators
and conductors, given the dielectric constants of the
materials involved. That is, we want to be able to calculate
the electric fields outside the dielectrics and the macroscopic
$\vc{E}$ inside, whatever boundary conditions are imposed
in terms of potentials and charges on the conductors.\label{polzn-problem-1}

\item The quantitative relation between the bulk polarizability of
a material, expressed by the susceptibility $\chi_e$, and the polar-
izability of the atoms or molecules of which the dielectric
is composed, remains rather mysterious. To discover it
we shall have to decide what field a polarizable atom
actually feels when the space average, or macroscopic, field
in its vicinity is known. What a fixed atom feels is not the
space average field, but another field we can call the local
field. It is the local field, $\vc{E}_{loc}$, that actually induces the
dipole moment of the atom. This question calls for another
``microscopic'' look at the interior of the dielectric.
\end{enumerate}

We turn first to problem (\ref{polzn-problem-1}).

\section{The field of a polarized sphere}

The solid sphere in Fig, 9.24a is supposed to be uniformly 
polarized, as if it had been carved out of the substance of the slab in
Fig. 9.18a. What must the electric field be like, both inside and outside
the sphere? This is an instructive problem, and the results will
be useful in other ways. $\vc{P}$ as usual will denote the density of
% p. 325
polarization, constant in magnitude and direction throughout the volume
of the sphere. The polarized material could be divided, like the slab
in Fig. 9.l8a, into columns parallel to $\vc{P}$, and each of these replaced
by a charge of magnitude $(P \times \text{column cross section})$ at top and
bottom. Thus the field we seek is that of a surface charge distribution
spread over a sphere with density $\sigma = P \cos \theta$. The factor $\cos\theta$
enters, as should be obvious from the figure, because a column of
cross section $\der a$ intercepts on the sphere a patch of surface of area
$\der a/\cos \theta$. Figure 9.24b is a cross section through this shell of equivalent
surface charge in which the density of charge has been indicated
by the varying thickness of the black semicircle above (positive
charge density) and the light semicircle below (negative charge
density).

If it has not already occurred to you, this figure may suggest that
we think of the polarization $\vc{P}$ as having arisen from the slight upward
displacement of a ball filled uniformly with positive charge of
volume density $\rho$, relative to a ball of negative charge of density  $-\rho$.
That would leave uncompensated positive charge poking out at the
top and negative charge showing at the bottom, varying in amount
precisely as $\cos\theta$ over the whole boundary. In the interior, where
the positive and negative charge densities still overlap, they would
% p. 326
exactly cancel one another. Taking this view, we see a very easy way
to calculate the field \emph{outside} the shell of surface charge. Any
spherical charge distribution, as we know, has an external field the
same as if its entire charge were concentrated at the center. So the
superposition of two spheres of total charge $+Q$ and  $-Q$,
 with their centers separated by a small displacement $s$, will
produce an external field the same as that of two point charges $Q$ and
$-Q$, $s$ cm apart. That is just a dipole with dipole moment $p_0=Qs$..

A microscopic description of the polarized substance leads us to
the same conclusion. In Fig. 9.25a the molecular dipoles actually
responsible for the polarization $\vc{P}$ have been crudely represented as
consisting individually of a pair of charges $q$ and  $-q$, $s$ cm apart, to
make a dipole moment $p = qs$. With $N$ of these per cubic 
centimeter, $P = Np = Nqs$, and the total number of such dipoles in the
sphere is $(4\pi/3)r_0^3N$. The positive charges, considered separately
(Fig. 9.25b), are distributed throughout a sphere with total charge
content $Q=(4\pi/3)r_0^3Nq$, and the negative charges occupy a similar
sphere with its center displaced (Fig. 9.25c). Clearly each of these
charge distributions can be replaced by a point charge at its center,
if we are concerned with the field well outside the distribution. ``Well
outside'' means far enough away from the surface so that the actual
% p. 327
graininess of the charge distribution doesn't matter, and of course
that is something we always have to ignore when we speak of the
macroscopic fields. So for present purposes the picture of overlapping
spheres of uniform charge density and the description in
terms of actual dipoles in a vacuum are 
equivalent,\footnote{This may have been obvious enough, but
we have labored the details in this one
case to allay any suspicion that the ``smooth-charge-ball'' picture, which is so different
from what we know the interior of a real substance to be like, might be leading us
astray.} and show that
the field outside the distribution is the same as that of a single dipole
located at the center. The moment of this dipole $p_0$ is simply the
total polarization in the sphere:
\begin{equation}
  p_0 = Qs = \frac{4\pi}{3} r_0^3Nqs = \frac{4\pi}{3} r_0^3P
\end{equation}
The quantities $Q$ and $s$ have, separately, no significance and may
now be dropped from the discussion.

The external field of the polarized sphere is that of a central
dipole $p_0$, not only at a great distance from the sphere; it is the pure
dipole field right down to the surface, macroscopically speaking.
All we had to do to construct Fig. 9.26, a representation of the external
field lines, was to block out a circular area from Fig. 9.5.

The internal field is a different matter. Let's look at the electric
potential, $\pot(x,y,z)$. We know the potential at all points on the
spherical boundary because we know the external field. It is just
the dipole potential, $p_0 \cos \theta/r^2$, which on the spherical boundary of
radius $r_0$ becomes
\begin{equation}
  \pot = p_0\frac{\cos\theta}{r_0^2} = \frac{4\pi}{3}Pr_0\cos\theta
\end{equation}
Since $r_0\cos\theta=z$, we see that the potential of a point on the sphere
depends only on its $z$ coordinate:
\begin{equation}
  \pot = \frac{4\pi}{3}Pz
\end{equation}

The problem of finding the internal field has boiled down to this:
Equation 41 gives the potential at every point on the boundary of
the region, inside which $\pot$ must satisfy Laplace's equation. 
According to the Uniqueness Theorem we proved in Chap. 3, that suffices
to determine $\pot$ throughout the interior. If we can find a solution,
it must be the solution. Now the function $Cz$, where $C$ is any 
constant, satisfies Laplace's equation, so Eq. 41 has actually handed us
% p. 328
the solution to the potential \emph{in the interior} of the sphere. It is the
potential of a uniform electric field in the  $-z$ direction:
\begin{equation}
  E_z = -\frac{\partial\pot_\text{in}}{\partial z}
      = -\frac{\partial}{\partial z}\left[\frac{4\pi Pz}{3}\right]
      = -\frac{4\pi P}{3}
\end{equation}
As the direction of $\vc{P}$ was the only thing that distinguished the $z$ axis,
we can write our result in more general form:
\begin{equation}
  \vc{E}_\text{in} = -\frac{4\pi \vc{P}}{3}
\end{equation}
This is the macroscopic field $\vc{E}$ in the polarized material.

Figure 9.27 shows both the internal and external field. At the
upper pole of the sphere, the strength of the upward-pointing external
field is, from Eq. 14 for the field of a dipole,
\begin{equation}
  E_z = \frac{2p_0}{r^3} = \frac{2(4\pi r_0^3 P/3)}{r_0^3} = \frac{8\pi P}{3} \qquad (\text{outside})
\end{equation}
which is just twice the magnitude of the downward-pointing internal
field.

This example illustrates the general rules for the behavior of the
field components at the surface of a polarized medium. $\vc{E}$ is discontinuous
at the boundary of a polarized medium exactly as it
would be at a surface in vacuum which carried a surface charge
density $\sigma=P_n$. The symbol $P_n$ stands for the component of $\vc{P}$
normal to the surface outward. It follows that the normal component
of $\vc{E}$ must change abruptly by an amount $4\pi P_n$, while the
component of $\vc{E}$ parallel to the boundary remains continuous, that
is, has the same value on both sides of the boundary. Indeed, at the
north pole of our sphere the net change in $E_z$ is $8\pi P/3-(-4\pi P/3)$
or $4\pi P$. Referring to Eq. 14 for the dipole field, you can check that
the component of $\vc{E}$ parallel to the surface is continuous from inside
to outside everywhere on the sphere.

None of these conclusions depends on how the polarization of
the sphere was caused. Assuming any sphere \emph{is} uniformly polarized,
Fig. 9.27 shows \emph{its} field. Onto this can be superposed any field from
other sources, thus representing many possible systems. This will
not affect the discontinuity in $\vc{E}$ at the boundary of the polarized
medium. The rules just stated therefore apply in any system, the
discontinuity in $\vc{E}$ being determined solely by the existing polarization.

% p. 329
\section{A dielectric sphere in a uniform field}

As an example, let us put a sphere of dielectric material characterized
by a dielectric constant $\epsilon$ into a homogeneous electric field $\vc{E}_0$
like the field between the parallel plates of a vacuum capacitor,
Fig. 9.28. Let the sources of this field, the charges on the plates,
be far from the sphere so that they do not shift as the sphere is 
introduced. Then whatever the field may be in the vicinity of the sphere,
it will remain practically $\vc{E}_0$ at a great distance. That is what is meant
by putting a sphere into a uniform field. The total field $\vc{E}$ is no
longer uniform in the neighborhood of the sphere. It is the \emph{sum} of
the uniform field $\vc{E}_0$ of the distant sources and a field $\vc{E}'$ generated
by the polarized matter itself:
\begin{equation}
  \vc{E} = \vc{E}_0 + \vc{E}'
\end{equation}
The field $\vc{E}'$ depends on the polarization $\vc{P}$ of the dielectric, which in
turn depends on the value of $\vc{E}$ inside the sphere:
\begin{equation}
  \vc{P} = \chi_e\vc{E} = \frac{\epsilon-1}{4\pi}\vc{E}
\end{equation}

We don't know yet what the total field $\vc{E}$ is; we only know that
Eq. 46 has to hold at any point inside the sphere. If the sphere becomes
uniformly polarized, an assumption that will need to be
justified by our results, the relation between the polarization of the
sphere and its own field $\vc{E}'$, at points inside, is given already by Eq. 43.
(In Eq. 43 we were using the symbol $\vc{E}$ for this field; in that case it
was the only field present.)
\begin{equation}
  \vc{E}'_\text{in} = -\frac{4\pi \vc{P}}{3} 
\end{equation}
Now we have enough equations to eliminate $\vc{P}$ and $\vc{E}'$, which should
give us a relation connecting $\vc{E}$ and $\vc{E}_0$. Using Eqs. 45 to 47 we find:
\begin{equation}
  \vc{E} = \vc{E}_0-\frac{4\pi \vc{P}}{3} = \vc{E}_0-\frac{\epsilon-1}{3}\vc{E}
\end{equation}
Solving for $\vc{E}$,
\begin{equation}
  \vc{E} = \left(\frac{3}{2+\epsilon}\right)\vc{E}_0
\end{equation}
Because $\epsilon$ is greater than one, the factor $3/ (2 + \epsilon)$ will be less than
one; the field inside the dielectric is weaker than $\vc{E}_0$. The polarization
is
\begin{equation}
  \vc{P} = \frac{\epsilon-1}{4\pi} \vc{E} = \frac{3}{4\pi}\left(\frac{\epsilon-1}{\epsilon+2}\right)\vc{E}_0
\end{equation}
% p. 330
The assumption of uniform polarization is now seen to be 
self-consistent.\footnote{That is what makes this system easy to deal 
with. For a dielectric cylinder of
finite length in a uniform electric field, the assumption would not work. The field $\vc{E}'$
of a uniformly polarized cylinder --- for instance one with its length about equal to its
diameter --- is not uniform inside the cylinder. (What must it look like?) Therefore
$\vc{E} = \vc{E}_0 + \vc{E}'$ cannot be uniform --- but in that case 
$\vc{P}=\chi_e\vc{E}$ could not be uniform after
all. In fact it is only dielectrics of ellipsoidal shape, of which the sphere is a special
case, which acquire uniform polarization in a uniform field.}
To compute the total field $\vc{E}$ outside the sphere we must
add vectorially to $\vc{E}_0$ the field of a central dipole with dipole moment
equal to $\vc{P}$ times the volume of the sphere. Some field lines of $\vc{E}$,
both inside and outside the dielectric sphere, are shown in Fig. 9.29.

\section[Charge in a dielectric]{The field of a charge in a dielectric medium, and Gauss's law}

Suppose that a very large volume of homogeneous dielectric has
somewhere within it a concentrated charge $Q$, not part of the regular
molecular structure of the dielectric. Imagine, for instance, that a
small metal sphere has been charged and then dropped into a tank
of oil. As was stated earlier, the electric field in the oil is simply $1/\epsilon$
times the field that $Q$ would produce in a vacuum.
\begin{equation}
  E = \frac{Q}{\epsilon r^2}
\end{equation}
It is interesting to see how Gauss's law works out. The surface integral
of $\vc{E}$ (which is the macroscopic, or space average, field, 
remember) taken over a sphere surrounding $Q$, gives $4\pi Q/\epsilon$, if we believe
Eq. 51, and \emph{not} $4\pi Q$. Why not? The answer is that $Q$ is not the only
charge inside the sphere. There are also all the charges that make up
the atoms and molecules of the dielectric. Ordinarily any volume
of the oil would be electrically neutral. But now the oil is radially
polarized, which means that the charge $Q$, assuming it is positive, has
pulled in toward itself the negative charge in the oil molecules and
pushed away the positive charges. Although the displacement may
be only very slight in each molecule, still on the average any sphere
we draw around $Q$ will contain more oil-molecule negative charge
than oil-molecule positive charge. Hence the net charge in the
sphere, including the ``foreign'' charge $Q$ at the center, is less than $Q$.
In fact it is $Q/\epsilon$.

It is often useful to distinguish between the ``foreign'' charge $Q$
and the charges that make up the dielectric itself. Over the former
% p. 331
we have some degree of control --- charge can be added to or removed
from an object, such as the plate of a capacitor. This is often called
\emph{free} charge.\index{charge!free}\index{free charge}
The other charges, which are integral parts of the atoms
or molecules of the dielectric, are usually 
called ``\emph{bound}'' charge.\index{charge!bound}\index{bound charge}
Structural charge might be a better name. These charges are not
mobile but more or less elastically bound, contributing, by their
slight displacement, to the polarization.

One can devise a vector quantity which is related by something
like Gauss's law to the free charge only. In the system we have just
examined, a point charge $Q$ immersed in a dielectric, the vector $\epsilon\vc{E}$
has this property. That is, $\int\epsilon\vc{E}\cdot\der\vc{a}$, taken over some closed surface $S$,
equals $4\pi Q$ if $S$ encloses $Q$, and zero if it does not. By superposition,
this must hold for any collection of free charges described by a free
charge density $\rho_\text{free}(x,y,z)$ in an infinite homogeneous dielectric
medium:
\begin{equation}
  \int_S \epsilon\vc{E}\cdot\der\vc{a} = 4\pi\int_V\rho_\text{free}\der v
\end{equation}
where $V$ is the volume enclosed by the surface $S$. An integral relation
like this implies a ``local'' relation between the divergence of
the vector field $\epsilon\vc{E}$ and the free charge density:
\begin{equation}
  \div(\epsilon\vc{E}) = 4\pi\rho_\text{free}
\end{equation}
Since $\epsilon$ has been assumed to be constant throughout the medium,
Eq. 53 tells us nothing new. However it can help us to isolate the
role of the bound charge. In any system whatever, the fundamental
relation between electric field $\vc{E}$ and total charge density 
$\rho_\text{free}+\rho_\text{bound}$
remains valid:
\begin{equation}
  \div\vc{E} = 4\pi(\rho_\text{free}+\rho_\text{bound})
\end{equation}
From Eqs. 53 and 54 it follows that
\begin{equation}
  \div\:(\epsilon-1)\vc{E} = -4\pi\rho_\text{bound}
\end{equation}
According to Eq. 38, $(\epsilon-1)\vc{E}=4\pi\vc{P}$, so Eq. 55 implies that
\begin{equation}
  \div\vc{P} = -\rho_\text{bound}
\end{equation}

Equation 56 is a statement about two aspects of the bound charge
distribution in any neighborhood, and about nothing else. 
Therefore it cannot depend on conditions elsewhere in the system, nor
on how the particular arrangement of bound charges is maintained.
Any arrangement of bound charge which has a certain local excess,
% p. 332
per unit volume, of nuclear protons over atomic electrons must
represent a polarization with a certain divergence. So Eq. 56 must
hold universally, not just in the unbounded dielectric. You can get
a feeling for the identity expressed in Eq. 56 by imagining a few polar
molecules arranged to give a polarization with a positive divergence
(Fig. 9.30). The dipoles point outward, which necessarily leaves a
little concentration of negative charge in the middle. Of course,
Eq. 56 refers to averages over volume elements so large that $\vc{P}$ and
$\rho_\text{bound}$ can be treated as smoothly varying quantities.

From Eqs. 54 and 56 we get the relation
\begin{equation}
  \div(\vc{E}+4\pi\vc{P}) = 4\pi\rho_\text{free}
\end{equation}
This is quite independent of any relation between $\vc{E}$ and $\vc{P}$. It is not
limited to those materials, which we call dielectrics, in which $\vc{P}$ is
proportional to $\vc{E}$.

It is customary to give the combination $\vc{E}+4\pi\vc{P}$ a special name,
the \intro{electric displacement} vector, and its own symbol, $\vc{D}$. That is. we
define $\vc{D}$ by
\begin{equation}
  \vc{D} = \vc{E}+4\pi\vc{P}
\end{equation}
In an isotropic dielectric, $\vc{D}$ is simply $\epsilon\vc{E}$, but the relation
\begin{equation}
  \div\vc{D} = 4\pi\rho_\text{free}
\end{equation}
holds in any situation in which the macroscopic quantities $\vc{P}$, $\vc{E}$, and
$\rho$ can be defined.

The appearance of Eq. 59 may suggest that we should look on $\vc{D}$
as a vector field whose source is the free charge distribution $\rho_\text{free}$, in
the same sense that the total charge distribution $\rho$ is the source of $\vc{E}$.
That would be wrong. The electrostatic field $\vc{E}$ is uniquely determined
--- except for the addition of a constant field --- by the charge
distribution $\rho$ because, supplementing the law $\div\vc{E}=4\pi\rho$, there
is another universal condition, $\curl\vc{E} = 0$. It is \emph{not} true, in general,
that $\curl\vc{D} = 0$. Thus the distribution of free charge is not sufficient
to determine $\vc{D}$ through Eq. 59. Something else is needed, such as
the boundary conditions at various dielectric surfaces. The boundary
conditions on $\vc{D}$ are of course merely an alternate way of expressing
the boundary conditions involving $\vc{E}$ and $\vc{P}$, already stated near
the end of Sec. 9.10.

In the approach we have taken to electric fields in matter the introduction
of $\vc{D}$ is an artifice which is not, on the whole, very helpful.
We have mentioned $\vc{D}$ because it is hallowed by tradition, beginning
% p. 333
with Maxwell,\footnote{The prominence of $\vc{D}$
in Maxwell's formulation of electromagnetic theory, and his
choice of the name displacement can perhaps be traced to his inclination toward a
kind of mechanical model of the ``aether.'' Whittaker has pointed out in his classic,
A History of the Theories of Aether and Electricity, Vol. I, p. 266 (Harper Torchbooks,
New York, 1960) that this inclination may have led Maxwell himself astray at one
point in the application of his theory of the problem of reflection of light 
from a dielectric.} and the student is sure to encounter it in other books,
many of which treat it with more respect than it deserves.

Our essential conclusions about electric fields in matter can be
summarized like this:

\begin{enumerate}[(i)]
\item Matter can be polarized, its condition being described 
completely, so far as the macroscopic field is concerned, by a
polarization density $\vc{P}$, which is the dipole moment per unit
volume. The contribution of such matter to the electric
field $\vc{E}$ is the same as that of a charge distribution $\rho_\text{bound}$,
existing in vacuum and having the density $\rho_\text{bound} =  - \div\vc{P}$.
In particular, at the surface of a polarized substance, where
there is a discontinuity in $\vc{P}$, this reduces to a surface charge
of density $\sigma =  -P_n$. Add any free charge distribution that
may be present, and the electric field is the field that this
\emph{total} charge distribution would produce in vacuum. That
is the macroscopic field $\vc{E}$ both inside and outside matter,
with the understanding that inside matter it is the spatial
average of the true microscopic field.

\item If $\vc{P}$ is proportional to $\vc{E}$ in a material, we call the material
a dielectric. We define the electric susceptibility $\chi_e$ and the
dielectric constant $\epsilon$ characteristic of that material:
$\chi_e=\vc{P}/\vc{E}$ and $\epsilon = 1+4\pi\chi_e$. Free charges immersed in
a dielectric give rise to electric fields which are $1/\epsilon$ times
as strong as the same charges would produce in vacuum.
\end{enumerate}

\section[Susceptibility and atomic polarizability]{The connection between electric susceptibility and atomic polarizability}

The ratio of polarization density $\vc{P}$ to macroscopic electric field $\vc{E}$
in a substance is the electric susceptibility $\chi_e$. Suppose the substance
is composed of atoms of atomic polarizability $\alpha$. Now $\vc{P}$ is nothing
but the sum, in unit volume, of the dipole moments $\vc{p}$ of the individual
atoms. We can predict the induced dipole moment of an atom if
we know $\alpha$ and the electric field acting on the atom to polarize it.
Hence we ought to be able, if we know $\alpha$ and the number of atoms
% p. 334
per unit volume, $N$, to predict the susceptibility $\chi_e$. Let us try to
make a theory connecting $\chi_e$ with $\alpha$.

The dipole moment induced in some atom A is determined by the
field, arising from all \emph{other} sources, that acts on that atom. This is
\emph{not} the same as the macroscopic electric field $\vc{E}$ in that neighborhood,
for that field $\vc{E}$ includes a contribution from the charges in the atom A
itself. So our problem takes an interesting twist right at the 
beginning. To make the situation clear, we consider a very specific system.
Our substance is made of identical atoms arrayed on a simple cubic
crystal lattice with spacing $b$ cm between nearest neighbors. The
polarizability of each atom is $\alpha$. Figure 9.31 is a cross section
through the lattice. The assumed direction of the macroscopic field
$\vc{E}$ in that region is indicated, as is the distortion of the polarized
atoms. Our question is, what is the magnitude of the field that is
causing that distortion? Each atom may be thought of as occupying
its own cubical box, and we shall assume that the atoms are a good
deal smaller than the lattice spacing, so that all the charge in an atom
is fairly near the center of its box.

We'll designate by $\vc{E}_\text{else}$ the field that acts on atom A. The sources
of this field are everything else in the system, including all the other
atoms and any other charges outside. $\vc{E}_\text{else}$ is the field that we would
find in box A if we could magically remove atom A while freezing all
other charge distributions in the shape they have with atom A
present. $\vc{E}_\text{else}$ will not be quite constant through box A, but we shall
suppose that its \emph{average value} within the box is close enough to what
we want. By an average over the volume enclosed by box A we
mean, as usual, the integral $\int\vc{E}\der v$ divided by the volume of the box.
We shall indicate such averages by $\langle\quad\rangle_\text{box}$.

Let $\vc{E}_\text{self}$ denote the field of the atom A. The total microscopic
field, $\vc{E}_\text{mic}$, is then at every point
\begin{equation}
  \vc{E}_\text{mic} = \vc{E}_\text{else} + \vc{E}_\text{self}
\end{equation}
We know that the macroscopic field $\vc{E}$ is equal to the space average
of the microscopic field $\vc{E}_\text{mic}$. The variation of $\vc{E}_\text{mic}$ is of course the
same in every box. Also, the boxes completely fill the space with
no gaps between them. Therefore the average of $\vc{E}_\text{mic}$ within any
\emph{one} box must be the same as its average over a larger region containing
many boxes.\footnote{Don't accept this crucial statement without thought. Why would gaps between
boxes spoil it?} It follows that
\begin{equation}
  \langle\vc{E}_\text{mic}\rangle_\text{box} = \vc{E}
\end{equation}
% p. 335
But $\langle\vc{E}_\text{mic}\rangle_\text{box}
=\langle\vc{E}_\text{else}\rangle_\text{box}+\langle\vc{E}_\text{self}\rangle_\text{box}$,
the average of a sum being
the sum of the averages, so that $\langle\vc{E}_\text{else}\rangle_\text{box}$, which is the quantity we're
after, is given by
\begin{equation}
  \langle\vc{E}_\text{else}\rangle_\text{box} = \vc{E} - \langle\vc{E}_\text{self}\rangle_\text{box}
\end{equation}

Our problem is now reduced to the calculation of $\langle\vc{E}_\text{self}\rangle_\text{box}$, the
average within the box of the field of the atom which occupies the
box.
\begin{equation}
  \langle\vc{E}_\text{self}\rangle_\text{box} = \frac{1}{b^3}\int_\text{box} \vc{E}_\text{self}\der v
\end{equation}
We must include in the integration all volume elements within the
box, inside as well as outside the atomic charge distribution. Figure
9.32 suggests what the field $\vc{E}_\text{self}$ might look like. It makes our task
seem formidable. However, we can always deal with a charge distribution
one element at a time. Let us calculate the box average
of the field $\vc{E}_q$ from a single point charge $q$.

If the point charge were at the center of the box, as in Fig. 9.33a,
the integral $\int_\text{box} \vc{E}_q\der v$ would be zero. Thanks to the symmetry,
every volume element in the box is matched by another in which the
field has equal strength in the opposite direction. Now shift the
charge $q$ upward by a small distance $z$, as in Fig. 9.33b. There is a
thin layer at the bottom of the box, of thickness $2z$, that is not balanced
% p. 336
by a layer at the top. It is this layer which now makes the only
contribution to $\int_\text{box} \vc{E}_q\der v$. Obviously, we only need to compute the
average of $E_{qz}$; $E_{qz}$ and $E_{qy}$ will still average to zero. If we neglect the
slight variation of $E_{qz}$ in the thickness of the layer, the volume integral
of $E_{qz}$ through the volume of the layer is just $2z$ times the 
surface integral of $E_{qz}$ over the square which forms the median plane
of the layer (Fig. 9.33c).
\begin{equation}
  \int_\text{layer} E_{qz}\der v = 2z \int_\text{square} E_{qz}\der a
\end{equation}
Gauss's law comes to our aid now, for the surface integral in Eq. 64
is just the flux of $\vc{E}_q$ through one side of a cube centered about the
charge $q$. That flux must be just $4\pi q/6$, because a cube has six 
equivalent faces. We conclude that $2z \int_\text{square} E_{qz}\der a=-2z(4\pi q/6)
=-4\pi qz/3$ so that
\begin{equation}
  \langle E_{qz}\rangle_\text{box} = \frac{1}{b^3} \int_\text{box} E_{qz}\der v = \frac{-4\pi qz}{3 b^3}
\end{equation}
The minus sign expresses the fact that upward displacement of a
positive charge results in a preponderance of downward field in the
box. A similar formula would apply to a displacement in the $x$ or
the $y$ direction. Therefore a small displacement $\vc{r}$ from the center in
any direction will result in the average field in the box being
$-4\pi q\vc{r}/3b^3$. Hence for our complete atom A, with its charge distribution
$\rho(x,y,z)$, the average field in the box will be
\begin{equation}
  \langle E_\text{self}\rangle_\text{box} = \frac{-4\pi}{3b^3} \int\vc{r}\rho\der v
\end{equation}
In the integral $\int\vc{r}\rho\der v$ we recognize the \emph{dipole moment} of the
charge distribution, $\vc{p}$. (Compare the definition of dipole moment,
Eq. 10.) We have now
\begin{equation}
  \langle E_\text{self}\rangle_\text{box} = \frac{-4\pi}{3b^3}\vc{p}
\end{equation}

The rest is clear sailing. From Eq. 62 we get at once:
\begin{equation}
  \langle E_\text{else}\rangle_\text{box} = \vc{E}-\frac{4\pi}{3b^3}\vc{p}
\end{equation}
If we say that $\langle E_\text{else}\rangle_\text{box}$ is the field effective in polarizing the atom,
then $\vc{p}$ is related to this field through the atomic polarizability:
\begin{equation}
  \vc{p} = \alpha \langle E_\text{else}\rangle_\text{box}
\end{equation}
% p. 337
From Eqs. 68 and 69 we get the relation connecting $\vc{p}$ and $\vc{E}$:
\begin{equation}
  \vc{p} = \alpha\left[\vc{E}+\frac{4\pi\vc{p}}{3b^3}\right]
\end{equation}
This can be written in terms of the macroscopic polarization density
$\vc{P}$. Since $N$, the number of polarized atoms per cubic centimeter, is
equal to $1/b^3$, $\vc{P} = N\vc{p} = \vc{p}/b^3$. Substituting in Eq. 70,
\begin{equation}
  \vc{P} = N\alpha\left[\vc{E}+\frac{4\pi\vc{P}}{3}\right]
\end{equation}
which, rearranged, is
\begin{equation}
  \vc{P} = \left[\frac{N\alpha}{1-(4\pi/3)N\alpha}\right]\vc{E}
\end{equation}
The factor in brackets must be the electric susceptibility $\chi_e$.

We made two approximations along the way to Eq. 72. We assumed
that all parts of the atomic charge distribution are near the
center of the box; Eqs. 65 and 66 are not accurate if $z$, or $\vc{r}$, is not
small compared to $b$. Also, we used the average of $\vc{E}_\text{else}$ in the box,
instead of the value of $\vc{E}_\text{else}$ at the center of the box, as the field that
polarizes the atom. For atoms packed as close together as they are
in most crystals, the first assumption is not very realistic, and the
question involved in the second assumption is irrelevant. We can't
expect Eq. 72 to be obeyed exactly by a real crystal.

If the atoms of a substance are very far apart, so that $4\pi N\alpha/3\ll 1$,
we can neglect that term in the denominator in Eq. 72, which leaves
us with
\begin{equation}
  \chi_e = N\alpha
\end{equation}
That is the result we would have obtained if we had forgotten about
the influence of the dipoles on one another. It applies quite well to
gases at normal density, for which the neglected term is of the order
of $10^{-3}$. In that limit the geometrical arrangement of the atoms
doesn't matter, only the number per cubic centimeter. Thus by
measuring accurately the dielectric constant of a low-pressure gas,
the atomic polarizability can be determined without any complications
from mutual influence of the dipoles. Measurements on a
denser form of the same substance\footnote{
It can be shown that the relation given in Eq. 72 should hold not only for a crystal
with cubic symmetry but also for a gas in which the atoms occupy random positions in
space. The best experimental confirmation of this comes from measurements of the
dielectric constants of gases at relatively high density and pressure 
(see Prob. 9.28).} can then be used to test a theoretical formula such as Eq. 72.

% p. 338 

The term  $-4\pi N\alpha/3$ in the denominator of Eq. 72 reflects the interaction
of the polarized atoms in the crystal. Evidently the interaction
is in the nature of a \emph{reinforcement}, leading to a larger polarization
than would exist without it. If we take the mathematical formula
seriously, it suggests a surprising possibility. What if $N\alpha$ should be
so large that $4\pi N\alpha/3$ would equal or exceed one? It appears that
$\chi_e$ would go to infinity. That might mean a polarization in zero
applied field! It sounds like nonsense but it isn't entirely. Some
crystals are known which do show a spontaneous electric 
polarization. However, something more than induced polarization is involved
in that case, so our theory isn't applicable. In fact, to make
$4\pi N\alpha/3$ approach one, the atoms have to be so close together that
our approximations are no good at all (see Prob. 9.30).

\section{Energy changes in polarization}

To charge a capacitor to a potential difference $V$, work amounting
to $\frac{1}{2}CV^2$ has to be done. That amount of energy can be recovered
by allowing the capacitor to discharge through an external circuit.
The energy has been stored in the charged capacitor. It was shown
in Chap. 2 that in any electrostatic system the stored energy can be
calculated by assigning $E^2/8\pi$ $\zu{erg}/\cmunit^3$ to the electric field. By way
of review, the field strength $E$ in the parallel-plate vacuum capacitor,
of plate area $A$ and separation $t$, is $V/ t$, so that 
$(E^2/8\pi) \times \text{volume} = V^2A/8\pi t = \frac{1}{2}CV^2$.

If the capacitor is filled with dielectric of dielectric constant c and
charged to the same potential difference $V$, the work done will be
greater by the factor $\epsilon$, because $C$ is that much greater. However,
$E$ is the same. Therefore the energy to be associated with unit
volume in the dielectric is not $E^2/8\pi$, but $\epsilon E^2/8\pi$. This can be generalized
to any electrostatic system. In place of Eq. 2.36 we have now:
\begin{equation}
  \text{Energy} = \frac{1}{8\pi}\int\epsilon E^2\der v
\end{equation}
How is the ``extra'' energy stored? Consider an isolated 
polarizable molecule, to which an electric field can be applied. In Fig. 9.34
the molecule is represented by two charges fastened to the ends of
an elastic spring. Its dipole moment $\vc{p}$ is a vector of magnitude $qs$.
The field $\vc{E}$ arises from some external source, such as the plates and
battery shown in the figure. Suppose that while the field $\vc{E}$ is present,
% p. 339
the charges move apart by an increment $\der s$. The dipole moment
thereby changes from $qs$ to $q(s + \der s)$. And there has been a motion
of charge in the direction of $\vc{E}$, equivalent to charge $+q$ moving distance
$\der s$. (It doesn't matter whether one end moves, or both.)
Hence work in amount $Eq\der s$ has been done on the molecule. The
ultimate supplier of this work is the source of the field --- in Fig. 9.34
the battery that maintains constant potential difference between the
plates. If $\der W$ stands for the work done on the molecule, then
\begin{equation}
  \der W = Eq\der s = \vc{E}\cdot\der\vc{p}
\end{equation}

The stored energy that corresponds to this can be found in two
places, in the elastic spring which has been stretched to a greater
length, and in the electric field of the molecular dipole itself, which
now has more total energy because the two charges are farther apart.
In the case of a real molecule we should not make such a distinction.
It is all energy belonging to the molecular structure, and if we were
to look into that dynamic structure we should find the energy as
electrostatic potential energy and kinetic energy of electron motion.
The point is simply this: The work that has been done on the molecule
to change its polarization, $\vc{E}\cdot\der\vc{p}$, has increased by just that much
the energy tied up in the molecule itself.

Let us see how much of the energy stored in the dielectric can be
accounted for in this way. With $N$ molecules per unit volume,
$\vc{P} = N\vc{p}$. When $\vc{P}$ changes by $\der\vc{P}$, then $\vc{E}\cdot\der\vc{p}$
is the increase in internal
energy of the molecules in $1\ \cmunit^3$. But since $\vc{P}=(\epsilon-1)\vc{E}/4\pi$,
\begin{equation}
  \vc{E}\cdot\der\vc{p} = \frac{1}{4\pi}(\epsilon-1)\vc{E}\cdot\der\vc{E}
                        = \frac{1}{8\pi}(\epsilon-1)\der(E^2)
\end{equation}
Therefore, of the $\epsilon E^2/8\pi$ ergs which appear to be stored in the 
dielectric, $(\epsilon-1)E^2/8\pi$ can be accounted for as increased internal energy
of the polarized molecules. The remainder, $E^2/8\pi$, is just the energy
stored in the vacuum field.

\section{Dielectrics made of polar molecules}

Molecules with permanent dipole moments, \emph{polar} molecules,\index{polar molecule}
respond to an electric field by trying to line up parallel to it. The
appropriate mechanical model is not two charges on the ends of a
spring, but two charges fastened to the ends of a stick (Fig. 9.35).
If the stick is not parallel to the field, there is a torque on it, of magnitude
% p. 340
$Eqs \sin \theta$. The work done in an angular displacement $\der\theta$ is
$(\text{torque} \times \text{angular displacement})$ or $Eqs \sin \theta\der\theta$. This can also
be written in terms of the vector dipole moment $\vc{p}$, which is a vector
of magnitude $qs$, and the change $\der\vc{p}$ which occurs through rotation
by $\der\theta$. From the diagram it is clear that the magnitude of $\der p$ is $p\der\theta$,
and its direction is such that $\vc{E}\cdot\der\vc{p}=E\der p\sin\theta$. Thus $\der W = \vc{E}\cdot\der\vc{p}$.
This agrees with Eq. 75, as it ought to.

If an isolated polar molecule occupies the position indicated in
Fig. 9.35 at the instant the electric field $\vc{E}$ is switched on, it will swing
into line with the field, but then it will continue swinging past equilibrium
and will oscillate like a pendulum, for it has no way to get
rid of its energy. However, a real molecule surrounded by other
molecules can exchange energy with its neighbors, providing a kind
of ``friction'' that damps the oscillation. It would seem that this
might result in all the polar molecules in a substance settling down
exactly parallel to any applied field, however weak. So it would at
the temperature absolute zero, assuming rotation were still possible.
At any temperature above absolute zero the random motion of
thermal agitation, which is the more intense the higher the tempera-
ture, works against orderly alignment. The applied field makes it
energetically favorable for a molecular dipole to point parallel to
the field, but continually jostled by its neighbors, the best it can do
is to spend slightly more time pointing in the right direction than in
the wrong direction. In water, for example, a field of 1000 volts/cm
results in a polarization equivalent to perfectly aligning about one
molecule in 3000. Even so, this is a much greater polarization than
a nonpolar substance would exhibit in the same field, and is the
reason for the extraordinarily high dielectric constant of water. The
bulk polarization in a polar dielectric is generally proportional to
the applied electric field strength and inversely proportional to the
absolute temperature.

\section{Polarization in changing fields}

So far we have considered only electrostatic fields in matter. We
need to look at the effects of electric fields that are varying in time.
like the field in a capacitor used in an alternating-current circuit.
The important question is, will the changes in polarization keep up
with the changes in the field? Will the ratio of $\vc{P}$ to $\vc{E}$, at any instant.
be the same as in a static electric field? For very slow changes we
should expect no difference but, as always, the criterion for slowness
depends on the particular physical process. It turns out that induced
% p. 341
polarization and the orientation of permanent dipoles are two
processes with quite different response times.

The induced polarization of atoms and molecules occurs by the
distortion of the electronic structure. Little mass is involved and
the structure is very stiff; its natural frequencies of vibration are extremely
high. To put it another way, the motions of the electrons
in atoms and molecules are characterized by periods of the order
of $10^{-16}$ seconds --- something like the period of a visible light wave.
To an atom, $10^{-14}$ seconds is a \emph{long} time. It has no trouble readjusting
its electronic structure in a time like that. Because of this,
strictly nonpolar substances behave practically the same from ``dc''
up to frequencies close to those of visible light. The polarization
keeps in step with the field and the susceptibility $\chi_e = P/E$ is independent
of frequency. What happens when the frequency of the
field alternation does approach some natural frequency of the electronic
structure is an interesting question which we must leave for
the next volume. (One consequence is the rainbow!)

The orientation of a polar molecule is a process quite different
from the mere distortion of the electron cloud. The whole molecular
% p. 342
framework has to rotate. On a microscopic scale, it is rather like
turning a peanut end for end in a bag of peanuts. The frictional
drag tends to make the rotation lag behind the torque and to reduce
the amplitude of the resulting polarization. Where on the time scale
this effect sets in, varies enormously from one polar substance to
another. In water, the ``response time'' for dipole reorientation is
something like $10^{-11}$ seconds. The dielectric constant remains
around 80 up to frequencies of the order of $10^{10}$ cycles per second.
Above $10^{11}$ Hz $\epsilon$ falls to a modest value typical of a nonpolar
liquid. The dipoles simply cannot follow so rapid an alternation of
the field. In other substances, especially solids, the characteristic
time can be much longer. In ice just below the freezing point the
response time for electric polarization is around $10^{-5}$ seconds.
Figure 9.36 shows some experimental curves of dielectric constant
versus frequency for water and ice.

You may question whether a polar molecule could actually turn
over inside a compact rigid substance like a crystal. It does happen
in many crystals, where thanks to the vibration of its neighbors, a
molecule may suddenly find itself with enough ``elbow room" to
flip over as a unit. But the question is a good one, because in some
solids, shifts of electric charge can occur which are not describable
as rotations of permanent molecular dipole moments. We shall return
to this question presently.

\section{The bound-charge current}

Wherever the polarization in matter changes with time there is
an electric current, a genuine motion of charge. Suppose there are
$N$ dipoles in a cubic centimeter of dielectric, and that in the time
interval $\der t$ each changes from $\vc{p}$ to $\vc{p} + \der\vc{p}$. Then the macroscopic
polarization density $\vc{P}$ changes from $\vc{P} = N\vc{p}$ to 
$\vc{P} + \der \vc{P}= N (\vc{p} + \der\vc{p})$.
Suppose the change $\der\vc{p}$ was effected by moving a charge $q$ through a
distance $\der\vc{s}$, in each atom:  $q \der\vc{s} = \der\vc{p}$.
Then during the time $\der t$ there
was actually a charge cloud of density $P = Nq$, moving with velocity
$\vc{v} = \der\vc{s}$/dt. That is a conduction current of a certain density $\vc{J}$ in
$(\zu{esu}/\sunit)/\cmunit^2$:
\begin{equation}
  \vc{J} = \rho\vc{v} = Nq\frac{\der\vc{s}}{\der t} = N\frac{\der\vc{p}}{\der t} = \frac{\der\vc{P}}{\der t}
\end{equation}
The connection between rate of change of polarization and current
density, $\vc{J} = \der\vc{P}/dt$, is independent of the details of the model. A
% p. 343
changing polarization \emph{is} a conduction current, not essentially different
from any other.

Naturally, such a current is a source of magnetic field. If there
are no other currents around, we should write Maxwell's second
equation, $\curl\vc{B} = (1/c) (\partial\vc{E}/\partial t + 4\pi\vc{J})$ as
\begin{equation}
  \curl\vc{B} = \frac{1}{c} \left(\frac{\partial\vc{E}}{\partial t} + 4\pi\frac{\partial\vc{P}}{\partial t}\right)
\end{equation}

The only difference between an ``ordinary'' conduction current
density and the current density $\partial\vc{P}/\partial t$ is that one involves \emph{free} charge
in motion, the other \emph{bound} charge in motion. There is one rather
obvious practical distinction --- you can't have a \emph{steady} bound charge
current, one that goes on forever unchanged. Usually we prefer to
keep account separately of the bound charge current and the free
charge current, retaining $\vc{J}$ as the symbol for the free charge current
density only. Then to include all the currents in Maxwell's equation
we have to write it this way:
\newcommand{\mycruftystrut}{\rule[-0.8\baselineskip]{0pt}{1.6\baselineskip}}
\begin{equation}
  \curl\vc{B} = \frac{1}{c} \left(
    \frac{\partial\vc{E}}{\partial t}\right.
    + \underbrace{4\pi\frac{\partial\vc{P}}{\partial t}}_{
        \begin{array}{c}\text{Bound charge}\\\text{current density}\end{array}
      }
    + \underbrace{\left.4\pi\vc{J}\mycruftystrut\right)}_{
        \begin{array}{c}\text{Free charge}\\\text{current density}\end{array}
      }
\end{equation}

In a dielectric medium, $\vc{E} + 4\pi\vc{P} = \epsilon\vc{E}$, allowing a shorter version
of Eq. 79:
\begin{equation}
  \curl\vc{B} = \frac{1}{c} \left(\epsilon\frac{\partial\vc{E}}{\partial t} + 4\pi\vc{J}\right)
\end{equation}

More generally, Eq. 79 can also be abbreviated by introducing the
vector $\vc{D}$, previously \emph{defined} as $\vc{E} + 4\pi\vc{P}$:
\begin{equation}
  \curl\vc{B} = \frac{1}{c} \left(\frac{\partial\vc{D}}{\partial t} + 4\pi\vc{J}\right)
\end{equation}
The term $\partial\vc{D}/\partial t$ is usually referred to as 
the displacement current.\index{displacement current}
Actually, that part of it which involves $\partial\vc{P}/\partial t$ represents, as we have
seen, an honest conduction current, real charges in motion. The
only part of the total current density that is not simply charge in
motion is the $\partial\vc{E}/\partial t$ part, the true vacuum displacement current
which we discussed at the end of Chap. 7. Incidentally, if we want
to express all components of the full current density in units corresponding
% p. 344
to those of $\vc{J}$, we should note that no $4\pi$ appears in the first
term, and fix that up by writing Eq. 79 as follows:
\begin{equation}
  \curl\vc{B} = \frac{4\pi}{c} 
      \underbrace{\left(\frac{1}{4\pi}\frac{\partial\vc{E}}{\partial t}\right.}_{
        \begin{array}{c}\text{Vacuum}\\\text{displacement}\\\text{current}\\\text{density}\end{array}
      }
    + \underbrace{\frac{\partial\vc{P}}{\partial t}}_{
        \begin{array}{c}\text{Bound}\\\text{charge}\\\text{current}\\\text{density}\end{array}
      }
    + \underbrace{\left.\vc{J}\mycruftystrut\right)}_{
        \begin{array}{c}\text{Free}\\\text{charge}\\\text{current}\\\text{density}\end{array}
      }
\end{equation}

Involved in the distinction between bound charge and free charge
is a question we haven't squarely faced: can one always identify
unambiguously the ``molecular dipole moments'' in matter, especially
solid matter? The answer is no. Let us take a microscopic view of
a thin wafer of sodium chloride crystal. The arrangement of the
positive sodium ions and the negative chlorine ions was shown in
Fig. 1.7. Figure 9.37 is a cross section through the crystal, which
extends on out to the right and the left. If we choose to, we may
consider an adjacent pair of ions as a neutral molecule with a dipole
moment. Grouping them as in Fig. 9.37a, we describe the medium
as having a uniform macroscopic polarization density $\vc{P}$, a vector
directed downward. At the same time, we observe that there is a
layer of positive charge over the top of the crystal, and negative
charge over the bottom which, not having been included in our 
molecules, must be accounted \emph{free charge}.

Now we might just as well have chosen to group the ions as in
Fig. 9.37b. According to that description, $\vc{P}$ is a vector \emph{upward}, but
we have a negative free charge layer on top of the crystal and a positive
free charge layer beneath. \emph{Either description is correct}. You will
have no trouble finding another one, also correct, in which $\vc{P}$ is zero
and there is no free charge. Each description predicts $\vc{E} = 0$. The
macroscopic field $\vc{E}$ is an observable physical quantity. It can depend
only on the charge distribution, not on how we choose to \emph{describe}
the charge distribution.

This example teaches us that in the real atomic world the distinction
between ``bound charge'' and ``free charge'' is more or less
arbitrary, and so, therefore, is the concept of polarization density $\vc{P}$.
The molecular dipole is a well-defined notion only where molecules
as such are identifiable --- where there is some physical reason for
saying, ``This atom belongs to this molecule and not to that.'' In
many crystals such an assignment is meaningless. An atom or ion
may interact about equally strongly with all its neighbors; one can
only speak of the whole crystal as a single molecule.

% p. 346

Any arbitrariness in the distinction between free and bound charge
persists, naturally, in the distinction between free charge current
density $\vc{J}$, and $\partial\vc{P}/\partial t$.
Consider the polarization of a crystal such as
ice. The lattice is three-dimensional, but we have drawn in Fig. 9.38
a two-dimensional array with somewhat similar features. Let's call
it ice. In Fig. 9.38a we readily identify, the $\zu{H}_2\zu{O}$ molecules, for we
notice that each oxygen atom has just two H atoms close to it. The
crystal as shown is polarized. $\vc{P}$ points downward because, as mentioned
earlier in this chapter, the oxygen end of the water molecule
has an excess of negative charge. We can think of the black hydrogen
parts in the diagram as positive charge. Suppose now that something
happens to change the internal condition of this crystal to that
shown in Fig. 9.38d, a microscopic view of the same neighborhood.
Now the dipoles are reversed, and we describe the crystal as having
a polarization \emph{upward}.

The change \emph{could} have been brought about in two essentially different
ways, illustrated in Fig. 9.38b and c. In Fig. 9.38b an 
upward-directed electric field $\vc{E}$ has been applied, pushing upward the positive
ends of the molecules, in effect turning each molecule over.
There is a net \emph{upward} motion of positive charge; the current that it
represents will be accounted for in the term $\partial\vc{P}/\partial t$ as we have just
learned.

Figure 9.38c depicts a quite different process, in which the application
of a downward electric field encourages the hydrogens to switch
partners. Each is migrating to the nearest O atom below it. (This
is all the easier, in the real crystal, because the H that lies between
two O atoms is to some degree shared by both, providing 
the ``hydrogen bond'' that holds the crystal together.) The final configuration
looks exactly the same. The dipoles are all reVersed --- but what occurred
was a \emph{downward} flow of positive charge. If we are reckoning
up the current in this process, for entry on the right side of Eq. 79,
we have to put in the same term as before, $\partial\vc{P}/\partial t$, which corresponds
to an upward current, but we must add a larger downward conduction
current $\vc{J}$, corresponding to the motion of each charge downward
by \emph{one whole lattice space} $d$. The difference will be the true net 
current arising from the actual downward displacement of the positive
charge through a distance $s$.

Notice that in each case the total current flows in the direction of
the applied electric field. From macroscopic measurements alone
we could not tell which microscopic process is taking place. Indeed,
people are still arguing about the mechanism of polarization in ice.
% p. 347
To settle the argument, one needs to know enough about the microscopic
structure to be sure which is the easier, turning a molecule
upside down, or transferring protons. For us, the lesson to be drawn
is simply this: The actual microscopic motion of \emph{all} the charges determines
the total conduction current, free \emph{and} bound.
