\chapter{Electrostatics: charges and fields}

\section{Electric charge}

Electricity appeared to its early investigators as an extraordinary phenomenon.
To draw from bodies the \emph{subtle fire}, as it was
sometimes called, to bring an object into a highly electrified state, to
produce a steady flow of current, called for skillful contrivance. Except for
the spectacle of lightning, the ordinary manifestations of nature, from the
freezing of water to the growth of a tree, seemed to have no relation to the
curious behavior of electrified objects. We know now that electrical forces
largely determine the physical and chemical properties of matter over the whole
range from atom to living cell. For this understanding we have to thank the
scientists of the nineteenth century, Ampere, Faraday, Maxwell, and many
others, who discovered the nature of electromagnetism,, as well as the
physicists and chemist of the twentieth century who unraveled the atomic
structure of matter. 

Classical electromagnetism deals with electric charges and
currents and their interactions as if all the quantities involved could be
measured independently, with unlimited precision. Here \emph{classical} means
simply ``non-quantum.'' The quantum law with its constant $h$ is ignored in the
classical theory of electromagnetism, just as it is in ordinary mechanics. Indeed,
the classical theory was brought very nearly to its present state of completion
before Planck's discovery. It has survived remarkably well. Neither the
revolution of quantum physics nor the development of special relativity dimmed
the luster of the electromagnetic field equations Maxwell wrote down a hundred
years ago. 

Of course, the theory was solidly based on experiment, and
because of that was fairly secure within its original range of application --to
coils, capacitors, oscillating currents, eventually radio waves and light
waves. But even so great a success does not guarantee validity in another
domain, for instance, the inside of a molecule.

Two facts help explain the continuing importance in modern
physics of the classical description of electromagnetism. First, special
relativity required no revision of classical electromagnetism. Historically
speaking, special relativity \emph{grew out of} classical electromagnetic
theory and experiments inspired by it. Maxwell's field equations, developed
long before the work of Lorentz and Einstein, proved to be entirely compatible
with relativity. Second, quantum modifications of the electromagnetic force
have turned out to be unimportant down to distances less than $10^{-10}\ \textup{cm}$, a
hundred times smaller than the atom. We can describe the repulsion and
attraction of particles in the atom using the same laws that apply to the
leaves of an electroscope, although we need quantum mechanics to predict how
the particles will behave under those forces. For still smaller distances,
there is a rather successful fusion of electromagnetic theory and quantum
theory, called \emph{quantum electrodynamics}, which seems to agree with
experiment down to the smallest distances yet explored. 

We assume the reader has some acquaintance with elementary
facts of electricity. We are not going to review all the experiments by which
the existence of electric was demonstrated or all the evidence for the
electrical constitution of matter. On the other hand, we do want to look
carefully at the experimental foundations of the basic laws on which all else
depends. In this chapter we shall study the physics of stationary electric
charges --- \emph{electrostatics}.

Certainly one fundamental property of electric charge is its
existence I the two varieties that were long ago named positive and negative.
The observed fact is that all charged particles can be divided into two classes
such that all members of one class repel each other, while attracting members
of the other class. If two small electrically charged bodies $A$ and
$B$, some distance apart, repel one another, and if $A$ attracts some
third electrified body $C$, then we always find that $B$ attracts
$C$. Why this universal law prevails we cannot say for sure. But today
physicists tend to regard positive and negative charge as, fundamentally,
opposite manifestations of one quality, much as ``right'' and ``left'' are
opposite manifestations of ``handedness.'' Indeed, the question of symmetry
involved in right and left seems to be intimately related to this duality of
electric charge, and to another fundamental symmetry, the two directions of
time. Elementary particle physics is throwing some light on these questions.

What we call negative charge could just as well have been
called positive, and vice versa.\footnote{The charge of the ordinary electron has nothing
\emph{intrinsically negative} about it. A negative integer, once multiplication has been
defined, differs essentially from a positive integer in that its square is an integer of
opposite sign. But the product of two charges is not a charge; there is no comparison.}
The choice was a historical accident. Our
universe appears to be very evenly balanced mixture of positive and negative
electric charge which, since like charges repel one another is not surprising. 

Two other observed properties of electric charge are
essential in the electrical structure of matter: charge is conserved, and
charge is quantized. These properties involve \emph{quantity} of charge, and
thus imply a measurement of charge. Presently we shall state precisely how
charge can be measured in terms of the force between charges a certain distance
apart, and so on. But let us take this for granted, for the time being, so that
we may talk freely about these fundamental facts. 

\section{Conservation of charge}

The total charge in an isolated system never changes. By
\emph{isolated} we mean that no matter is allowed to cross the boundary of the
system. We could let light pass into or out of the system without affecting the
principle, since photons carry no charge. For instance, a thin-walled box in
vacuum, exposed to gamma rays, might become the scene of a ``pair-creation''
event in which a high-energy photon ends its existence with the creation of a
negative electron and a positive electron (Fig. 1.1). Two electrically charged
particles have been newly created but the net change in total charge, in and on
the box, is zero. An event that \emph{would} violate the law we have just
stated would be the creation of a positively charged particle \emph{without}
the simultaneous creation of a negatively charged particle. Such an occurrence
has never been observed. 

Of course, if the electric charges of electron and positron
were not precisely equal in magnitude, pair creation would still violate the
strict law of charge conservation. As well as can be determined from
experiment, their charges \emph{are} equal. An interesting experimental test is
provided by the structure called \emph{positronium}, a structure composed of an
electron and a positron, and nothing else. This curious ``atom'' can live long
enough-a tenth of a microsecond or so-to be studied in detail. It behaves as if
it were quite neutral, electrically. Actually, most physicists would be
astonished, not to say incredulous, if \emph{any} difference were found in the
magnitudes of these charges, for we know that electron and positron are related
to one another as \emph{particle} to \emph{antiparticle}. Their exact equality
of charge, like their equality of mass, is a manifestation of an apparently
universal symmetry in nature, the particle-antiparticle duality. One might
wonder whether charge conservation, then, is merely a corollary of some broader
conservation law governing the creation and annihilation of particles; or is
charge conservation a primary requirement, with which other laws have to fall
in line? Or do these questions make sense? We do not know for sure. 

One thing will become clear in the course of our study of
electromagnetism: nonconservation of charge would be quite incompatible with
the structure of our present electromagnetic theory. We may therefore state
either as a postulate of the theory or as an empirical law supported without
exception by all observations so far, the \emph{charge conservation law:}

The total electric charge in an isolated system, that is, the
algebraic sum of the positive and negative charge present at any time, never
changes. 

Sooner or later we must ask whether this law meets the test
of relativistic invariance. We shall postpone until Chap. 5 a thorough
discussion of this important question. But the answer is that it does, and not
merely in the sense that the statement above holds in any given inertial frame,
but in the stronger sense that observer in different frames, measuring the
charge, get the same number. In other words the total electric charge of an
isolated system is relativistically invariant number. 

\section{Quantization of charge} 

Milikan's oil-drop experiment, and innumerable other
experiments, have shown that in nature electric charge comes in units of one
magnitude only. That magnitude we denote by $e$, the electronic charge. We have
already noted that the positron has precisely this amount of charge. What seems
more remarkable is the exact equality in the charges carried by all other
charged particles-the equality, for instance, in the magnitude of the positive
charge on the proton and the negative charge on the electron. 

That particular equality, the proton-electron charge
balance, is open to a very sensitive test. One can test the normal hydrogen
atom or molecule for overall electrical neutrality. Thus, one could try to
deflect a beam of atoms or molecules by an electrical field. In a sensitive
experiment devised for this purpose,\footnote{J.C.~Zorn, G.E.~Chamberlin,
and V.W.~Hughes, Phys. Rev. 129, 2566 (1963)} a sharply defined beam of cesium atoms was
sent, in a high vacuum, through a strong electric field. From the absence of
any observable deflection it could be concluded that the net charge on a cesium
atom must be less than $10^{-16}e$. An even more sensitive test has recently
been made by a different method.\footnote{J.G.~King, Phys. Rev. Letters 5, 562 (1960).
References to previous tests of charge equality will be found in this article and
in the chapter by V.W.~Hughes in Gravitation and Relativity, edited by H.Y.~Chiu and
W.F.~Hoffman (W.A.~Benjamin, Inc., New York, 1964), chap. 13.} A large amount of hydrogen gas was compressed
into a tank which was itself highly insulated, electrically, from its
surroundings. The gas was then allowed to escape from the tank by means which
prevented the escape of any ordinary ions. \emph{If} the charge on the proton
differed from that on the electron by, say, one part in a billion, then each
hydrogen molecule, composed of two protons and two electrons, would carry a
charge of $2\times 10^{-9}e$, and the departure of the whole mass of hydrogen would
measurably alter the electrical charge and potential of the tank. In fact, the
experiment could have revealed a residual charge as small as $10^{-20}e$ per
atom, and none was observed! We conclude that the electron and proton have
equal charge, to an accuracy of 1 part in $10^{20}$. 

On present ideas, the electron and the proton are about as
unlike as two elementary particles can be. No one yet understands why their
charges should have to be equal to such a fantastically precise degree.
Evidently the quantization of charge is a deep and universal law of nature.
\emph{All} charged elementary particles, as far as we can determine, carry
charges of precisely the same magnitude. We can only hope that some future
discovery or theoretical insight may reveal to us why a particle with a charge
$0.500e$, or $0.999e$, cannot exist.\footnote{In some recent theoretical speculations
about elementary particles the possible existence of charge $\frac{1}{3}e$ and $\frac{2}{3}e$
was suggested. In a subsequent search for such particles, under conditions believed
favorable for their production and detection, none turned up. [L.B.~Leipuner, W.T.~Chu,
R.C.~Larsen, R.K.~Adair, Particles with a charge of $\frac{1}{3}e$, Phys. Rev. Letters 12, 423 (1964)].
As this is written, however, the speculation continues.}

The fact of charge quantization lies outside the scope of
classical electromagnetism, of course. We shall usually ignore it, and act as
if our point charges $q$ could have any strength whatever. This will not get us
into trouble. Still, it is worth remembering that classical theory cannot be
expected to explain the structure of the elementary particles. (It is not
certain that present quantum theory can either!) What holds the electron
together is as mysterious as what fixes the precise value of its charge.
Something more than electrical forces must be involved, for the electrostatic
forces between different parts of the electron would be repulsive. 

In our study of electricity and magnetism we shall treat the
charged particles simply as carriers of charge, with dimensions so small
that

%%% ... partial sentence
