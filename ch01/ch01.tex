\chapter{Electrostatics: charges and fields}

\section{Electric charge}\index{charge}

Electricity appeared to its early investigators as an extraordinary phenomenon.
To draw from bodies the \emph{subtle fire}, as it was
sometimes called, to bring an object into a highly electrified state, to
produce a steady flow of current, called for skillful contrivance. Except for
the spectacle of lightning, the ordinary manifestations of nature, from the
freezing of water to the growth of a tree, seemed to have no relation to the
curious behavior of electrified objects. We know now that electrical forces
largely determine the physical and chemical properties of matter over the whole
range from atom to living cell. For this understanding we have to thank the
scientists of the nineteenth century, Ampere, Faraday, Maxwell, and many
others, who discovered the nature of electromagnetism,, as well as the
physicists and chemist of the twentieth century who unraveled the atomic
structure of matter. 

Classical electromagnetism deals with electric charges and
currents and their interactions as if all the quantities involved could be
measured independently, with unlimited precision. Here \emph{classical} means
simply ``non-quantum.'' The quantum law with its constant $h$ is ignored in the
classical theory of electromagnetism, just as it is in ordinary mechanics. Indeed,
the classical theory was brought very nearly to its present state of completion
before Planck's discovery. It has survived remarkably well. Neither the
revolution of quantum physics nor the development of special relativity dimmed
the luster of the electromagnetic field equations Maxwell wrote down a hundred
years ago. 

Of course, the theory was solidly based on experiment, and
because of that was fairly secure within its original range of application --to
coils, capacitors, oscillating currents, eventually radio waves and light
waves. But even so great a success does not guarantee validity in another
domain, for instance, the inside of a molecule.

Two facts help explain the continuing importance in modern
physics of the classical description of electromagnetism. First, special
relativity required no revision of classical electromagnetism. Historically
speaking, special relativity \emph{grew out of} classical electromagnetic
theory and experiments inspired by it. Maxwell's field equations, developed
long before the work of Lorentz and Einstein, proved to be entirely compatible
with relativity. Second, quantum modifications of the electromagnetic force
have turned out to be unimportant down to distances less than $10^{-10}\ \textup{cm}$, a
hundred times smaller than the atom. We can describe the repulsion and
attraction of particles in the atom using the same laws that apply to the
leaves of an electroscope, although we need quantum mechanics to predict how
the particles will behave under those forces. For still smaller distances,
there is a rather successful fusion of electromagnetic theory and quantum
theory, called \emph{quantum electrodynamics}, which seems to agree with
experiment down to the smallest distances yet explored. 

We assume the reader has some acquaintance with elementary
facts of electricity. We are not going to review all the experiments by which
the existence of electric was demonstrated or all the evidence for the
electrical constitution of matter. On the other hand, we do want to look
carefully at the experimental foundations of the basic laws on which all else
depends. In this chapter we shall study the physics of stationary electric
charges --- \emph{electrostatics}.

Certainly one fundamental property of electric charge is its
existence I the two varieties that were long ago named positive and negative.
The observed fact is that all charged particles can be divided into two classes
such that all members of one class repel each other, while attracting members
of the other class. If two small electrically charged bodies $A$ and
$B$, some distance apart, repel one another, and if $A$ attracts some
third electrified body $C$, then we always find that $B$ attracts
$C$. Why this universal law prevails we cannot say for sure. But today
physicists tend to regard positive and negative charge as, fundamentally,
opposite manifestations of one quality, much as ``right'' and ``left'' are
opposite manifestations of ``handedness.'' Indeed, the question of symmetry
involved in right and left seems to be intimately related to this duality of
electric charge, and to another fundamental symmetry, the two directions of
time. Elementary particle physics is throwing some light on these questions.

What we call negative charge could just as well have been
called positive, and vice versa.\footnote{The charge of the ordinary electron has nothing
\emph{intrinsically negative} about it. A negative integer, once multiplication has been
defined, differs essentially from a positive integer in that its square is an integer of
opposite sign. But the product of two charges is not a charge; there is no comparison.}
The choice was a historical accident. Our
universe appears to be very evenly balanced mixture of positive and negative
electric charge which, since like charges repel one another is not surprising. 

Two other observed properties of electric charge are
essential in the electrical structure of matter: charge is conserved, and
charge is quantized. These properties involve \emph{quantity} of charge, and
thus imply a measurement of charge. Presently we shall state precisely how
charge can be measured in terms of the force between charges a certain distance
apart, and so on. But let us take this for granted, for the time being, so that
we may talk freely about these fundamental facts. 

\section{Conservation of charge}\index{charge!conservation of}\index{conservation!of charge}

The total charge in an isolated system never changes. By
\emph{isolated} we mean that no matter is allowed to cross the boundary of the
system. We could let light pass into or out of the system without affecting the
principle, since photons carry no charge. For instance, a thin-walled box in
vacuum, exposed to gamma rays, might become the scene of a ``pair-creation''
event in which a high-energy photon ends its existence with the creation of a
negative electron and a positive electron (Fig. 1.1). Two electrically charged
particles have been newly created but the net change in total charge, in and on
the box, is zero. An event that \emph{would} violate the law we have just
stated would be the creation of a positively charged particle \emph{without}
the simultaneous creation of a negatively charged particle. Such an occurrence
has never been observed. 

Of course, if the electric charges of electron and positron
were not precisely equal in magnitude, pair creation would still violate the
strict law of charge conservation. As well as can be determined from
experiment, their charges \emph{are} equal. An interesting experimental test is
provided by the structure called \emph{positronium}, a structure composed of an
electron and a positron, and nothing else. This curious ``atom'' can live long
enough-a tenth of a microsecond or so-to be studied in detail. It behaves as if
it were quite neutral, electrically. Actually, most physicists would be
astonished, not to say incredulous, if \emph{any} difference were found in the
magnitudes of these charges, for we know that electron and positron are related
to one another as \emph{particle} to \emph{antiparticle}. Their exact equality
of charge, like their equality of mass, is a manifestation of an apparently
universal symmetry in nature, the particle-antiparticle duality. One might
wonder whether charge conservation, then, is merely a corollary of some broader
conservation law governing the creation and annihilation of particles; or is
charge conservation a primary requirement, with which other laws have to fall
in line? Or do these questions make sense? We do not know for sure. 

One thing will become clear in the course of our study of
electromagnetism: nonconservation of charge would be quite incompatible with
the structure of our present electromagnetic theory. We may therefore state
either as a postulate of the theory or as an empirical law supported without
exception by all observations so far, the \emph{charge conservation law:}

The total electric charge in an isolated system, that is, the
algebraic sum of the positive and negative charge present at any time, never
changes. 

Sooner or later we must ask whether this law meets the test
of relativistic invariance. We shall postpone until Chap. 5 a thorough
discussion of this important question. But the answer is that it does, and not
merely in the sense that the statement above holds in any given inertial frame,
but in the stronger sense that observer in different frames, measuring the
charge, get the same number. In other words the total electric charge of an
isolated system is relativistically invariant number. 

\section{Quantization of charge}\index{charge!quantization of}\index{quantization of charge}

Millikan's oil-drop experiment,\index{Millikan oil drop experiment} and innumerable other
experiments, have shown that in nature electric charge comes in units of one
magnitude only. That magnitude we denote by $e$, the electronic charge. We have
already noted that the positron has precisely this amount of charge. What seems
more remarkable is the exact equality in the charges carried by all other
charged particles-the equality, for instance, in the magnitude of the positive
charge on the proton and the negative charge on the electron. 

That particular equality, the proton-electron charge
balance, is open to a very sensitive test. One can test the normal hydrogen
atom or molecule for overall electrical neutrality. Thus, one could try to
deflect a beam of atoms or molecules by an electrical field. In a sensitive
experiment devised for this purpose,\footnote{J.C.~Zorn, G.E.~Chamberlin,
and V.W.~Hughes, Phys. Rev. 129, 2566 (1963)} a sharply defined beam of cesium atoms was
sent, in a high vacuum, through a strong electric field. From the absence of
any observable deflection it could be concluded that the net charge on a cesium
atom must be less than $10^{-16}e$. An even more sensitive test has recently
been made by a different method.\footnote{J.G.~King, Phys. Rev. Letters 5, 562 (1960).
References to previous tests of charge equality will be found in this article and
in the chapter by V.W.~Hughes in Gravitation and Relativity, edited by H.Y.~Chiu and
W.F.~Hoffman (W.A.~Benjamin, Inc., New York, 1964), chap. 13.} A large amount of hydrogen gas was compressed
into a tank which was itself highly insulated, electrically, from its
surroundings. The gas was then allowed to escape from the tank by means which
prevented the escape of any ordinary ions. \emph{If} the charge on the proton
differed from that on the electron by, say, one part in a billion, then each
hydrogen molecule, composed of two protons and two electrons, would carry a
charge of $2\times 10^{-9}e$, and the departure of the whole mass of hydrogen would
measurably alter the electrical charge and potential of the tank. In fact, the
experiment could have revealed a residual charge as small as $10^{-20}e$ per
atom, and none was observed! We conclude that the electron and proton have
equal charge, to an accuracy of 1 part in $10^{20}$. 

On present ideas, the electron and the proton are about as
unlike as two elementary particles can be. No one yet understands why their
charges should have to be equal to such a fantastically precise degree.
Evidently the quantization of charge is a deep and universal law of nature.
\emph{All} charged elementary particles, as far as we can determine, carry
charges of precisely the same magnitude. We can only hope that some future
discovery or theoretical insight may reveal to us why a particle with a charge
$0.500e$, or $0.999e$, cannot exist.\footnote{In some recent theoretical speculations
about elementary particles the possible existence of charge $\frac{1}{3}e$ and $\frac{2}{3}e$
was suggested. In a subsequent search for such particles, under conditions believed
favorable for their production and detection, none turned up. [L.B.~Leipuner, W.T.~Chu,
R.C.~Larsen, R.K.~Adair, Particles with a charge of $\frac{1}{3}e$, Phys. Rev. Letters 12, 423 (1964)].
As this is written, however, the speculation continues.}

The fact of charge quantization lies outside the scope of
classical electromagnetism, of course. We shall usually ignore it, and act as
if our point charges $q$ could have any strength whatever. This will not get us
into trouble. Still, it is worth remembering that classical theory cannot be
expected to explain the structure of the elementary particles. (It is not
certain that present quantum theory can either!) What holds the electron
together is as mysterious as what fixes the precise value of its charge.
Something more than electrical forces must be involved, for the electrostatic
forces between different parts of the electron would be repulsive. 

In our study of electricity and magnetism we shall treat the
charged particles simply as carriers of charge, with dimensions so small
that

%%% ... partial sentence

\section{Flux}\index{flux}

The relation between the electric field and its sources can be
expressed in a remarkable simple way, one that we shall find very
useful. For this we need to define a quantity called \emph{flux}.

Consider some electric field in space and in this space some
arbitrary closed surface, like a balloon of any shape. Figure 1.13
shows such a surface, the field being suggested by a few field lines.
Now divide the whole surface into little patches which are so small
that over any on patch the surface is practically flat and the vector
field does not change appreciably from one part of a patch to
another. In other words, don't let the balloon be too crinkly, and
don't let its surface pass right through a singularity \footnote{By a
singularity of the field we would ordinarily mean not only a point
source where the field approaches infinity, but any place where the
field changes magnitude or direction discontinuously, such as an
infinitesimally thin layer of concentrated charge.
Actually this latter, milder, kind of singularity would cause no
difficulty here unless out balloon's surface were to coincide with
the surface of discontinuity over some finite area.} of the field
such as a point charge. The area of a patch defines a unique
direction ---the outward-pointing normal to its surface. (Since the
surface is closed you can tell its inside from its outside; there is
no ambiguity.) Let this magnitude and direction be represented by a
vector. Then for every patch into which the surface has been divided,
such as patch number $j$, we have a vector $\vc{a}_j$, giving its area
and orientation. The steps we have just taken are pictured in Fig.
1.13b and c. Note the vector $\vc{a}_j$ does not depend at all on
the shape of the patch; it doesn't matter how we have divided up the
surface, as long as the patches are small enough.

Let $\vc{E}_j$ be the electric field vector at the location of patch
number $j$. The scalar product $\vc{E}_j \cdot a_j$ is a number.
We call this number the \emph{flux} through that bit of surface.
To understand the origin of the same name, imagine a vector function
which represents the velocity of motion in a fluid ---say in a river,
where the velocity varies from one place to another but is constant
in time at any one position. Denote this vector field by $v$,
measured, say, in feet per second. Then if $a$ is the oriented area,
in square feed, of a frame lowered into the water, $\vc{v} \cdot \vc{a}$
is the \emph{rate of flow} of water through the fame in cubic feet
per second (Fig. 1.14). We must emphasize that our definition of
flux is applicable to any vector function, whatever physical variable
it may represent.

Now let us add up all the flux through all the patches to get the
flux through the entire surface, a scalar quantity which we shall
denote by $\Phi$:
\begin{equation} 
  \Phi= \sum\limits_{\text{All}~j}\vc{E}_j \cdot \vc{a}_j 
\end{equation}
Letting the paths become smaller and more numerous without limit, we
pass from the sum in Eq. 16 to a surface integral:
\begin{equation}
  \Phi= \int\nolimits_{\substack{\text{Entire}\\\text{surface}}}\vc{E} \cdot d\vc{a}
\end{equation}
A surface integral of any vector function $\vc{F}$, over a surface
$S$, means just this: Divide $S$ into small patches, each represented
by a vector outward, of magnitude equal to the patch area; at every
patch, take the scalar product of the patch area vector and the local
$\vc{F}$; sum all these products, and the limit of this sum, as the
patches shrink, is the surface integral, Do not be alarmed by the
prospect of having to perform such a calculation for an awkwardly
shaped surface like the one in Fig. 1.13. The surprising property
we are about to demonstrate makes that unnecessary!

\section{Gauss's law}\index{Gauss's law}

Take the simplest case imaginable; suppose the field is that of a
single isolated positive point charge $q$ and the surface is a sphere
of radius $r$ centered on the point charge (Fig. 1.15. What is the
flux $\Phi$ through this surface? The answer is easy because the
magnitude of $\vc{E}$ at every point on the surface is $q/r^2$ and its
direction is the same as that of the outward normal at that point.
So we have:
\begin{equation} 
  \Phi =  {\vc{E}}\times \text{total area} = \frac{q}{r^2} \times 4\pi r^2 =  4\pi q 
\end{equation}
The flux is independent of the size of the sphere.

% p. 24

Now imagine a second surface, or balloon, enclosing the first, but
\emph{not} spherical, as in Fig 1.16. We claim that the total flux
through this surface is the same as that through the sphere.
To see this, look at a cone, radiating from $q$, which cuts a small
patch $\vc{a}$ out of the sphere and continues on to the outer surface
where it cuts out a patch $\vc{A}$ at a distance $R$ from the point
charge. The area of the patch $\vc{A}$ is larger than that of the patch
$\vc{a}$ by two factors: first, by the ration of the distance squared
$(R/r)^2$; and second, owing to its inclination, by the factor
$(1/\cos\theta)$. The angle $\theta$ is the angle between the outward
normal and the radial direction (see Fig. 1.16). The electric field
in that neighborhood is reduced from its magnitude on the sphere by
the factor $(r/R)^2$ and is still radially directed.
Letting $\vc{E}_{(r)}$ be the field at the sphere,
we have:

\begin{gather*}
\text{Flux through outer patch} =  \vc{E}_{(R)} \cdot \vc{A} =   E_ {(R)}  A\cos\theta \\
\text{Flux through inner patch} =  \vc{E}_ {(r)} \cdot \vc{a} =   E_ {(r)}  a \\
E_r A\cos\theta =  \left[E_r \left(\frac{r}{(R)} \right)^2  \right]
                   \left[a\left(\frac{R}{r} \right)^2  \frac{1}{\cos\theta}\right]
                   \cos\theta =  E_r a
\end{gather*}

\noindent This proves the flux through the two patches is the same.

Now every patch on the outer surface can in this way be put into
correspondence with part of the spherical surface, so the total flux
must be the same through the two surfaces. That is, the flux through
the new surface must be just $4 \pi q$. But this was a surface of
\emph{arbitrary} shape and size. \footnote{To be sure, we had the
second surface enclosing the sphere, but it didn't have to, really.
Besides, the sphere can betaken as small as we please.} We conclude:
the flux of the electric field through \emph{any} surface enclosing a
point charge $q$ is $4 \pi q$. As a corollary we can say that the
goal flux through a closed surface is \emph{outside} the surface. We
leave the proof of this to the reader, along with Fig. 1.17 as a
hint of one possible line of argument.

There is a way of looking at all this which makes the result seem
obvious. Imagine at $q$ a source which emits particles ---such as
bullets or photons ---in all directions at a steady rate. Clearly the
flux of particles through a window of unit area will fall off with
the inverse square of the window's distance from $q$. Hence we can
draw an analogy between the electric field strength $E$ and the
intensity of particle flow in bullets per unit area per unit time. It
is pretty obvious that the flux of bullets through any surface
completely surrounding $q$ is independent of the size and shape of
that surface, for it is just the total number emitted per unit time.
Correspondingly, the flux of $E$ through the closed surface must be
independent of size 
% p. 25
and shape. The common feature
responsible for this is the inverse-square behavior of the intensity.

The situation is now ripe for superposition! Any electric field is
the sum of the fields of its individual sources. This property was
expressed in our statement, Eq. 13, of Coulomb's law. Clearly flux
is an additive quantity in the same sense, for if we have a number of
sources $q_1$, $q_2$, \ldots $q_N$, the fields of which, if each were
present alone, would be $\vc{E}_1$, $\vc{E}_2$, \ldots $\vc{E}_N$, the
flux $\Phi$ through some surface $S$ in the actual field can be written

\begin{equation}
  \Phi =  \int\nolimits_{S}{} E \cdot d
  \vc{a} =  \int\nolimits_{{S}}{}[\vc{E}_1 +
  \vc{E}_2 + \ldots + \vc{E}_N] \cdot d \vc{a} 
\end{equation}

We have just learned that $\int\nolimits_{S}{} \vc{E} \cdot d\vc{a}$
equals $4 \pi q_n$ if the charge $q_n$ is inside $S$ and
equals zero otherwise. So every charge $q$ inside the surface
contributes exactly $4 \pi q$ to the surface integral of Eq. 20 and
all charges outside contribute nothing. We have arrived at Gauss's
law:

\begin{framed}
The flux of the electric field \vc{E} through any closed surface,
that is, the integral $\int {\vc{E}} \cdot d \vc{a}$ over
the surface, equals $4\pi$ times the goal charge enclosed by the
surface:

\begin{equation} 
  \int {\vc{E}} \cdot d \vc{a} =  4\pi \sum_i q_i =  4\pi \int \rho dv 
\end{equation}
\end{framed}

We call the statement in the box a \emph{law} because it is
equivalent to Coulomb's law and could serve equally well as the basic
law of electrostatic interactions, after charge and field have been
defined. Gauss's law and Coulomb's law are not two independent
physical laws, but the same law expressed in different ways.
\footnote{There is one difference, inconsequential here, but relevant
to our later study of the fields of moving charges. Gauss's law is
obeyed by a wider class of fields than those represented by the
electrostatic fired. In particular, a field that is inverse-square in
$r$ but not spherically symmetrical can satisfy Gauss's law.

In other words, Gauss's law alone does not imply the symmetry of the
field of a point source which is implicit in Coulomb's law.}

Looking back over our proof, we see that it hinged on the
inverse-square nature of the interaction and of course on additivity
of interactions, or superposition. Thus the theorem is applicable to
any inverse-square field in physics, for instance to the
gravitational field, as discussed in Vol. I, Chap. $9$.

It is easy to see that Gauss's law would \emph{not} hold if the law
of force were, say, inverse-cube. For in that case the flux of
electric field 
% p. 26
from a point charge $q$ through a sphere
through a sphere of radius $R$ centered on the charge would be

\begin{equation} 
\Phi =  \vc{\int E} \cdot d \vc{a} = \frac{q}{R^3} \cdot 4 \pi R^2 =  \frac{4 \pi q}{R}
\end{equation}

By making the sphere large enough we could make the flux through it
as small as we pleased, while the total charge inside remained
constant.

This remarkable theorem enlarges our graph in two ways. First, it
reveals a connection between the field and its sources that is
converse to Coulomb's law. Coulomb's law tells us how to derive the
electric field if the charges are given; with Gauss's law we can
determine how much charge is in any region if the field is known.

Second, the mathematical relation here demonstrated is a powerful
analytic tool; it can make complicated problems easy, as we shall see.

\section{Field of a spherical charge distribution}

We can use Gauss's law to find the electric field of a spherically
symmetrical distribution of charge, that is, a distribution in which
the charge density $\rho$ depends only on the radius from a central
point. Figure 1.18 depicts a cross section through some such
distribution. In this one the charge density is high at the center,
decreases and then rises again with increasing distance from the
center, and is ero beyond $r_0$. What is the electric field at some
point such as $P_1$ arising from each elementary volume in the charge
distribution. Lets try a different approach which exploits both the
symmetry of the system and Gauss's law.

Because of the spherical symmetry, the electric field at any point
must be radially directed ---no other direction is unique.

Likewise, the field magnitude $E$ must be the same at all points on a
spherical surface $S_1$ of radius $r_1$, for all such points are
equivalent. Call this field magnitude $E_1$. The flux through this
surface $S_1$ is therefore simply $4 \pi r_1^2E_1$, and by Gauss's
law this must be equal to $4\pi$ times the charge enclosed by the
surface. That is, $4 \pi r_1^2E_1 =  4\pi$ (charge inside $S_1$ or

\begin{equation} 
  E_1 =  \frac{\text{charge inside $S_1$}}{r_1^2}
\end{equation}
