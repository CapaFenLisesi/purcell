\chapter{The fields of moving charges}

\section{From Oersted to Einstein}

In the winter of 1819-20 Hans Christian Oersted was lecturing on
Electricity, Galvanism, and Magnetism to advanced students at the
University of Copenhagen. \emph{Electricity} meant electrostatics; galvanism
referred to the effects produced by continuous currents from
batteries, a subject opened up by Galvani's chance discovery and the
subsequent experiments of Volta; \emph{magnetism} dealt with the already
ancient lore of lodestones, compass needles, and the terrestrial magnetic
field. It seemed clear to some that there must be a relation between
galvanic currents and electric charge, although there was little
more direct evidence than the fact that both could cause shocks. On
the other hand, magnetism and electricity appeared to have nothing
whatever to do with one another. Still, Oersted had a notion, vague
perhaps, but tenaciously pursued, that magnetism like the galvanic
current might be a sort of ``hidden form'' of electricity. Groping for
some manifestation of this, he tried before his class the experiment
of passing a galvanic current through a wire which ran above and at
right angles to a compass needle. It had no effect. After the lecture,
something impelled him to try the experiment with a wire running
parallel to the compass needle. The needle swung wide---and when
the galvanic current was reversed it swung the other way!

The scientific world was more than ready for this revelation. A
ferment of experimentation and discovery followed as soon as the
word reached other laboratories. Before long Ampere, Faraday, and
others had worked out an essentially complete and exact description
of the magnetic action of electric currents. Faraday's crowning discovery
of electromagnetic induction came less than 12 years after
Oersted's experiment. In the previous two centuries since the publication
in 1600 of William Gilbert's great work \emph{De Magnete}, man's
understanding of magnetism had advanced not at all. Out of these
experimental discoveries there grew the complete classical theory of
electromagnetism. Formulated mathematically by Maxwell, it was
triumphantly corroborated by Hertz's demonstration of electromagnetic
waves in 1888.

Special relativity has its historical roots in electromagnetism.
Lorentz, exploring the electrodynamics of moving charges, was led
very close to the final formulation of Einstein. And Einstein's great
paper of 1905 was entitled not ``Theory of Relativity,'' but rather ``On
148
% p. 149
the Electrodynamics of Moving Bodies.'' Today we see in the
postulates of relativity and their implications a wide framework, one
that embraces all physical laws and not solely those of electromagnetism.
We expect any complete physical theory to be relativistically
invariant. It ought to tell the same story in all inertial frames of
reference. As it happened, physics already \emph{had} one relativistically
invariant theory---Maxwell's electromagnetic theory---long before
the significance of relativistic invariance was recognized. Whether
the ideas of special relativity could have evolved in the absence of a
complete theory of the electromagnetic field is a question for the
historian of science to speculate about; probably it can't be answered.
We can only say that the actual history shows rather plainly a path
running from Oersted's compass needle to Einstein's postulates.

In this chapter and Chap. 6 we are going to follow that path \emph{almost
in reverse}. This implies no disrespect for history. Indeed, we think
a student of the history of those magnificent discoveries will not be
handicapped by a clear view of the essential relation between electricity
and magnetism. That relation can be exposed very directly
and simply by looking, in the light of special relativity, at what we
have already learned about electric charge and the electric field.

Before we do that, let's review some of the phenomena we shall be
trying to explain.

\section{Magnetic forces}

Two wires running parallel to one another and carrying currents
in the same direction are drawn together. The force on one of the
wires, per unit length of wire, proves to be proportional to the product
of the two currents and inversely proportional to the distance between
the wires (Fig. 5.1a). Reversing the direction of one of the
currents changes the force to one of repulsion. Thus the two sections
of wire in Fig. 5.1b which are part of the same circuit tend to fly apart.
There is some sort of ``action-at-a-distance'' between the two filaments
of steady electric current. It seems to have nothing to do with
any static electric charge on the surface of the wire. There may be
some such charge and the wires may be at different potentials, but
the force we are concerned with depends only on the charge \emph{movement}
in the wires, that is, on the two currents. You can put a sheet
of metal between the two wires without affecting this force at all
(Fig. 5.1c). These new forces that come into play when charges are
moving are called \emph{magnetic}.

% p. 150
 
Oersted's compass needle (Fig. 5.2a) doesn't look much like a
direct-current circuit. We now know, however, as Ampere was the
first to suspect, that magnetized iron is full of perpetually moving
charges---electric currents on an atomic scale. A slender coil of wire
with a battery to drive current through it (Fig. 5.2b) behaves just
like the compass needle under the influence of a nearby current.

Observing the motion of a free charged particle, instead of awire
carrying current, we find the same thing happening. In a cathode-ray
tube, electrons that would otherwise follow a straight path are 
deflected toward or away from an external current-carrying wire, depending
on the relative direction of the current in that wire (Fig. 5.3).
You are already familiar with this from the laboratory, and you know
that this interaction of currents and other moving charges can be
described by introducing a \emph{magnetic field}. (The electric field,
remember, was simply a way of describing the ``action-at-a-distance''
between stationary charges that is expressed in Coulomb's law.)
We say that an electric current has associated with it a magnetic field
which pervades the surrounding space. Some other current, or any
moving charged particle which finds itself in this field, experiences a
force proportional to the strength of the magnetic field in that
locality. The force is always perpendicular to the velocity, for a
charged particle. The entire force on a particle carrying charge $q$ is
given by
\begin{equation}
  \vc{F} = q\vc{E} + \frac{q}{c}\vc{v}\times\vc{B}
\end{equation}
where $\vc{B}$ is the magnetic field.

% p. 151
We shall take Eq. 1 as the definition of $\vc{B}$. The magnetic field
strength is a vector which determines the velocity-proportional part
of the force on a moving charge. In other words, the command,
``Measure the direction and magnitude of the vector $\vc{B}$ at such and
such a place,'' calls for the following operations: Take a particle of
known charge $q$. Measure the force on $q$ at rest, to determine \vc{E}.
Then measure the force on the particle when its velocity is $\vc{v}$; repeat
with $\vc{v}$ in some other direction. Now find a $\vc{B}$ that will make Eq. 1
fit all these results; that is the magnetic field at the place in question.

Clearly this doesn't explain anything. Why does Eq. 1 work?
Why can we always find a $\vc{B}$ that is consistent with this simple 
relation, for all possible velocities? We want to understand why there
is a velocity-proportional force. It is really most remarkable that
this force is strictly proportional to $\vc{v}$, and that the effect of the electric
field does not depend on $\vc{v}$ at all! In the following pages we'll see
how this comes about.

\section{Measurement of charge in motion}

How are we going to measure the quantity of electric charge on a
moving particle? Until this question is settled, it is pointless to ask
what effect motion has on charge itself. A charge can only be measured
by the effects it produces. A point charge $Q$ which is at rest can
be measured by determining the force that acts on a test charge q a
certain distance away (Fig. 5.4a). That is based on Coulomb's law.
But if the charge we want to measure is moving, we are on uncertain
ground. There is now a special direction in space, the instantaneous
direction of motion. It could be that the force on the test charge $q$
depends on the direction from $Q$ to $q$, as well as on the distance between
the two charges. For different positions of the test charge, as
in Fig. 5.4b, we would observe different forces. Putting these into
Coulomb's Law would lead to different values for the same quantity
$Q$. Also we have as yet no assurance that the force will always
be in the direction of the radius vector $\vc{r}$.

To allow for this possibility, let's agree to define $Q$ by averaging
over all directions. Imagine a large number of infinitesimal test
charges distributed evenly over a sphere (Fig. 5.4c). At the instant
the moving charge passes the center of the sphere, the radial component
of force on each test charge is measured, and the average of these
force magnitudes is used to compute $Q$. Now this is just the operation
% p. 152
that would be needed to determine the surface integral of the
electric field over that sphere, at time $t$. The test charges here are all
at rest, remember; the force on $q$ per unit charge gives, by definition,
the electric field at that point. This suggests that Gauss's law, rather
than Coulomb's law, offers the natural 
way\footnote{It is not the only \emph{possible} way. You could, for instance, adopt the arbitrary rule that
the test charge must always be placed directly ahead (in the direction of motion) of the
charge to be measured. Charge so defined would \emph{not} have the simple properties we are
about to discuss, and your theory would prove clumsy and complicated.} to define quantity of
charge for a moving charged particle, or for a collection of moving
charges. We can frame such a definition as follows.

The amount of electric charge in a region is defined by the surface
integral of the electric field $\vc{E}$ over a surface $S$ enclosing the region.
This surface $S$ is fixed in some coordinate frame $F$. The field $\vc{E}$ is
measured, at any point $(x,y,z)$ and at time $t$ in $F$, by the force on a
test charge at rest in $F$, at that time and place. The surface integral
is to be determined for a particular time $t$. That is, the field values
used are those measured simultaneously by observers deployed all
over $S$. (This presents no difficulty, for $S$ is stationary in the frame $F$.)
Let us denote such a surface integral, over $S$ at time $t$, by
\begin{equation}
  \int_{S(t)} \vc{E}\cdot\der\vc{a}
\end{equation}
We define the amount of charge inside $S$ as $1/4\pi$ times this integral:
\begin{equation}
  Q = \frac{1}{4\pi} \int_{S(t)} \vc{E}\cdot\der\vc{a}
\end{equation}

It would be embarrassing if the value of $Q$ so determined depended
on the size and shape of the surface $S$. For a stationary charge it
doesn't---that is Gauss's law. But how do we know that Gauss's law
holds when charges are moving? Fortunately it does. We can take
that as an experimental fact. This fundamental property of the
electric field of moving charges permits us to define quantity of
charge by Eq. 3. From now on we can speak of the amount of charge
in a region or on a particle, and that will have a perfectly definite
meaning even if the charge is in motion.

Figure 5.5 summarizes these points in an example. Two protons
and two electrons are shown in motion, at a particular instant of
time. It is a fact that the surface integral of the electric field $\vc{E}$ over
the surface $S_1$ is precisely equal to the surface integral over $S_2$
evaluated at the same instant, and we may use this integral, as we
% p. 153
always have used Gauss's law in electrostatics, to determine the total
charge enclosed. Figure 5.6 raises a new question. What if the
same particles had some other velocities? For instance, suppose the
two protons and two electrons combine to form a hydrogen molecule.
Will the total charge appear exactly the same as before?

\section{Invariance of charge}

There is conclusive experimental evidence that the total charge in
a system is not changed by the motion of the charge carriers. We
are so accustomed to taking this for granted that we seldom pause
to think how remarkable and fundamental a fact it is. For proof,
we can point to the exact electrical neutrality of atoms and molecules.
We have already described in Chap. 1 the experimental test of the
neutrality of the hydrogen molecule which proved that the electron
and proton carry charges equal in magnitude to better than 1 part
in $10^{20}$. A similar experiment was performed with helium atoms.
Now the helium atom contains two protons and two electrons, the
same charged particles that make up the hydrogen molecule. In the
helium atom their motion is very different. The protons, in par-
ticular, instead of revolving slowly 0.07 nm apart, are tightly
bound into the helium nucleus where they move with kinetic energies
in the range of a million electron volts. If \emph{motion} had any effect on
the amount of charge, we could not have exact cancellation of nuclear
and electronic charge in \emph{both} the hydrogen molecule and the helium
atom. In fact, the helium atom was shown to be neutral with nearly
the same experimental accuracy.

Another line of evidence comes from the optical spectra of isotopes
of the same element, atoms with different nuclear masses but,
nominally at least, the same nuclear charge. Here again we find a
marked difference in the motion of the protons within the nucleus,
but comparison of the spectral lines of the two species shows no discrepancy
that could be attributed to even a slight difference in total
nuclear charge.

Mass is \emph{not} invariant in the same way. We know that the mass of
a particle is changed by its motion, by the factor $1/(1-v^2/c^2)^{1/2}$.
To emphasize the difference, we show in Fig. 5.7 an imaginary ex-
periment. In the box on the right the two massive charged particles,
which are fastened to the end of a pivoted rod, have been set revolving
with speed $v$. The entire mass on the right is greater than the
mass on the left, as demonstrated by weighing the box on a spring
% p. 154
balance or by measuring the force required to 
accelerate it.\footnote{The difference in mass depends not only on the kinetic energy of the particles but
also on any change in potential energy as in the elastic strain in the rod that holds the
particles. If the rod is fairly stiff, this contribution will be small compared to the $v^2/c^2$
term. See if you can show why.} The
total electric charge, however, is unchanged. A real experiment
equivalent to this can be carried out with a mass spectrograph, which
can reveal quite plainly a mass difference between an ionized deuterium
molecule (2 protons, 2 neutrons, 1 electron) and an ionized
helium atom (also 2 protons, 2 neutrons, and 1 electron). These are
two very different structures, in which the component particles are
moving with very different speeds. The difference in energy shows
up as a measurable difference in mass. There is no detectable dif-
ference, to very high precision, in the electric charge of the two ions.

% p. 155

This invariance of charge lends a special significance to the fact of
charge quantization. We emphasized in Chap. 1 the importance-
and the mystery---of the fact that every elementary charged particle
has a charge equal in magnitude to that of every other such particle.
We now observe that this precise equality holds not only for two
particles at rest with respect to one another, but for \emph{any} state of relative
motion.

The experiments we have described, and many others, show that
the value of our 
Gauss's law surface integral $\int_S\vc{E}\cdot\der\vc{a}$ \emph{depends only on
the number and variety of charged particles inside $S$, and not on how
they are moving.} According to the postulate of relativity, such a
statement must be true for any inertial frame of reference if it is true
for one. Therefore if $F'$ is some \emph{other} inertial frame, moving with
respect to $F$, and if $S'$ is a closed surface in that frame which at time $t'$
encloses the same charged bodies that were enclosed by $S$ at time $t$,
we must have
\begin{equation}
  \int_S\vc{E}\cdot\der\vc{a} = \int_{S'}\vc{E}'\cdot\der\vc{a}'
\end{equation}

The field $\vc{E}'$ is of course measured in $F'$, that is, it is defined by the
force on a test charge at rest in $F'$. The distinction between $t$ and $t'$
must not be overlooked. As we know, events that are simultaneous
in $F$ need not be simultaneous in $F'$. Each of the surface integrals in
Eq. 4 is to be evaluated at one instant in its frame. If charges lie on
the boundary of $S$, or of $S'$, one has to be rather careful about ascertaining
that the charges within $S$ at $t$ are the same as those within $S'$
at $t'$. If the charges are well away from the boundary, as in Fig. 5.8
which is intended to illustrate the relation in Eq. 4, there is no problem
in this respect.

Equation 4 is a formal statement of the relativistic invariance of
charge. We can choose our Gaussian surface in \emph{any} inertial frame;
the surface integral will give a number independent of the frame.
This is not the same as charge conservation, which was discussed in
Chap. 4 and expressed mathematically in the equation
\begin{equation*}
  \div\vc{J} = -\frac{\partial\rho}{\partial t}
\end{equation*}
Charge \emph{conservation} implies that if we take a closed surface fixed in
some coordinate system, containing some charged matter, and if no
particles cross the boundary, then the total charge inside that surface
% p. 156
remains constant. Charge \emph{invariance} implies that if we look at this
collection of stuff from any other frame of reference we will measure
exactly the same amount of charge. Energy is conserved, but energy
\emph{is not} a relativistic invariant. Charge is conserved and charge is a
relativistic invariant. In the language of relativity theory, energy is
one component of a four-vector, while charge is a scalar, an invariant
number, with respect to the Lorentz transformation. This is an observed
fact with far-reaching implications. It completely determines
the nature of the field of moving charges.

\section{Electric field measured in different frames of reference}

If charge is to be invariant under a Lorentz transformation, the
electric field $\vc{E}$ has to transform in a particular way. ``Transform-
ing $\vc{E}'$' means answering a question like this: If an observer in a certain
inertial frame $F$ measures an electric field $\vc{E}$ as so-and-so-many
statvolts/cm, at a given point in space and time, what field will be
measured at the same space-time point by an observer in a different
inertial frame $F'$? For a certain class of fields, we can answer this
question by applying Gauss's law to some simple systems.

In the frame $F$ (Fig. 5.9a) there are two stationary sheets of charge
of uniform density $+\sigma$ and $-\sigma$ $\esu/\cmunit^2$. They are
squares $b$ cm on a side lying parallel to the $xy$ plane, and their separation
is supposed to be so small compared to their extent that the field
between them can be treated as uniform. The magnitude of this
field, as measured by an observer in $F$, is $4\pi\sigma$. Now
consider an inertial frame $F'$ which moves toward the left, with
respect to $F$, with velocity $\vc{v}$. To an observer in $F'$, the charged
``squares'' are no longer square. Their $x'$ dimension is contracted
from $b$ to $b\sqrt{1-\beta^2}$, where $\beta$ stands for $v/c$, as usual. But total
charge is invariant, that is, independent of reference frame, so the
charge \emph{density} measured in $F'$ is greater than $\sigma$ in the ratio $1/\sqrt{1-\beta^2}$.
Figure 5.9 shows the system in cross section, (b) as seen in $F$ and (c)
as seen in $F'$. What can we say about the electric field in $F'$, if all
we know about the electric field of moving charges is contained in
Eq. 4?

For one thing, we can be sure that the electric field is zero outside
the sandwich, and uniform between the sheets, at least in the limit
as the extent of the sheets becomes infinite. The field of an infinite
uniform sheet could not depend on the distance from the sheet, nor
on position along the sheet. (There is nothing in the system to fix
% p. 157
a distance scale or a position; if the field varied according to a power
law like the field of a point or line charge, it would be infinite at the
sheet.) However, for all
we know now,\footnote{Remember this is a moving sheet of charge, in $F'$; we have as yet no guarantee that
its field will be like that of a stationary sheet. In fact, as it turns out, the electric field
of a moving sheet of charge is perpendicular to the sheet, unlike the hypothetical fields
in Fig. 5.9d and e} the field of a single moving
sheet of positive charge could look like Fig. 5.9d. But if it did, the
field of a sheet of negative charge would have to look like Fig. 5.9e,
so the superposition of the two would still give a field of the character
indicated in Fig. 5.9f.

We can apply Gauss's law to a box stationary in frame $F'$, the box
shown in cross section in Fig. 5.9f. The charge content is determined
by $\sigma'$, and the field is zero outside the sandwich. Gauss's law tells us
that the magnitude of $E_z'$, which is the only field component inside,
must be $4\pi\sigma'$, or $4\pi\sigma/\sqrt{1-\beta^2}$.
\begin{equation*}
  E_z' = \frac{E_z}{\sqrt{1-\beta^2}} = \gamma E_z
\end{equation*}
(For the factor $1/\sqrt{1-\beta^2}$ we shall often use the symbol $\gamma$, introduced
in Vol. 1, Chap. 11, Eq. 13. It saves a lot of writing. Remember,
$\gamma\ge1$, always.)

Now imagine a different situation with the stationary charged
sheets in the frame $F$ oriented perpendicular to the $x$ axis, as in
Fig. 5.10. The observer in $F$ now reports a field in the $x$ direction of
magnitude $E_x=4\pi\sigma$. In this case, the surface charge density observed
in the frame $F'$ is the same as that observed in $F$. The sheets
are not contracted; only the distance between them is contracted, but
that doesn't enter into the determination of the field. This time we
find by applying Gauss's law to the box stationary in $F'$:
\begin{equation}
  E_x' = 4\pi\sigma' = 4\pi\sigma = E_x
\end{equation}

That is all very well for the particularly simple arrangement of
charges here pictured; do our conclusions have more general validity?
This question takes us to the heart of the meaning of \emph{field}.
If the
electric field $\vc{E}$ at a point in space-time is to have a unique meaning,
then the way $\vc{E}$ appears in other frames of reference, in the same
space-time neighborhood, cannot depend on the nature of the
sources, wherever they may be, that produced $\vc{E}$. In other words,
the observer in $F$, having measured the field in his neighborhood at
some time, ought to be able to predict \emph{from these measurements alone}
what observers in other frames of reference would measure at the
% p. 158
same space-time point. Were this not true, \emph{field} would be a useless
concept. The evidence that it is true is the eventual agreement of
our field theory with experiment.

Seen in this light, the relations expressed in Eqs. 5 and 6 take on
a significance beyond the special case of charges on parallel sheets.
Consider \emph{any} charge distribution all parts of which are at rest with
respect to the frame $F$. If an observer in $F$ measures a field $E_z$ in the
$z$ direction, then an observer in the frame $Ff$' will report, for the same
space-time point, a field $E_z'=\gamma E_z$. That is, he will get a number, as
the result of his $E_z'$ measurement, which is larger by the factor $\gamma$ than
the number the $F$ observer got in his $E_z$ measurement. On the
other hand, if the observer in $F$ measures a field $E_x$ in the $x$ direction,
the direction of the velocity of $F'$ with respect to $F$, then the observer
in $F'$ reports a field $E_x'$ equal to $E_x$. Obviously the $y$ and the $z$ directions
are equivalent, both being transverse to the velocity $\vc{v}$. Anything
we have said about $E_z'$ applies to $E_y'$, too. Whatever the direction
of $\vc{E}$ in the frame $F$, we can treat it as a superposition of fields in
the $x$, the $y$, and the $z$ directions, and from the transformation of each
of these predict the vector field $\vc{E}'$ at that point in $F'$. Let's summarize
this in words appropriate to relative motion in any direction:
Charges at rest in frame $F$ are the source of a field $\vc{E}$. Let frame $F'$
move with speed $\vc{v}$ relative to $F$. At any point in $F$, resolve $\vc{E}$ into a
``longitudinal'' component $E_\parallel$ parallel to $\vc{v}$ and a transverse component
$E_\perp$ perpendicular to the direction of $\vc{v}$. At the same space-time
point in $F'$, the field $\vc{E}'$ is to be resolved into $E'_\parallel$ and $E'_\perp$,
$E'_\parallel$ being
parallel to $\vc{v}$ and $E'_\perp$ perpendicular thereto. We have learned that
\begin{framed}
\begin{align}
  \begin{split}
    E'_\parallel &= E_\parallel \\
    E'_\perp     &= \gamma E_\perp
  \end{split}
\end{align}
\end{framed}

Our conclusion holds only for fields that arise from charges stationary
in $F$. As we shall see presently, if charges in $F$ are moving, the prediction
of the electric field in $F'$ involves knowledge of two fields in $F$,
the electric and the magnetic. But we already have a useful result,
one that suffices whenever we can find any inertial frame of reference
in which all the charges remain at rest. We shall use it now to study
the electric field of a point charge moving with constant velocity.

\section{Field of a point charge moving with constant velocity}
In the frame $F$ the point charge $Q$ remains at rest at the origin
(Fig. 5.11a). At every point the electric field $\vc{E}$ has the magnitude
% p. 159
$Q/r^2$ and is directed radially outward. In the $xz$ plane its components
at any point $(x,z)$ are
\begin{align}
\begin{split}
  E_x &= \frac{Q}{r^2}\cos\theta = \frac{Qx}{(x^2+z^2)^{3/2}} \\
  E_z &= \frac{Q}{r^2}\sin\theta = \frac{Qz}{(x^2+z^2)^{3/2}} 
\end{split}
\end{align}
Let the frame $F'$ move in the negative $x$ direction, with speed $v$. The
relations between the coordinates of an event, or space-time point,
in the two frames are
\begin{equation}
  x=\gamma(x'-\beta c t') \quad y=y' \quad z=z' \quad t=\gamma\left(t'-\frac{\beta x'}{c}\right)
\end{equation}
This is the Lorentz transformation given in Eq. 15 of Chap. 11, Vol.1.
We have minus signs in our equations above because our frame $F'$ is
moving in the negative $x$ direction, seen from $F$. The clocks have
been set to read zero when $x=0$ and $x' = 0$ coincide.

According to Eqs. 5 and 6, $E_z' = \gamma E_z$, and $E_x'=E_x$. Using Eqs. 8
and 9, we can express the field components $E_z'$ and $E_x'$ in terms of the
coordinates in $F'$. For the instant $t' = 0$, when $Q$ passes the origin
in $F'$, we have:
\begin{align}
\begin{split}
  E_x' = E_x \;     &= \frac{\gamma Qx'}{((\gamma x')^2+z'^2)^{3/2}} \\
  E_z' = \gamma E_z &= \frac{\gamma Qz'}{((\gamma x')^2+z'^2)^{3/2}} 
\end{split}
\end{align}
% p. 160

Note first that $E_z'/E_x' = z' / x'$. This tells us that the vector $\vc{E}'$ makes
the same angle with the $x'$ axis as does the radius vector $r'$. Hence
$\vc{E}'$ points radially outward along a line drawn from the instantaneous
position of $Q$, as in Fig. 5.11b. Pause a moment to let this conclusion
sink in! It means that if $Q$ passed the origin of the primed system at
precisely 12:00 noon, ``prime time,'' an observer stationed anywhere
in the primed system will report that the electric field in his vicinity
was pointing, at 12:00 noon, exactly radially from the origin. This
sounds at first like instantaneous transmission of information! How
can an observer a mile away know where the particle is at the same
instant? He can't. That wasn't implied. This particle, remember,
has been moving at constant speed forever, on a ``flight plan'' that
calls for it to pass the origin at noon. That information has been
available for a long time. It is the past history of the particle that
determined the field observed, if you want to talk about cause and
effect. We'1l inquire presently into what happens when there is an
unscheduled change in the flight plan.

To find the strength of the field, we compute $E_x'^2+E_z'^2$, which is
the square of the magnitude of the field, $E'^2$.
\begin{align}
\begin{split}
  E'^2 = E_x'^2+E_z'^2 
        &= \frac{\gamma^2 Q^2 (x'^2+z'^2)}{\left[(\gamma x')^2+z'^2\right]^3}
         = \frac{Q^2 (x'^2+z'^2)}{\gamma^4\left[x'^2+z'^2-\beta^2 z'^2\right]^3} \\
        &= \frac{Q^2(1-\beta^2)^2}{(x'^2+z'^2)^2\left(1-\frac{\beta^2 z'^2}{x'^2+z'^2}\right)^3}
\end{split}
\end{align}
(Here, for once, it was neater with $\beta$ worked back into the expression.) 
Let $r'$ denote the distance from the charge $Q$, which is
momentarily at the origin, to the point $(x',z')$ where the field is
measured: $r' = (x'^2 + z'^2)^{1/2}$. Let $\theta'$ denote the angle between this
radius vector and the velocity of the charge $Q$, which is moving in
the positive $x'$ direction in the frame $F'$. Then since $z' = r' \sin \theta'$, the
magnitude of the field can be written as:
\begin{equation}
  E' = \frac{Q}{r'^2} \frac{1-\beta^2}{(1-\beta^2\sin^2\theta')^{3/2}}
\end{equation}
There is nothing special about the origin of coordinates, nor about
the $x'z'$ plane as compared to any other plane through the $x'$ axis.
Therefore we can say quite generally that the electric field of a charge
in uniform motion, at a given instant of time, is directed radially from
the instantaneous position of the charge, while its magnitude is given
by Eq. 12 with $\theta'$ the angle between the direction of motion of the
% p. 161
charge and the radius vector from the instantaneous position of the
charge to the point of observation.

For low speeds the field reduces simply to $E' \approx Q/r'^2$, and is practically
the same, at any instant, as the field of a point charge stationary
in $F'$ at the instantaneous location of $Q$. But if $\beta^2$ is not
negligible, the field is stronger at right angles to the motion than in
the direction of the motion, at the same distance from the charge.
If we were to indicate the intensity of the field by the density of field
lines, as is often done, the lines tend to concentrate in a pancake
perpendicular to the direction of motion. Figure 5.12 shows the
density of lines as they pass through a unit sphere, from a charge
moving in the $x'$ direction with a speed $v/c = 0.866$. A simpler representation
of the field is shown in Fig. 5.13, a cross section through
the field with some field lines in the $x'z'$ plane 
indicated.\footnote{A \emph{two-dimensional}
diagram like Fig. 5.13 cannot faithfully represent the field intensity
by the density of field lines. Unless we arbitrarily break off some of the lines,
the density of lines in the picture will fall off as $1/r'$, whereas the intensity of the field
we are trying to represent falls off as $1/r'^2$. So Fig. 5.13 gives only a qualitative 
indication of the variation of $E'$ with $r'$ and $\theta'$.}

This is a remarkable electric field. It is not spherically symmetrical,
which is not surprising because in this frame there is a preferred
direction, the direction of motion of the charge. Also, it is a field
that \emph{no stationary charge distribution}, whatever its form, could produce.
For in this field the line integral of $\vc{E}'$ is not zero around every
closed path. Consider, for example, the closed path $ABCD$ in
Fig. 5.13. The circular arcs contribute nothing to the line integral,
being perpendicular to the field; on the radial sections, the field is
stronger along $BC$ than along $DA$, so the circulation of $\vc{E}'$ on this path
is not zero. But remember, this is not an electrostatic field. In the
% p. 162
% p. 163
course of time the electric field $\vc{E}'$ at any point in the frame $F'$ changes
as the source charge moves.

Figures 5.14 and 5.15 show the electric field at certain instants of
time observed in a frame of reference through which an electron is
moving at constant velocity in the $x$
direction.\footnote{Previously we had the charge at rest in the unprimed frame, moving in the primed
frame. Here we adopt $xyz$ for the frame in which the charge is moving, to avoid cluttering
the subsequent discussion with primes.} In Fig. 5.14, the
speed of the electron is $0.33c$, its kinetic energy therefore (Vol. 1,
Chap. 12) about 30,000 electron-volts (30 keV). The value of $\beta^2$
is $1/9$ and the electric field does not differ greatly from that of a charge
at rest. In Fig. 5.15, the speed is $0.8c$, corresponding to a kinetic
energy of 335 keV. If the time unit for each diagram is taken as
$1.0 \times 10^{-10}\ \sunit$, the distance scale is life-size, as drawn. Of course,
the diagram holds equally well for \emph{any} charged particle moving at
the specified fraction of the speed of light. We mention the equivalent
energies for an electron merely to remind the reader that relativistic
speeds are nothing out of the ordinary in the laboratory.

\iffalse

\section{Field of a charge that starts or stops}

It must be clearly understood that uniform velocity, as we have
been using the term, implies a motion at constant speed in a straight
line that has been going on forever. What if our electron had not
been traveling in the distant past along the negative x axis until it
came into view in our diagram at t = 0? Suppose it had been sitting
quietly at rest at the origin, waiting for the clock to read t = 0. Just
prior to t = 0, something gives the electron a sudden large accelera-
tion, up to the speed v, and it moves away along the positive x axis
at this speed. Its motion from then on precisely duplicates the motion
of the electron for which Fig. 5.15 was drawn. But Fig. 5.15
does not correctly represent the field of the electron whose history
was just described. To see that it cannot do so, consider the field at
the point marked P, at time t = 2, which means 2 X 10'10 sec. In
2 x 10-10 see a light signal travels 6 cm. Since this point lies more
than 6 cm distant from the origin, it could not have received the news
that the electron had startedto move at t = 0! Unless there is a
gross violation of relativity---and we are taking the postulates of
relativity as basis for this whole discussion---the field at the point P
at time t = 2, and indeed at all points outside the sphere of radius

6 cm centered on the origin, must be the field of a charge at rest at the
origin.


% p. 164
 

W

On the other hand, close to the moving charge itself, what happened
in the remote past can't make any difference. The field must
somehow change, as we consider regions farther and farther from
the charge, at the given instant t = 2, from the field shown in the
second diagram of Fig. 5.15 to the field of a charge at the origin. We
can't deduce more than this without knowing how fast the news does
travel. Suppose---just suppose---it travels as fast as it can without
confficting with the relativity postulates. Then if the period of acceleration
is neglected, we should expect the field within the entire
6-cm-radius sphere, at t = 2, to be the field of a uniformly moving
point charge. If that is so, the field of the electron which starts from
rest, suddenly acquiring the speed 12 at t = 0, must look something
like Fig. 5.16. There is a thin spherical shell (whose thickness in an
actual case will depend on the duration of the interval required for
acceleration) within which the transition from one type of field to
the other takes place. This shell simply expands with speed c, its
center remaining at x = 0. The arrowheads on the field lines indicate
the direction of the field when the source is a negative charge
as we have been assuming.

Figure 5.17 shows the field of an electron which had been moving
with uniform velocity until t = 0, at which time it reached x = 0
where it was abruptly stopped. Now the news that it was stopped

% p. 165
  

cannot reach, by time t, any point farther than ct from the origin.
The field outside the sphere of radius R = ct must be that which
would have prevailed if the electron had kept on moving at its original
speed. That is why we see the ``brush'' of field lines on the right
in Fig. 5.17 pointing precisely down to the position where the electron
would be if it hadn't stopped. (Note that this last conclusion
does not depend on the assumption we introduced in the previous
paragraph, that the news travels as fast as it can.) The field almost
seems to have a life of its own!

It is a relatively simple matter to connect up the inner and outer
field lines. There is only one way it can be done that is consistent
with Gauss's law. Taking Fig. 5.17 as an example, from some point
such as A on the radial field line making angle 00 with the x axis,
follow the field line wherever it may lead until you emerge in the
outer field on some line making an angle that we may call (pg with
the x axis. (This line of course is radial from the extrapolated position
of the charge, the apparent source of the outer field.) Connect
A and D to the x axis by circular arcs, arc AE centered on the source
of the inner field, arc DF centered on the apparent source of the
outer field. Rotate the curve EABCDF around the x axis to generate
a surface of revolution. As the surface encloses no charge, the surface
integral of E over the entire surface must be zero. The only

% p. 166
 

contributions to the integral come from the spherical caps, for the
rest of the surface generated by ABCD is parallel to the field by the
very mode of its construction. The field over the inner cap is the
field of a point charge at rest; the field over the outer cap is the field
of a point charge with constant speed 0, as given by Eq. 12. Let us
calculate the flux through the inner cap, shown in Fig. 5.18. The
integral of E over this cap is:

\begin{equation}
\end{equation}
_f()0°rl22wr2sin0d0 = 27rq Laosinfidff (13)
The integral of E over the outer cap is:

'Poi 1-32 2 2.
L ,2 (1_'32sin2q,)3/2 7'' Smtpdl'

0 1- .
\begin{equation}
\end{equation}
:2wqL¢#S1n¢d@ (14)

The condition that the flux in at the left equal the flux out at the
right is therefore:

\begin{equation}
\end{equation}
90 . 'Po 1-32 .
J;)s1n0d0=L  s1n¢d¢ (15)

You can work this out as an exercise in integration? It yields the
following relation between 90 and (p0:

COS (Po

7T  "Q

COS 00 :

'['The integral needed is

I do: _ x
(a2 + x2)3/2 _ a2(a2 + x2)1/2 '

% p. 167
An equivalent, and simpler, way of expressing the same relation is:
tan (P0 = ytan 00 (17)
\begin{equation}
\end{equation}

Incidentally, this relation between 00 and cpo, based on the inclusion
of equal amounts of flux, is the same as the relation between the
angles that a rigid rod makes with the direction of relative motion,
in its rest frame and in the moving frame. This allows us to picture
the transformation to the field of a moving charge in a very simple
way: Let each ``line'' represent a certain amount of flux, and imagine
the lines in the rest frame of the charge as fixed rods sticking out in
all directions. In the moving frame, each rod represents the same
amount of flux, and the rods now appear at steeper angles, so that
the cluster of rods looks like Fig. 5.13.

It is only the width of the transition region in Fig. 5.17 that depends
on our as-yet-unsupported assumption that ``the news travels
as fast as it can.'' The connection expressed in Eq. 17 must hold if
there is any region close to the charge now at rest from which traces
of its history before t = 0 have vanished. Therefore in the field lines
that connect the near field to the remote field, there must be a trans-
verse, that is, nonradial, component.

The field lines in Figs. 5.16 and 5.17 were connected so as to conform
to the requirement of Eq. 17. The result is a rather intense
field in the transition region, with the field there running mainly perpendicular
to the radius vector from the origin. Remembering that
this field configuration expands with speed c as time goes on, we see
that we have something very much like an outgoing wave of transverse
electric field---transverse to the direction of propagation, that is.

We were led to this by following through consistently the consequences
of the relativity postulates and the observed fact that electric
charge is a relativistic invariant. We shall be able to use these ideas
later to understand the nature of the radiation from an accelerated
charge. First, however, we must return to the uniformly moving
charge, which has more surprises in store.

\section{Force on a moving charge}

Equation 12 tells us the force experienced by a stationary charge
in the field of another charge that is moving at constant velocity.
We now ask a different question: What is the force that acts on a
moving charge. one that moves in the field of some other charges?

% p. 168
We shall look first into the case of a charge moving through the field
produced by stationary charges. It might be an electron moving
between the charged plates of an oscilloscope, or an alpha particle
moving through the Coulomb field around an atomic nucleus. The
sources of the field, in any case, are all at rest in some frame of reference
which we shall call the ``lab.'' At some place and time in the
``lab'' frame we observe a particle carrying charge q which is moving,
at that instant, with velocity v through the electrostatic field. What
force appears to act on q?

Force is only a name for rate-of-change-of-momenturn, so we are
really asking, what is the rate of change of momentum of the particle,
dp/dt, at this place and time, as measured in our ``lab'' frame of ref-
erence? (That is all we mean by the force on a moving particle.)
The answer is contained, by implication, in what we have already
learned. Let's look at the system from a coordinate frame F' moving,
at the time in question, along with the particle. In this ``particle''
frame the particle will be, at least momentarily, at rest. It is the
other charges that are now moving. This is a situation we know
how to handle. The force on the stationary charge q is just $\vc{E}'$q where
$\vc{E}'$ is the electric field observed in the frame F'. Then we know how
to find $\vc{E}'$ when E is given; Eq. 7 provides our rule. Thus knowing
E, we can find the rate of change of momentum of the particle as
observed in F'. All that remains is to transform this quantity back
to F. So our problem hinges on the question, how does force, that
is, rate of change of momentum, transform from one inertial frame
to another?

That question was studied in Vol. 1, Chap. 12. You may remember
how it came out. However, instead of lifting the appropriate formulas
from Vol. I, we are going to review here some steps which lead
to them. This will help us to understand exactly what is going on.
Consider, then, any inertial frame F 'moving in the positive x direction
with speed 0 as observed in another frame F. In F', let a particle
of rest mass m be moving in the positive x' direction with speed v'.
Use p, to denote the x component of momentum (measured in F)

and p,' to denote the x' component of momentum (measured in F').
To find the relation between p, and p_,' we note that

I

1110

;=_____= " w

Here we have used the obvious abbreviations, B' = u'/c and

y=um-az

% p. 169
In the frame F, on the other hand, the speed of the particle is
(u + u')/(1 + vv'/c2) which can be written c(,B + ,8')/(1 + BB'),
50 that
\begin{equation}
\end{equation}

W : mc(B + ,8') = mc(B + ,8')
' 2 1/2 _ 2 __ '2 /2
(1 + BB,)[1_(1BJ;+B.l:;,) J [(1 .3 )(1 .3 )1'
= mm/(B + .3') (19)
\begin{equation}
\end{equation}

)n comparing Eq. 18 with Eq. 19, we find that p, is related to p,' as
bllows:

p1 = 7(1); + /3Y'mC) (20)
\begin{equation}
\end{equation}

We note that in the term ,8y'mc, the part y'mc is y'mc2/c, or $\vc{E}'$/c
/here E ' (not to be confused with the electric field in which we have
amporarily lost interest!) is the total energy, rest plus kinetic, of the
article in F'. Let us write Eq. 20 this way, p, = y(p,' + BE'/c), and
ause brieffy to compare it with the Lorentz transformation of the
oordinate x in the same situation: x = y(x' + ,8ct'). The similarity
zminds us of a general fact: in a Lorentz transformation the four
uantities p,, p,,, P24, and E/c behave exactly like the four space~time
aordinates, x, y, z, and at. Indeed, if you were firmly in command
f that fact you could have written Eq. 20 down directly and yomare
ntitled to regard our little review as a waste of time. Let's use the
[Ct as a substitute for a derivation for the transverse momentum
)mponents. Since y = y' in a Lorentz transformation with
directed relative velocity, we must expect that

Pa = Pu' (21)
\begin{equation}
\end{equation}

The relation between t and t' is given by the familiar formula:

_c__

t: y(r + Bx') (22)
\begin{equation}
\end{equation}

e are interested in the relation between dp,/dt and dp;/dt'. Differ-
tiating Eq. 22, we get:

dx'
dt'

dt: ydt'+ yE(

C

)dt" = ydr(1+ BB') (23)
\begin{equation}
\end{equation}

ice dx'/dt' is simply v'. Equation 21 gives us

dm = dry' (24)
\begin{equation}
\end{equation}

% p. 170
and differentiating Eq. 20 gives

dm = yap; + yBm(j*',)dp; <25)
\begin{equation}
\end{equation}

1'

In this last expression the factor mc(dy'/dpg) can be evaluated from
Eq. 18, which says that

Pr' = moi/B' = m0\/Y'2 - 1 (26)
\begin{equation}
\end{equation}

Thus

dp,' _ mcy' _ Lo 27

\begin{equation}
\end{equation}

dv" y'2-1'/3* "
Then dyl = TL = 'B/ and putting this into Eq. 25 we get

\begin{equation}
\end{equation}


dpé (dp;/dv') mo
dm = v dp;<1 + BB') (28)
Comparing Eqs. 23 and 28 we see that
d .z' d z'
5; = 2:, <2»

and this holds no matter how large 1)' is since the factor (1 + /3,8')
appeared in both equations. Actually, we shall be interested only in
situations where v' is very small---that is, where the particle is very
nearly at rest in the F' frame. In that case the term BB' can be neglected
and if we then compare Eqs. 23 and 24, we find, for the transverse
momentum change:

\begin{equation}
\end{equation}

dpv : L dpl;
dt y dt'

(30)

Summarizing the important results: F' is an inertial frame in
which a particle is instantaneously at rest or moving very slowly. F is
any other inertial frame with respect to which F' is moving with any
speed whatever. Using H and _L to label momentum components

respectively parallel to and transverse to the relative velocity of F'
and F, we can state that

\begin{equation}
\end{equation}

(31)

 

Equipped with the force transformation law, Eq. 31, and the transformation
law for the electric field, Eq. 7, we return now to our

% p. 171
charged particle moving through the field E, and we discover an
astonishingly simple fact. Consider first E", the component of E
parallel to the instantaneous direction of motion of our charged
particle. Transform to a frame F' moving, at that instant, with the
particle. In that frame the longitudinal electric field is Efi, and
according to Eq. 7, EN' = E". So the force dpfi /dt' is

\begin{equation}
\end{equation}

d ' ,

dig': = E.q = qE». (32)
Back in frame F, observers are measuring the longitudinal force, that
is, the rate of change of the longitudinal momentum component,

dpu/df. According to Eq. 31, dp"/dt : dpll/dt', so in frame F the
longitudinal force component they find is:

\begin{equation}
\end{equation}

d d '
72¢ = d";'l = GEN (33)

Of course the particle does not remain at rest in F' as time goes on.
It will be accelerated by the field $\vc{E}'$, and v', the velocity of the particle
in the inertial frame F', will gradually increase from zero. However,
as we are concerned with the instantaneous acceleration, only
infinitesimal values of v' are involved anyway, and the restriction
on Eq. 31 is rigorously fulfilled. For E L, the transverse field component
in F, the transformation is E' = yEi, so that (dpl/dt') =
qEi = qyEi. But on transforming the force back to frame F we have
(dpi/dt) = (1/7) (dpi/dt') so the 7 drops out after all:

\begin{equation}
\end{equation}

% = §(vEiq) = qEl (34)
The message of Eqs. 33 and 34 is simply this: the force acting on a
charged particle in motion through F is q times the electric field E in
that frame, strictly independent of the velocity of the particle.
Figure 5.19 is a reminder of this fact, and of the way we discovered it.

You have already used this result earlier in the course, where you
were simply told that the contribution of the electric field to the force
on a moving charge is qE. Because this is familiar and so simple, you
may think it is obvious and we have been wasting our time proving it.
Now we could have taken it as an empirical fact. It has been verified
over an enormous range, up to velocities so close to the speed of light,
in the case of electrons, that the factor 7 is 104. From that point of
view it is a most remarkable law. Our discussion in this chapter has
shown it to be a direct consequence of charge invariance.

% p. 172
\section{Interaction between a moving charge and other moving charges}

We know that there can be a velocity-dependent force on a moving
charge. That force is associated with a magnetic field, the sources of
which are electric currents, that is, other charges in motion.
Oersted's experiment showed that electric currents could influence
magnets, but at that time the nature of a magnet was totally
mysterious. Soon Ampere and others unraveled the interaction of
electric currents with each other, as in the attraction observed between
two parallel wires carrying current in the same direction. This
led Ampere to the hypothesis that a magnetic substance contains
permanently circulating electric currents. If so, Oersted's experi-

% p. 173
ment could be understood as the interaction of the ``galvanic'' current
in the wire with the permanent microscopic currents which gave
the compass needle its special properties. Ampere gave a complete
and elegant mathematical formulation of the interaction of steady
currents, and of the equivalence of magnetized matter to systems of
permanent currents. His brilliant conjecture about the actual nature
of magnetism in iron had to wait a century, more or less, for its
ultimate confirmation.

Whether the magnetic manifestations of electric currents arose
from anything more than the simple transport of charge was not clear
to Ampere and his contemporaries. Would the motion of an electro-
statically charged object cause effects like those produced by a continuous
galvanic current? Later in the century Maxwell's theoretical
work suggested the answer should be yes. The first direct evidence
was obtained by Henry Rowland, to whose experiment we shall return
at the end of Chap. 6.

From our present vantage point, the magnetic interaction of
electric currents can be recognized as an inevitable corollary
of Coulomb's law. If the postulates of relativity are valid, if electric
charge is invariant, and if Coulomb's law holds, then the effects we
commonly call ``magnetic'' are bound to occur. They will emerge as
soon as we examine the electric interaction between a moving charge
and other moving charges. A very simple system will illustrate this.

In the ``lab'' frame, as shown in Fig. 5.20:1, we have an infinitely
long procession of positive charges moving to the right with speed 00
and superimposed on it a procession of negative charges moving
toward the left with the same speed. These charges are supposed to
be so numerous and closely spaced that their discreteness can be
ignored at the distances we shall be concerned with. In the figure, for
clarity, we have drawn the two processions slightly separated. For
this system there exists no frame of reference in which all charges
are at rest. Suppose the density of positive charge along the line,
as measured in the laboratory frame, is A, in esu/cm, and the density
of negative charge is the same. 'The net density of charge on the line,
in the ``lab'' frame, is then zero. It follows that the electric field E
is zero in the lab frame. What we have here is equivalent to an uncharged
wire carrying a steady electric current. In a metallic wire
the negative charges only (electrons) would be moving, with the positive
charges at rest. We have taken a more symmetrical model only
to simplify the discussion slightly.

If you were to move at the same speed as the positive charges, you
would find their density along the line different. The situation is like

% p. 174
 

that in the capacitor of Fig. 5.19. In the lab frame the positive charge
distribution would be contracted in the x direction by the factor
(1 --- 1202/ c2)1/ 2, which would make it more dense than in a frame in
which the positive charges were at rest. Since we have specified that
the density in the lab frame shall be A, the density in the rest frame of
the positive charges must be less, namely >\(l - v02/c2)1/2. The
same holds for the linear density of negative charges in their rest
frame. This will be useful presently.

A stationary test charge q some distance r from the ``wire'' experiences
no force because the electric field is zero. We are interested
now in the force on a moving test charge. Suppose the charge q is set
in motion to the right with speed 12, in the lab frame. What force, as
observed in the lab frame, acts on it then? We have learned how such
questions can be answered. We get into a coordinate system moving
with the test charge q. In that ``particle'' frame the charge q is at
rest and the force on it is determined solely by the electric field in
that frame.

Why should there be an electric field in the ``particle'' frame when
there is none in the ``lab'' frame? The reason is that the magnitudes
of the line charge densities seen in the ``particle'' frame, which we
shall denote by M, and >\'_ respectively, are not equal. As observed
in the ``particle'' frame, the wire is charged! It has an excess of negative
charge in every unit of its 1ength.1'

1``'What about charge invariance?'' you may ask. We have always emphasized that
the total charge enclosed within some boundary is the same no matter what frame it is
measured in. In this case no boundary can enclose the total charge on the wire,which
stretches out to infinity; whatever happens at its ``ends'' does not matter here.

% p. 175
To work this out we need to know the speed of the positive charges
and the speed of the negative charges in the new frame of reference.
It is obvious that these are not going to be equal. In fact, since our
``particle'' frame has to move toward the right, as seen in the lab, it
tends to catch up with the positive charges and run away even faster
from the negatives. Figure 5.21 will help to identify the velocities we
have to consider. We are not going to make any approximations, so
we must use the relativistic velocity-addition formula to find the
speeds 0; and v', of the positives and negatives in the ``particle''
frame. These speeds are

\begin{equation}
\end{equation}

J _ D0 - D I _ D0 + D 
M -1 - vov/02 0- _ 1+ vov/02 ( )

Here the ,8 and y symbols will be handy. Let B0 = 130/0, 70 =
(1 - Bo2)""2'; B; = v'+/c, 7; = (1 - B'+2)""2', and so on. In
this notation Eq. 35 is written

\begin{equation}
\end{equation}

Br :80 ' B ,8! _ :80 + :8 

'T173 "1+#oB<

The two charge distributions will sufier different amounts of
Lorentz contraction---that is the crux of the matter. We can find the
linear density of the positive charges by starting with the density in
the positives' own rest frame and applying the factor for contraction
to the ``particle'' frame. We worked out earlier the density of positives
in their own rest frame; it was Ml - 1202/c2)1/2, or A/yo in our
new notation. The factor by which the distribution is linearly con-
tracted, as seen from the particle frame is 1/7;, so that the reciprocal
of this, y'+, is the factor by which the linear charge density is increased,
over the density in the rest frame of the positives. Thus the linear
density of positive charge in the particle frame must be:

\begin{equation}
\end{equation}

M. = Y'+ (1) (37)

Yo

Similarly, the density of negative charge in the particle frame is:

\begin{equation}
\end{equation}

>«_ = y:  (38)

We want to find the net density of line charge, A; - >\'_, eliminating
y; and y'_ from these equations by the use of Eq. 36. It looks as

 

% p. 176
though we might be getting into thick algebra, but the substitution
leads promptly to a delightful simplification, as follows:

\begin{equation}
\end{equation}

M ~ >«_ = %(v'+ - w.) (39)
From Eq. 36,
Y; _ Y; 2 ___1i _ .i1_i
_ B - /3 2 _ 30 + B 2
1 (10-303) \/1 l1+/203)
_ 1 - /303 _ 1 + 30.3
' \/1 - B02 - B2 + 30232 M1 - B02 - /22 + 30252
= 43°'; = -2.8oBvov <40)

\begin{equation}
\end{equation}

\/(1- Bo2)(1- 32)

Hence the net line charge density is:

\begin{equation}
\end{equation}

A; _ >v_ 2 -2}\,B0By = _ 2>':;"'° (41)

This linear electric charge in the particle frame gives the same electric
field as any line charge of the same density. We need only apply
Gauss's law to a cylinder enclosing the line to obtain the familiar
result that there is a radial electric field of magnitude

\begin{equation}
\end{equation}

2(>\; - >\_) _ _ 4}\yvvo
r - M2

E; = (42)

Hence the force on the positive test charge q is directed radially in-
ward, that is, in the positive y' direction in the particle frame, and its
magnitude is

4qy?\vv0

F' :
y r02

\begin{equation}
\end{equation}

(43)

This is a transverse force in the particle frame. As measured in the
lab frame, its magnitude will be different. According to our rules
for transforming forces (Eqs. 31), Fy : (1 /-y)F;. This knocks out
the y. We conclude that the magnitude of the force on the charge q,
moving with speed 0 parallel to the ``wire'' in the lab frame, is

_ 4q}\vv0

F
1' M2

\begin{equation}
\end{equation}

(44)

% p. 177
Now the quantity 2?\v0, which we could factor out of Eq. 44 if we
wanted to, is just the electric current in our ``wire,'' in esu per second.
That is, A00 is the rate of transport of positive charge to the right---
the amount of positive charge that the procession carries past a given
point per second---and the transport of negative charge to the left
contributes an equal amount to the current. Denoting this current
by I, the magnitude of the force on the moving charge is given by

_ Zqvl

F
re?

\begin{equation}
\end{equation}

(45)

It is a remarkable fact that the force on the moving test charge
does not depend separately on the velocity or density of the charge
carriers in the wire, but only on the combination that determines the
net charge transport. If we have a certain current, say 107 esu/sec,
which is the same as 3.3 milliamperes, it does not matter whether this
current is composed of high-energy electrons moving with 99 percent
of the speed of light, of electrons in a metal executing random thermal
motion with a slight superimposed drift in one direction, or of
charged ions in solution with positives moving one way, negatives
the other. Also, the force on the test charge is strictly proportional
to the velocity of the test charge. Our derivation was in no way
restricted to small velocities, either for the charge carriers in the wire
or for the test charge. Equation 45, is exact, with no restrictions.

Let's see how this explains the mutual repulsion of conductors
carrying currents in opposite directions, as shown in Fig. 5.1b at the
beginning of this chapter. Assume first that in each conductor there
are equal numbers of positive and negative charge carriers, moving
in opposite directions at the same speed. In the ``lab'' frame we have
something like Fig. 5.22a. Switching to a frame of reference which
moves along with the negative charges in conductor 1 and the positive
charges in conductor 2, we see the system as it appears in
Fig. 5.2212. In this frame conductor 1 has an excess of positive charge
in every unit length; hence it 'repels the positive charges in conductor
2. Similarly, the negatives in conductor 1 are repelled by the
excess negative charge in conductor 2. To find the forces on the remaining
charge carriers---the positives in 1 and the negatives in 2-
we transfer to the frame in which they are at rest, Fig. 5.220. Here
conductor 2 appears with an excess of positive charge, so the positives
in 1 must feel a repulsion. Similarly for the negatives in 2.

Thus every charge carrier, in its own rest frame, experiences a net
renulsion from the charge carriers in the other conductor. To find

% p. 178
 
 
 

the exact strength of the force in the lab frame we would have to
transform the forces back, as we did in the step from Eq. 43 to Eq. 44.
But that can't change the Sign of a force. Hence we are bound to

observe, in the laboratory, a repulsion of each conductor by the
other.

The model ust described could represent conduction in an electrolyte
or an ionized gas, although in general the two types of carriers
might have rather different speeds. In a metal, however, only the
negative charge carriers (electrons) move, while the corresponding
positive charges remain fixed in the crystal lattice. Two such wires
carrying currents in opposite directions are seen in the lab frame in
Fig. 5.23a. The wires being neutral, there is no electric force from
the opposite wire on the positive ions which are stationary in the lab
frame. Transferring to a frame in which one set of electrons is at
rest, Fig. 5.2319, we find that in the other wire the electron distribution
is Lorentz-contracted more than the positive-ion distribution. A
similar situation is found in Fig. 5.230. So this model, too, predicts
a repulsion between parallel currents flowing in opposite directions.
It illustrates, in a qualitative way, the earlier statement that the forces
between currents depend only on the amount of current flow, not on
how the charge is transported. (Problem 5.15 outlines a general
proof of that statement.)

In this chapter we have seen how the fact of charge invariance implies
forces between electric currents. That does not oblige us to
look on one fact as the cause of the other. These are simply two
aspects of electromagnetism whose relationship beautifully illustrates
the more general law: physics is the same in all inertial frames
of reference.

If we had to analyze every system of moving charges by transforming
back and forth among various coordinate systems, our task would
grow both tedious and confusing. There is a better way. The overall
effect of one current on another, or of a current on a moving charge,

can be described completely and concisely by introducing a new field,
the magnetic field.

\fi
