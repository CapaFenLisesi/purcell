\chapter{Electric fields around conductors}

\section{Conductors and insulators}

The earliest experimenters with electricity observed that substances
differed in their power to hold the ``Electrick Vertue.'' Some
materials could be easily electrified by friction and maintained in an
electrified state; others, it seemed, could not be electrified that way,
or did not hold the ``Vertue'' if they acquired it. Experimenters of
the early eighteenth century compiled lists in which substances were
classified as ``electricks'' or ``non-electricks.'' Around 1730, the important
experiments of Stephen Gray in England showed that the
``Electrick Vertue'' could be conducted from one body to another by
horizontal string, over distances of several hundred feet, provided
that the string was itself supported from above by silk 
threads.\footnote{The ``pack-thread'' he used for his string was doubtless a rather poor conductor
compared to metal wire, but good enough for transferring charge in electrostatic 
experiments. Gray found, too, that fine copper wire was a conductor, but mostly he
used the packthread for the longer distances.}
Once this distinction between conduction and nonconduction had
been grasped, the electricians of the day found that even a ``non-
electrick'' could be highly electrified if it was supported on glass or
suspended by silk threads. A spectacular conclusion of one of the
popular electric exhibitions of the time was likely to be the electrification
of a boy suspended by many silk threads from the rafters; his
hair stood on end and sparks could be drawn from the tip of his nose.

After the work of Gray and his contemporaries the elaborate lists
of electricks and non-electricks were seen to be, on the whole, a
division of materials into electrical insulators and electrical con-
ductors. This distinction is still one of the most striking and extreme
contrasts that nature exhibits. Common good conductors like
ordinary metals differ in their electrical conductivity from common
insulators like glass and plastics, by factors of the order of $10^{20}$. To
express it in a way the eighteenth-century experimenters like Gray
or Benjamin Franklin would have understood, a metal globe on a
metal post can lose its electrification in a millionth of a second; a
metal globe on a glass post could hold its ``Vertue'' for many years.
(To make good on the last assertion we should need to take some 
precautions beyond the capability of an eighteenth-century laboratory.
Can you suggest some of them?)


% p. 81
The electrical difference between a good conductor and a good
insulator is as vast as the mechanical difference between a liquid and
a solid. That is not entirely accidental. Both properties depend on
the mobility of atomic particles: in the electric case, the mobility of
the carriers of charge, electrons or ions; in the case of the mechanical
properties, the mobility of the atoms or molecules that make up the
structure of the material. To carry the analogy a bit further, we
know of substances whose fluidity is intermediate between that of a
solid and that of a liquid---substances such as tar or ice cream. Indeed
some substances---glass is a good example---change gradually
and continuously from a mobile liquid to a very permanent and rigid
solid with a few hundred degrees' lowering of the temperature. In
electrical conductivity, too, we find examples over the whole wide
range from ``good conductor'' to ``good insulator,'' and some substances
that can change conductivity over nearly as wide a range,
depending on conditions such as their temperature. A fascinating
and useful class of materials called semiconductors have this property
and even more curious properties.

Whether we call a material solid or liquid sometimes depends on
the time scale, and perhaps also on the scale of distances involved.
Natural asphalt seems solid enough if you hold a chunk in your hand.
Viewed geologically, it is a liquid, welling up from underground deposits
and even forming lakes. We may expect that, for somewhat
similar reasons, whether a material is to be regarded as an insulator
or a conductor will depend on the time scale of the phenomenon we
are interested in. We shall discover that for a rather simple and
general class of phenomena \emph{only} the time scale provides the criterion,
not the distance magnitudes. For immediate purposes, however, we
don't even need to formulate the distinction precisely.

\section{Conductors in the electrostatic field}

We shall look first at electrostatic systems involving conductors.
That is, we shall be interested in the \emph{stationary} state of charge and
electric field that prevails after all redistributions of charge have
taken place in the conductors. Any insulators present are assumed
to be perfect insulators. As we have already mentioned, quite
ordinary insulators come remarkably close to this idealization, so
the systems we shall discuss are not too artificial. The systems we
have in mind might be typified by some such example as this: Bring
in two charged metal spheres, insulated from one another and from
% p. 82
everything else. Fix them in positions relatively near one another.
What is the resulting electric field in the whole space surrounding
and between the spheres, and how is the charge that was on each
sphere distributed? We begin with a more general question; after
the charges have become stationary, what can we say about the electric
field inside conducting matter? 

In the static situation there is no further motion of charge. You
might be tempted to say that the electric field must then be zero
within conducting material. You might argue that if the field were
not zero, the mobile charge carriers would experience a force, would
be thereby set in motion, and thus we would not have a static situation
after all. Such an argument overlooks the possibility of \emph{other}
forces which may be acting on the charge carriers, and which would
have to be counterbalanced by an electric force to bring about a
stationary state. To remind ourselves that it is physically possible
to have other than electrical forces acting on the charge carriers we
need only think of gravity. A positive ion has weight; it experiences
a steady force in a gravitational field, and so does an electron; also,
the forces they experience are not equal. This is a rather absurd ex-
ample. We know that gravitational forces are utterly negligible on
an atomic scale. There are other forces at work, however, which we
may very loosely call ``chemical.'' In a battery and in many many
other theaters of chemical reaction, including the living cell, charge
carriers sometimes move \emph{against} the general electric field; they do so
because a reaction may thereby take place which yields more energy
than it costs to buck the field. One hesitates to call these forces non-
electrical, knowing as we do that the structure of atoms and molecules
and the forces between them can be explained in terms of
Coulomb's law and quantum mechanics. Still, from the viewpoint of
our \emph{classical} theory of electricity, they must be treated as quite 
extraneous. Certainly they behave very dilferently from the inverse-
square force upon which our theory is based. The general necessity
for forces that are in this sense nonelectrical was already foreshadowed
by our discovery in Chap. 2 that inverse-square forces alone
cannot make a stable, static structure.

The point is simply this: We must be prepared to find, in some
cases, unbalanced, non-Coulomb forces acting on charge carriers
inside a conducting medium. When that happens, the electrostatic
situation is attained when there is a finite electric field in the con-

ductor that just offsets the influence of the other forces, whatever
they may be.

% p. 83
Having issued this warning, however, we turn at once to the very
familiar and important case in which there is no such force to worry
about, the case of a homogeneous, isotropic conducting material.
In the interior of such a conductor, in the static case, we can state
confidently that the electric field must be zero.\footnote{In speaking of the electric field inside matter, we mean an average field, averaged
over a region large compared to the details of the atomic structure. We know, of course,
that very strong fields exist in all matter, including the good conductors, if we search
on a small scale near an atomic nucleus. It was an electric field, after all, that deffected
the alpha particles that Rutherford and Geiger and Marsden shot through the gold foil
(see Vol. 1, Chap. 15, Historical Note 1). The nuclear electric field does not contribute
to the average field in matter, ordinarily, because it points in one direction on one side
of a nucleus and in the opposite direction on the other side. Just how this average field
ought to be defined, and how it could be measured, are questions we need not face now.}
If it weren't, charges
would have to move. It follows that all regions inside the conductor,
including all points just below its surface, must be at the same po-
tential. For all we know, the potential could jump abruptly between
the inside and the outside of the conductor (see Prob. 3.21). But in
a homogeneous isotropic conductor, the only kind we are now con-
sidering, the jump would be the same everywhere on the conductor.
Outside the conductor, the electric field is not zero. The surface of
the conductor must be an equipotential surface of this field.

Imagine that we could change a material from insulator to conductor
at will. (It's not impossible---glass becomes conducting when
heated; any gas can be ionized by x rays.) In Fig. 3.1a is shown an
uncharged nonconductor in the electric field produced by two fixed
layers of charge. The electric field is the same inside the body as
outside. (A dense body such as glass would actually distort the field,
an effect we'll study in Chap. 9, but that is not important here.)
Now, in one way or another, let mobile charges (or \emph{ions}) be created,
making the body a conductor. Positive ions are drawn in one direction
by the field, negative ions in the opposite direction, as indicated
in Fig. 3.1b. They can go no farther than the surface of the conductor.
Piling up there, they begin themselves to create an electric
field inside the body which tends to \emph{cancel} the original field. And
in fact the movement goes on until that original field is \emph{precisely} canceled.
The final distribution of charge at the surface, shown in
Fig. 3.1c, is such that its field and the field of the fixed external
sources combine to give \emph{zero} electric field in the interior of the 
conductor. Because this ``automatically'' happens in every conductor,
it is really only the surface of a conductor that we need to consider
when we are concerned with the external fields.

% p. 84

With this in mind, let us see what can be said about a system of
conductors, variously charged, in otherwise empty space. In Fig. 3.2
we see some objects. Think of them, if you like, as solid pieces of
metal. They are prevented from moving by invisible insulators---
perhaps by Stephen Gray's silk threads. The total charge of each
object, by which we mean the net excess of positive over negative
charge, is fixed because there is no way for charge to leak on or off.
We denote it by $Q_k$, for the $k$th conductor. Each object can also be
characterized by a particular value $\pot_k$ of the electric potential function
$\pot$. We say that conductor 2 is ``at the potential $\pot_2$.'' With a
system like the one shown, where no physical objects stretch out to
infinity, it is usually convenient to assign the potential zero to points
infinitely far away. In that case $\pot_2$ is the work per unit charge required
to bring an infinitesimal test charge in from infinity and put
it anywhere on conductor 2. (Notice, by the way, that this is just
the kind of system in which the test charge needs to be kept small, a
point raised in Sec. 1.7.)

Because the surface of a conductor in Fig. 3.2 is necessarily a surface
of constant potential, the electric field, which is $-\grad\pot$, must
be \emph{perpendicular} to the surface at every point on the surface. Proceeding
from the interior of the conductor outward, we find at the surface
an abrupt change in the electric field; $\vc{E}$ is not zero outside the
surface, and it is zero inside. The discontinuity in $\vc{E}$ is accounted for
by the presence of a surface charge, of density $\sigma$, which we can relate
directly to $\vc{E}$ by Gauss's law. We can use a flat box enclosing a patch
of surface (Fig. 3.3) like the one we used in analyzing the charged
% p. 85
disk in Sec. 2.6. Here, there is \emph{no} flux through the ``bottom'' of the
box, which lies inside the conductor, and we conclude that $E_n=4\pi\sigma$,
where $E_n$ is the component of electric field normal to the surface. As
we have already seen, there is no other component in this case, the
field being always perpendicular to the surface. The surface charge
must account for the whole charge $Q_k$. That is, the surface integral
of a over the whole conductor must equal $Q_k$. In summary, we can
make the following statements about \emph{any} such system of conductors,
whatever their shape and arrangement:
%
\begin{framed}
\begin{align}
\begin{split}
&\text{$\pot = \pot_k$ at all points on the surface}\\
&\text{of the $k$th conductor} 
\end{split}
\end{align}
%
\begin{align}
\begin{split}
&\text{At any point just outside the conductor, $\vc{E}$ is}\\
&\text{perpendicular to the surface, and $E=4\pi\sigma$,}\\
&\text{where $\sigma$ is the local density of surface charge.} 
\end{split}
\end{align}
%
\begin{equation}
Q_k = \int_{S_k}\sigma\:\der a = \frac{1}{4\pi} \int_{S_k}\vc{E}\cdot\der\vc{a}
\end{equation}
\end{framed}
Because (2) uniquely relates $\vc{E}$ to $\sigma$, the local surface charge density,
you may be tempted to think of $\sigma$ as the source of $\vc{E}$. That would be
a mistake. $\vc{E}$ is the total field arising from all the charges in the
system, near and far, of which the surface charge is only a part. The
surface charge on a conductor is obliged to ``readjust itself'' until the
relation (2) is fulfilled. That the conductor presents a special case,
in contrast to other surface charge distributions, is brought out by
the comparison in Fig. 3.4.

Figure 3.5 shows the field and charge distribution for a simple
system like the one mentioned earlier. There are two conducting
spheres, a sphere of unit radius carrying a total charge $+1$ unit, the
other a somewhat larger sphere with total charge zero. Observe that
the surface charge density is not uniform over either of the con-
ductors. The sphere on theright, with total charge zero, has a negative
surface charge density in the region which faces the other sphere,
and a positive surface charge on the rearward portion of its surface.
The dashed curves in Fig. 3.5 indicate the equipotential surfaces, or
rather, their intersection with the plane of the figure. If we were to
go a long way out, we would find the equipotential surfaces becoming
nearly spherical, the field lines nearly radial, and the field would
begin to look very much like that of a point charge of magnitude $+1$,
which is the net charge on the entire system.

% p. 86
% p. 87
Figure 3.5 illustrates, at least qualitatively, all the features we
anticipated, but we have an additional reason for showing it. Simple
as the system is, the exact mathematical solution for this case cannot
be obtained in a straightforward way. Our Fig. 3.5 was constructed
from an approximate solution. In fact, the number of three-
dimensional geometrical arrangements of conductors which permit
a mathematical solution in closed form is lamentably small. One
does not learn much physics by concentrating on the solution of the
few neatly soluble examples. Let us instead try to understand the
general nature of the mathematical problem such a system presents.

% p. 88

\section{The general electrostatic problem; uniqueness theorem}

We can state the problem in terms of the potential function $\pot$,
for if $\pot$ can be found we can at once get $\vc{E}$ from it. Everywhere outside
the conductors $\pot$ has to satisfy the partial differential equation
we met in Chap. 2, Laplace's equation: $\nabla^2\pot=0$. Written out in
Cartesian coordinates, Laplace's equation reads,
\begin{equation}
   \frac{\partial^2\pot}{\partial x^2}
  +\frac{\partial^2\pot}{\partial y^2}
  +\frac{\partial^2\pot}{\partial z^2} = 0
\end{equation}
The problem is to find a function that satisfies Eq. 4 and also meets
the specified conditions on the conducting surfaces. These conditions
might have been set in various ways. It might be that the potential
of each conductor, $\pot_k$, is fixed or known. (In a real system
the potentials may be fixed by permanent connections to batteries
or other constant-potential ``power supplies.'') Then our solution
$\pot(x,y,z)$ has to assume the correct value at all points on each of the
surfaces. These surfaces in their totality \emph{bound} the region in which
$\pot$ is defined, if we include a large surface ``at infinity,'' where we
require $\pot$ to approach zero. Sometimes the region of interest is
totally enclosed by a conducting surface; then, we can assign this
conductor a potential and ignore anything outside it. In either case,
we have a typical \emph{boundary-value problem}, in which the value the
function has to assume on the boundary is specified for the entire
boundary.

One might, instead, have specified the total charge on each conductor,
$Q_k$. (We could not specify arbitrarily all charges and po-
tentials; that would overdetermine the problem.) With the charges
specified, we have in effect fixed the value of the surface integral of
$\grad \pot$ over the surface of each conductor. This gives the mathematical
problem a slightly different aspect. Or one can ``mix'' the
two kinds of boundary conditions.

A general question of some interest is this: With the boundary
conditions given in some way, does the problem have no solution,
one solution, or more than one solution? We shall not try to answer
this question in all the forms it can take, but one important case will
show how such questions can be dealt with, and will give us a useful
result. Suppose the potential of each conductor, $\pot_k$, has been
specified, together with the requirement that $\pot$ approach zero at
infinite distance, or on a conductor which encloses the system. We
shall prove that this boundary-value problem has no more than one
% p. 89
solution. It seems obvious, as a matter of physics, that it has a solution,
for if we should actually arrange the conductors in the prescribed
manner, connecting them by infinitesimal wires to the proper
potentials, the system would have to settle down in \emph{some} state. How-
ever, it is quite a different matter to prove mathematically that a solution
always exists and we shall not attempt it. Instead, we assume
that there is a solution $\pot(x,y,z)$ and show that it must be unique.
The argument, which is typical of such proofs, runs as follows.

Assume there is another function $\psi(x,y,z)$ which is also a solution
meeting the same boundary conditions. Now Laplace's equation is
\emph{linear}. That is, if $\psi$ and $\pot$ satisfy Eq. 4, then so does $\pot+\psi$ or any
linear combination such as $c_1\pot+c_2\psi$, where $c_1$ and $c_2$ are constants.
In particular, the difference between our two solutions, $\pot-\psi$, must
satisfy Eq. 4. Call this function $W$:
\begin{equation}
  W(x,y,z) = \pot(x,y,z)-\psi(x,y,z)
\end{equation}
Of course, $W$ does \emph{not} satisfy the boundary conditions. In fact, at
the surface of every conductor, $W$ is zero, because $\psi$ and $\pot$ take on
the same value, $\pot_k$, at the surface of a conductor $k$. Thus $W$ is a solution
of $\emph{another}$ electrostatic problem, one with the same conductors
but with all conductors held at zero potential. We can now assert
that if this is so, $W$ must be zero at all points in space. For if it is not,
it must have either a maximum or a minimum somewhere---
remember that $W$ is zero at infinity as well as on all the conducting
boundaries. If $W$ has an extremum at some point $P$, consider a
sphere centered on that point. As we saw in Chap. 2, the average
over a sphere of a function that satisfies Laplace's equation is equal
to its value at the center. This could not be true if the center is a
maximum or minimum. Thus $W$ cannot have a maximum or minimum;
it must therefore be zero everywhere. It follows that $\psi=\pot$
everywhere, that is, there can be only \emph{one} solution of Eq. 4 that
satisfies the prescribed boundary conditions.

We can now demonstrate easily another remarkable fact. \emph{In the
space inside a hollow conductor of any shape whatever, if that space
itself is empty of charge, the electric field is zero.} This is true whatever
the field may be outside the conductor. We are already familiar
with the fact that the field is zero inside an isolated uniform spherical
shell of charge, just as the gravitational field inside the shell of a
hollow spherical mass is zero. The theorem we just stated is, in a
% p. 90
way, more surprising. Consider the closed metal box shown partly
cut away in Fig. 3.6. There are charges in the neighborhood of the
box, and the external field is approximately as depicted. There is a
highly nonuniform distribution of charge over the surface of the box.
Now the field everywhere in space, \emph{including the interior of the box},
is the sum of the field of this charge distribution and the fields of the
external sources. It seems hardly credible that the surface charge
has so cleverly arranged itself on the box that its field precisely cancels
the field of the external sources at every point inside the box. Yet
this must have happened, as we can prove in a few sentences.

The potential function inside the box, $\pot(x,y,z)$, must satisfy
Laplace's equation. The entire boundary of this region, namely the
box, is an equipotential, so we have $\pot=\pot_0$, a constant everywhere
on the boundary. One solution is obviously $\pot=\pot_0$ throughout
% p. 91
the volume. But there can be only one solution, according to our
uniqueness theorem, so this is it. ``$\pot$ = constant'' implies $\vc{E} = 0$,
because $E = -\grad \pot$.

The absence of electric field inside a conducting enclosure is useful
as well as theoretically interesting. It is the basis for electrical
shielding. For most practical purposes the enclosure does not need
to be completely tight. If the walls are perforated with small holes,
or made of metallic screen, the field will be extremely weak except
in the immediate vicinity of a hole. A metal pipe with open ends,
if it is a few diameters long, very effectively shields the space inside
that is not close to either end. We are considering only static fields
of course, but for slowly varying electrical fields these remarks still
hold.

\section{Some simple systems of conductors}

In this section we shall investigate a few particularly simple arrangements
of conductors. We begin with two concentric metal
spheres, of radii $R_1$ and $R_2$, carrying total charges $Q_1$ and $Q_2$ respectively
(Fig. 3.7). This situation presents no new challenge. It is
obvious from symmetry that the charge on each sphere must be distributed
uniformly, so our example really belongs back in Chap. 1!
Outside the larger sphere the field is that of a point charge of magnitude
$Q_1 + Q_2$, so $\pot_1$, the potential of the outer sphere, is
\begin{equation*}
  \frac{Q_1+Q_2}{R}
\end{equation*}
The potential of the inner sphere is given by
\begin{align}
\begin{split}
  \pot_2 &= \frac{Q_1+Q_2}{R} + \int_{R_1}^{R_2} -\frac{Q_2}{r^2}\der r
         = \frac{Q_1}{R_1}+\frac{Q_2}{R_1}+\frac{Q_2}{R_2}-\frac{Q_2}{R_1} \\
       &=\frac{Q_1}{R_1}+\frac{Q_2}{R_2}
\end{split}
\end{align}
$\pot_2$ is also the potential at all points inside the inner sphere. We could
have found $\pot_2 = (Q_1/R_1) + (Q_2/R_2)$ by simple superposition:
$Q_1/R_1$ is the potential inside the larger sphere if it alone is present,
$Q_2/R_2$ the potential inside the inner sphere if it alone is present. If
the spheres carried equal and opposite charges, $Q_1 = - Q_2$, only the
space between them would have a nonvanishing electric field.
% p. 92
About the simplest system in which the mobility of the charges in
the conductor makes itself evident is the point charge near a conducting
plane. Suppose the $xy$ plane is the surface of a conductor
extending off to infinity. Let's assign this plane the potential zero.
Now bring in a positive charge $Q$ and locate it $h$ cm above the plane
on the $z$ axis, as in Fig. 3.8a. What sort of field and charge distribution
can we expect? We expect the positive charge $Q$ to attract negative
charge, but we hardly expect the negative charge to pile up in an
infinitely dense concentration at the foot of the perpendicular from $Q$.
Why not? Also, we remember that the electric field is always perpendicular
to the surface of a conductor, at the conductor's surface.
Very near the point charge $Q$, on the other hand, the presence of the
conducting plane can make little difference; the field lines must start
out from $Q$ as if they were leaving a point charge radially. So we
might expect something qualitatively like Fig. 3.8b, with some of the
details still a bit uncertain. Of course the whole thing is bound to
be quite symmetrical about the $z$ axis.

But how do we really solve the problem? The answer is, by a trick,
but a trick that is both instructive and frequently useful. We find an
easily soluble problem whose solution, or a piece of it, can be made
to fit the problem at hand. Here the easy problem is that of two equal
and opposite point charges, $Q$ and $- Q$. On the plane which bisects
the line joining the two charges, the plane indicated in cross section
by the line $AA$ in Fig. 3.80, the electric field is everywhere perpendicular
to the plane. If we make the distance of $Q$ from the plane
agree with the distance it in our original problem, the upper half of
the field in Fig. 3.8c meets all our requirements: The field is perpendicular
to the plane of the conductor and in the neighborhood of $Q$
it approaches the field of a point charge.

The boundary conditions here are not quite those that figured in
our uniqueness theorem in the last section. The potential of the conductor
is fixed, but we have in the system a point charge, at which
the potential approaches infinity. We can regard the point charge
as the limiting case of a small spherical conductor on which the total
charge $Q$ is fixed. For this ``mixed'' boundary condition---potentials
given on some surfaces, total charge on others---a uniqueness
theorem also holds. If our ``borrowed'' solution fits as well as this,
it must be the solution.

Figure 3.9 shows the final solution for the field above the plane,
with the density of the surface charge suggested. We can calculate
the field strength and direction at any point by going back to the two-
% p. 93
charge problem, Fig. 3.80, and using Coulomb's law. Consider a
point on the surface, a distance r from the origin. The square of its
distance from $Q$ is $r^2 + h^2$, and the $z$ component of the field of $Q$,
at this point, is $-Q \cos \theta/(r^2 + h^2)$. The ``image charge,'' $-Q$,
below the plane contributes an equal $z$ component. Thus the electric
field here is given by:
\begin{equation}
  E_z = \frac{-2Q}{r^2+h^2}\cos\theta - \frac{-2Q}{r^2+h^2}\cdot\frac{h}{(r^2+h^2)^{1/2}}
       = \frac{-2Qh}{(r^2+h^2)^{3/2}}
\end{equation}
This tells us the surface charge density $\sigma$:
\begin{equation}
  \sigma = \frac{E_z}{4\pi}  = \frac{-Qh}{2\pi(r^2+h^2)^{3/2}}
\end{equation}
The total surface charge ought to amount to $-Q$. Just as a check,
we could integrate over the surface and see if it is:
\begin{align}
\begin{split}
  \text{Total surface charge} &= \int_0^\infty \sigma\cdot2\pi r\:\der r \\
           &= -Q \int_0^\infty \frac{hr\:\der r}{(r^2+h^2)^{3/2}} = -Q
\end{split}
\end{align}
% p. 94
The method of solution used here has traditionally been called
the \intro{method of images}. One thinks of the fictitious negative charge
located a distance $h$ below the plane of the conductor, toward which
the field lines appear to plunge, as the ``image'' of the point charge $Q$,
something like the virtual image behind a mirror. The electric force
which acts on the charge $Q$, owing to the attraction of the surface
charge, is equal to the force that a charge $-Q$ in the image position
would cause. Note that the actual origin of this force is the surface
charge.

However, the mirror analogy is really not very close, and not very
helpful. One might better describe the method as an example of a
more general approach which could be called ``fitting the boundaries
to the solution.'' To show what we mean by this, let us look at some
of the equipotential surfaces in the field of two equal and opposite
charges shown in Fig. 3.10a. The plane was only one of these. The
others are closed surfaces, none of them quite spherical, but surfaces
we could locate by an elementary calculation if we had to. If now
we were to take any two of these surfaces, construct metal shells of
exactly that shape and similarly placed with respect to one another,
as in Fig. 3.10b, we would have already in hand the exact solution
for the electrostatic field of two such charged conductors! It would
be the corresponding piece of the two-charge field. Unfortunately,
no one is likely to come around with electrodes of precisely that
shape, looking for a method of solution, although as an approximate
solution for spheres it might attract some customers.

We could proceed to examine the equipotential surfaces of other
simple systems, looking for examples that might be more useful.
Perhaps we should call the method ``a solution in search of a
problem.'' A good example of its utility can be studied in Prob. 3.22.
The situation was well described by Maxwell: ``It appears, therefore,
that what we should naturally call the inverse problem of determining
the forms of the conductors when the expression for the potential
is given is more manageable than the direct problem of determining
the potential when the form of the conductors is given.''\footnote{
James Clerk Maxwell, Treatise on Electricity and Magnetism, vol. I, chap. VII.
(3d ed., Oxford University Press, 1891; reprint ed., Dover, New York, 1954). Every
student of physics ought sometime to look into Maxwell's book. Chapter VII is a good
place to dip in while we are on the present subject. At the end of Volume I you will find
some beautiful diagrams of electric fields, and shortly beyond the quotation we have just
given, Maxwe11's reason for presenting these figures. One may suspect that he also took
delight in their construction and their elegance.}

% p. 95
\section{Capacitors and capacitance}

Two similar flat conducting plates are arranged parallel to one
another, separated by a distance $s$, as in Fig. 3.l1a. Let the area of
each plate be $A$ and suppose that there is a charge $Q$ on one plate
and $-Q$ on the other. $\pot_1$ and $\pot_2$ are the values of the potential at
each of the plates. Figure 3.11b shows in cross section the field lines
in this system. Away from the edge, the field is very nearly uniform
in the region between the plates. Treating it as uniform, its magnitude
must be $(\pot_1-\pot_2)/s$. The corresponding density of the surface
charge on the inner surface of one of the plates is
\begin{equation}
  \sigma = \frac{E}{4\pi} = \frac{\pot_1-\pot_2}{4\pi s}
\end{equation}
% p. 96
If we may neglect the actual variation of $E$ and therefore of $\sigma$ which
occurs principally near the edge of the plates, we can write a simple
expression for the total charge on one plate:
\begin{equation}
  Q = A \: \frac{\pot_1-\pot_2}{4\pi s}
\end{equation}
We should expect Eq. 11 to be more nearly accurate the smaller
the ratio of the plate separation $s$ to the lateral dimension of the
plates. Of course, if we were to solve exactly the electrostatic problem,
edge and all, for a particular shape of plate, we could replace
Eq. 11 by an exact formula. To show how good an approximation
Eq. 11 is, there are listed in Fig. 3.12 values of the correction factor $f$
by which the charge $Q$ given by Eq. 11 differs from the exact result,
in the case of two conducting disks at various separations. The
total charge is always a bit greater than Eq. 11 would predict. That
seems reasonable, as we look at Fig. 3.11b, for there is evidently an
extra concentration of charge at the edge, and even some charge on
the outer surfaces, near the edge.

We are not concerned now with the details of such corrections but
with the general properties of a two-conductor system. Our pair of
plates is one example of a common element in electrical systems, the
\intro{capacitor}. A capacitor is simply two conductors close together, at
different potentials, and carrying different charges. We are interested
in the relation between the charge $Q$ on one of the plates and
% p. 97
the potential difference between them. For the particular system
to which Eq. 11 applies, the quotient $Q/(\pot_1-\pot_2)$ is $A/4\pi s$. Even
if this is only approximate, it is clear that the exact formula will depend
only on the size and geometrical arrangement of the plates.
That is, for a fixed pair of conductors, the ratio of charge to potential
difference will be a constant. We call this constant the \intro{capacitance}
of the capacitor and denote it usually by $C$.
\begin{equation}
  Q = C(\pot_1-\pot_2)
\end{equation}
Thus the capacity of the parallel-plate capacitor, with edge fields
neglected, is given by
\begin{equation}
  C = \frac{\text{$A$ (in $\cmunit^2$)}}{\text{$4\pi s$ (in cm)}}
\end{equation}
In the CGS units we are using, with charge in esu and potential in
statvolts, capacitance has the dimensions of length and the unit of
capacity can simply be called the \emph{centimeter}. Two plates each
$100\ \cmunit^2$ in area, and 1 mm apart, form a capacitor with a capacitance
of $100/(4\pi)(0.1)\ \cmunit$, or 79.5 cm.

In the other system of units that we have to be familiar with, the
``practical'' system, electric charge is expressed in coulombs, or
ampere-seconds, and the unit of potential is the volt. Unit capacitance
in this system implies a capacitor that has one coulomb on
one of its plates when the potential difference is one volt. The unit
is called the \emph{farad}. To relate the farad to the CGS unit of capacitance,
the centimeter, remember that 1 volt = 1/300 statvolt, and
1 coulomb = $3 \times 10^9$ esu. Thus
\begin{align*}
  1\ \zu{farad} &= \frac{1\ \zu{coulomb}}{1\ \zu{volt}} = \frac{3 \times 10^9\ \zu{esu}}{(1/300)\ \zu{statvolt}} \\
                &= 9\times 10^{11}\frac{\zu{esu}}{\zu{statvolt}} = 9\times 10^{11}\ \cmunit
\end{align*}
A one-farad capacitor would be gigantic. To make it with plates
1 mm apart would require each plate to have an area of about
$100\ \zu{km}^2$.\footnote{Of course there are more compact ways of making a capacitor of high capacitance!
You can buy a microfarad of capacitance at any electrical shop and easily carry it home.
In biological material the wall of a cell forms an electrically insulating layer separating
the interior of the cell from the external ffuid environment. This membrane behaves
electrically like a capacitance of, typically, $1\ \mu\zu{F}/\cmunit^2$ of membrane area. What ``plate
separation'' does this imply? (Actually, the capacitance depends also on the dielectric
constant, that is, the electrical polarizability, of the medium between the plates. This
will be discussed in Chap. 9.)}
Because this is one place where the ``practical'' units
% p. 98
happen to come out a rather impractical size, one commonly uses
the \emph{microfarad} ($\mu\zu{F}$) and the
\emph{picofarad} (pF).
% Got rid of archaic description of micromicrofarad as a unit of capacitance.
Notice
that the latter unit is about the same magnitude as our CGS unit of capacity, the
centimeter. The word condenser, by the way, is an older term for
capacitor.

Any pair of conductors, regardless of shape or arrangement, can
be considered a capacitor. It just happens that the parallel-plate
capacitor is a common arrangement and one for which an approximate
calculation of the capacitance is very easy. Figure 3.13 shows
two conductors, one inside the other. We can call this arrangement,
too, a capacitor. As a practical matter, some mechanical support
for the inner conductor would be needed, but that does not concern
us. Also, to convey electric charge to or from the conductors we
would need leads which are themselves conducting bodies. Since
a wire leading out from the inner body, numbered 1, necessarily
crosses the space between the conductors, it is bound to cause some
perturbation of the electric field in that space. To minimize this we
may suppose the lead wires to be extremely thin. Or we might
imagine the leads removed before the potentials are determined.

In this system we can distinguish three charges: $Q_1$, the total
charge on the inner conductor; $Q_2^{(i)}$, the amount of charge on the
inner surface of the outer conductor; $Q_2^{(o)}$ the charge on the outer
surface of the outer conductor. Observe first that $Q_2^{(o)}$ must equal
$-Q_1$. We know this because a surface such as $S$, in Fig. 3.13 encloses
both these charges and no others and the flux through this
surface is zero. The flux is zero because on the surface $S$, lying, as
it does, in the interior of a conductor, the electric field is zero.

Evidently the value of $Q_1$ will uniquely determine the electric field
within the region between the two conductors, and thus will determine
the difference between their potentials, $\pot_1-\pot_2$. For that
reason, if we are considering the two bodies as ``plates'' of a capaci-
tor, it is only $Q_1$, or its counterpart $Q_2^{(o)}$, that is involved in determining
the capacitance. The capacitance is:
\begin{equation}
  C = \frac{Q_1}{\pot_1-\pot_2}
\end{equation}
$Q_2^{(o)}$, on which $\pot_2$ itself depends, is here irrelevant. In fact, the complete
enclosure of one conductor by the other makes the capacitance

% p. 99
independent of everything outside. If we were faced instead with
two unsymmetrical capacitor plates not so enclosed---something like
Fig. 3.14, say,---we might be puzzled by the question: What is the
charge which plays the role of $Q_1$, in terms of which the capacitance
should be defined? The answer is that it is the amount of charge
that would have to be transferred from conductor 1 to conductor 2
(thus keeping the sum of the charges on the two conductors con-
stant) to cause their potentials to become equal.

\section{Potentials and charges on several conductors}

We have been skirting the edge of a more general problem, the
relations among the charges and potentials of any number of conductors
of some given configuration. The two-conductor capacitor
is just a special case. It may surprise you that anything useful can
be said about the general case. In tackling it, about all we can use
is the Uniqueness Theorem and the Superposition Principle. To
have something definite in mind, consider three separate conductors,
all enclosed by a conducting shell, as in Fig. 3.15. The potential of
this shell we may choose to be zero; with respect to this reference the
potentials of the three conductors, for some particular state of the
system, are $\pot_1$, $\pot_2$, and $\pot_3$. The Uniqueness Theorem guarantees
that with $\pot_1$, $\pot_2$, and $\pot_3$ given, the electric field is determined throughout
the system. It follows that the charges $Q_1$, $Q_2$, and $Q_3$ on the
individual conductors are likewise uniquely determined.

We need not keep account of the charge on the inner surface of
the surrounding shell, since it will always be $-(Q_1 + Q_2 + Q_3)$.
If you prefer, you can let ``infinity'' take over the role of this shell,
imagining the shell to expand outward without limit. We have kept
it in the picture because it makes the process of charge transfer easier
to follow, for some people, if we have something to connect to.

Among the possible states of this system are ones with $\pot_2$ and $\pot_3$
both zero. We could enforce this condition by connecting conductors
2 and 3 to the zero-potential shell, as indicated in Fig. 3.15a.
As before, we may suppose the connecting wires are so thin that any
charge residing on them is negligible. Of course, we really do not
care how the specified condition is brought about. In such a state,
which we shall call state I, the electric field in the whole system and
the charge on every conductor is determined uniquely by the value
of $\pot_1$. Moreover, if $\pot_1$ were doubled that would imply a doubling
% p. 100
of the field strength everywhere, and hence a doubling of each of
the charges $Q_1$, $Q_2$, and $Q_3$. That is, with $\pot_2 = \pot_3 = 0$, each of the
three charges must be proportional to $\pot_1$. Stated mathematically:
\begin{equation}
  \left.
  \begin{array}{cc}
    \text{\emph{State I}} \\
    \pot_2 = \pot_3 = 0
  \end{array}
  \right\}
  Q_1=C_{11}\pot_1; \qquad Q_2=C_{21}\pot_1; \qquad Q_3=C_{31}\pot_1
\end{equation}
The three constants, $C_{11}$, $C_{21}$, and $C_31$, can depend only on the shape
and arrangement of the conducting bodies.

In just the same way we could analyze states in which $\pot_1$ and $\pot_3$
are zero, calling such a condition state II (Fig. 3.15b). Again, we
must find a linear relation between the only nonzero potential, $\pot_2$ in
this case, and the various charges:
\begin{equation}
  \left.
  \begin{array}{cc}
    \text{\emph{State II}} \\
    \pot_1 = \pot_3 = 0
  \end{array}
  \right\}
  Q_1=C_{12}\pot_2; \qquad Q_2=C_{22}\pot_2; \qquad Q_3=C_{32}\pot_2
\end{equation}
Finally, when $\pot_1$ and $\pot_2$ are held at zero, the field and the charges
are proportional to $\pot_3$:
\begin{equation}
  \left.
  \begin{array}{cc}
    \text{\emph{State III}} \\
    \pot_1 = \pot_2 = 0
  \end{array}
  \right\}
  Q_1=C_{13}\pot_3; \qquad Q_2=C_{23}\pot_3; \qquad Q_3=C_{33}\pot_3
\end{equation}

Now the superposition of three states like I, II, and III is also a
possible state. The electric field at any point is the vector sum of
the electric fields at that point in the three cases, while the charge on
a conductor is the sum of the charges it carried in the three cases.
In this new state the potentials are $\pot_1$, $\pot_2$, and $\pot_3$, none of them necessarily
zero. In short, we have a completely general state. The
relation connecting charges and potentials is obtained simply by
adding Eqs. 15 through 17:
\begin{align}
\begin{split}
  Q_1 = C_{11}\pot_1 + C_{12}\pot_2 + C_{13}\pot_3 \\
  Q_2 = C_{21}\pot_1 + C_{22}\pot_2 + C_{23}\pot_3 \\
  Q_3 = C_{31}\pot_1 + C_{32}\pot_2 + C_{33}\pot_3
\end{split}
\end{align}

It appears that the electrical behavior of this system is characterized
by the nine constants $C_{11}$, $C_{12}$, \ldots , $C_{33}$. In fact only six constants
are necessary, for it can be proved that in \emph{any} system
$C_{12} = C_{21}$, $C_{13} = C_{31}$, and $C_{23} = C_{32}$.  Why this should be so is not
obvious. Problem 3.27 will suggest a proof based on conservation
% p. 101
of energy, but for that purpose you will need an idea developed in
Sec. 3.7. The $C$'s in Eqs. 18 are called the coefficients of capacitance.
It is clear that our argument would extend to any number of conductors.
Incidentally, what was defined earlier as the capacitance
of a two-plate capacitor, is not the same as $C_{11}$ (or $C_{22}$ or $C_{12}$), but
it is of course related to them.

A set of equations like (18) can be solved for the $\pot$'s in terms of
the $Q$'s. That is, there is an equivalent set of linear relations of the
form:
\begin{align}
\begin{split}
  \pot_1 = P_{11}Q_1 + P_{12}Q_2 + P_{13}Q_3 \\
  \pot_2 = P_{21}Q_1 + P_{22}Q_2 + P_{23}Q_3 \\
  \pot_3 = P_{31}Q_1 + P_{32}Q_2 + P_{33}Q_3
\end{split}
\end{align}
The $P$'s are called the potential coefficients; they could be computed
from the $C$'s, or vice versa.

We have here a simple example of the kind of relation we can
expect to govern any \emph{linear} physical system. Such relations turn up
in the study of mechanical structures (connecting the strains with
the loads), in the analysis of electrical circuits (connecting voltages
and currents), and generally speaking, wherever the superposition
principle can be applied.

\section{Energy stored in a capacitor}

Consider a capacitor of capacitance $C$, with a potential difference
$\pot_{12}$ between the plates. The charge $Q$ is equal to $C\pot_{12}$. There is a
charge $Q$ on one plate and $-Q$ on the other. Suppose we increase
the charge from $Q$ to $Q + \der Q$ by transporting a positive charge $\der Q$
from the negative to the positive plate, working against the potential
difference $\pot_{12}$. The work that has to be done is 
$\der W = \pot_{12} \der Q = Q \der Q/C$.
Therefore to charge the capacitor starting from the uncharged
state to some final charge Q, requires the work
\begin{equation}
  W = \frac{1}{C}\int_{Q=0}^{Q_f} Q\;\der Q = \frac{Q_F^2}{2C}
\end{equation}
This is the energy U which is ``stored'' in the capacitor. It can also
be expressed by
\begin{equation}
  U = \frac{1}{2}C\pot_{12}^2
\end{equation}
% p. 102
For the parallel-plate capacitor with plate area $A$ and separation $s$
we found the capacitance $C = A /4\pi s$ and the electric field $E = \pot_{12}/s$.
Hence Eq. 21 is also equivalent to
\begin{equation}
  U = \frac{1}{2}\left(\frac{A}{4\pi s}\right)(Es)^2 = \frac{E^2}{8\pi}\cdot As 
     = \frac{E^2}{8\pi}\cdot\text{volume}
\end{equation}
This agrees with our general formula, Eq. 2.36, for the energy stored
in an electric field.\footnote{All this applied to
the ``vacuum capacitor'' consisting of conductors with empty space
in between. As you know from the laboratory, most capacitors used in electrical circuits
are filled with an insulator or ``dielectric.'' We are going to study the effect that has in
Chap. 9.}

\section{Other views of the boundary-value problem}

It would be wrong to leave the impression that there are no general
methods for dealing with the Laplacian boundary-value problem.
Although we cannot pursue this question much further, we
shall mention three useful and interesting approaches which you are
likely to meet in future study of physics or applied mathematics.

First, an elegant method of analysis, called \intro{conformal mapping},
is based on the theory of functions of a complex variable. Unfortunately
it applies only to two-dimensional systems. These are systems
in which $\pot$ depends only on $x$ and $y$, for example, all conducting
boundaries being cylinders (in the general sense) with elements
running parallel to $z$. Laplace's equation then reduces to
\begin{equation}
  \frac{\partial^2\pot}{\partial x^2}+\frac{\partial^2\pot}{\partial y^2}=0
\end{equation}
with boundary values specified on some lines or curves in the xy
plane. Many systems of practical interest are like this, or sufficiently
like this to make the method useful, quite apart from its intrinsic
mathematical interest. For instance, the exact solution for the potential
around two long parallel strips is easily obtained by the
method of conformal mapping. The field lines and equipotentials
are shown in a cross-section plane in Fig. 3.16. This provides us
with the edge field for any parallel-plate capacitor in which the edge
is long compared to the gap. The field shown in Fig. 3.11b was
copied from such a solution. You will be able to apply this method
after you have studied in more advanced mathematics functions of
a complex variable.

% p. 103
Second, we mention a numerical method for finding approximate
solutions of the electrostatic potential with given boundary values.
Surprisingly simple and almost universally applicable, this method
is based on that special property of harmonic functions with which
we are already familiar: The value of the function at a point is equal
to its average over the neighborhood of the point. In this method
the potential function $\pot$ is represented by values at an array of discrete
points only, including discrete points on the boundaries. The
values at nonboundary points are then adjusted until each value is
equal to the average of the neighboring values. In principle one
could do this by solving a large number of simultaneous equations---
as many as there are interior points. But an approximate solution
% p. 104
can be obtained much more simply by systematically changing each
value to agree with the average of its neighbors and repeating this
process until the changes become negligibly small. This is called
the \intro{relaxation method}. The only limitation is the patience of the
computer, and that has been eliminated by high-speed computing
machines, to which this method is ideally suited. If you want to see
how it works, Probs. 3.29 and 3.30 will provide an introduction.

A third method for approximate solution of boundary-value problems
is the \intro{variational method}. This involves an idea that we shall
meet in many parts of physics, ranging from Newtonian dynamics
through optics to quantum mechanics. In electrostatics the principle
turns up in the following form: We have already learned that
the total energy associated with an electrostatic field can be expressed
by
\begin{equation}
  U = \frac{1}{8\pi}\int E^2\;\der v
\end{equation}
If you worked Prob. 2.19, you found that in that very simple case
the charge on a conducting boundary of constant potential (there
consisting of the two spheres connected by a wire) distributed itself
so as to \emph{minimize} the energy stored in the entire field. This happens
quite generally. That is, with any system of conductors, at various
fixed potentials, charge will flow around on each conductor until the
energy stored in the field is as small as it can be. This is almost 
self-evident if we note that with every reduction of the total energy in
the field, energy becomes available to promote the flow 
of charge.\footnote{In putting it this way we are thinking of the flow of charge as involving some dissipation
of energy. It usually does. If it didn't, a system not originally in equilibrium could
not get rid of its energy in order to reach the equilibrium state. What do you think would
happen in that case?}
The reason why the surface of water in a bowl is flat is essentially
the same.

Now consider the potential function $\pot(x,y,z)$ in some region enclosed
by a number of boundaries at given potentials. The correct
$\pot(x,y,z)$, that is, the solution of $\nabla^2\pot = 0$ which fits the given boundary
potentials, is distinguished from all other functions which fit the
boundary conditions but \emph{don't} satisfy Laplace's equation, such as
$\psi(x,y,z)$, by the fact that the stored energy is \emph{less} for $\pot$ than for $\psi$.
Expressing the energy in terms of $\pot$, as we did in Eq. 2.38,
\begin{equation}
  U = \frac{1}{8\pi}\int |\nabla\pot|^2\;\der v
\end{equation}

% p. 105

We can now state the boundary-value problem in a new way,
without mentioning the Laplacian. The potential function $\pot$ is that
function which \emph{minimizes the integral of Eq. 25 as compared with any
other functions fitting the same boundary values}. So one way of getting
at least an approximate solution to a given boundary-value
problem is to try a lot of functions, requiring only that they take on
the assigned values at the boundary, and pick the one that gives the
lowest value for $U$. Or we could try a function with an adjustable
parameter or two, and turn these mathematical ``knobs'' to minimize
$U$. The method is especially good for estimating the energy
itself, often the most important unknown quantity. Since $U$ is a
minimum for the correct $\pot$, it is very insensitive to departures from
correctness in $\pot$. Problem 3.32 will show the simplicity and accuracy
of the variational method.

More significant to us than its utility in calculations is the fact that
this variational principle represents an \emph{alternative formulation} of the
basic law of the electrostatic field. The reformulation of physical
laws as variational principles has often proved to be a fruitful and
enlightening enterprise. Professor R. P. Feynman, known for his
own brilliant work along such lines, has given a lively elementary
exposition of variational ideas in one chapter of the book \emph{The 
Feynman Lectures in Physics}, vol. II, chap. 19 (Addison-Wesley, Reading,
Mass., 1964).
